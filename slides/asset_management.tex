% Choose one to switch between slides and handout
%\documentclass[]{beamer}
\documentclass[handout]{beamer}

% Video Meta Data
\title{Smart Contracts and Decentralized Finance}
\subtitle{Asset Management}
\author{Prof. Dr. Fabian Schär}
\institute{University of Basel}

% Config File
% Packages
\usepackage[utf8]{inputenc}
\usepackage{hyperref}
\usepackage{gitinfo2}
\usepackage{tikz}
 \usetikzlibrary{calc}
\usepackage{amsmath}
\usepackage{mathtools}
\usepackage{bibentry}
\usepackage{xcolor}
\usepackage{colortbl} % Add colour to LaTeX tables
\usepackage{caption}
\usepackage[export]{adjustbox}
\usepackage{pgfplots} \pgfplotsset{compat = 1.17}
\usepackage{makecell}
\usepackage{fancybox}
\usepackage{ragged2e}
\usepackage{fontawesome}
\usepackage{seqsplit}
\usepackage{tabularx}
\usepackage{tcolorbox}
\usepackage{booktabs} % use instead  \hline in tables

% Color Options
\definecolor{highlight}{rgb}{0.65,0.84,0.82}
\definecolor{focus}{rgb}{0.72, 0, 0}
\definecolor{lightred}{rgb}{0.8,0.5,0.5}
\definecolor{midgray}{RGB}{190,195,200}

 %UniBas Main Colors
\definecolor{mint}{RGB}{165,215,210}
\definecolor{anthracite}{RGB}{45,55,60}
\definecolor{red}{RGB}{210,5,55}

 %UniBas Color Palette (for graphics)
\definecolor{strongmint}{RGB}{30,165,165}
\definecolor{darkmint}{RGB}{0,110,110}
\definecolor{softanthracite}{RGB}{140,145,150}
\definecolor{brightanthracite}{RGB}{190,195,200}
\definecolor{softred}{RGB}{235,130,155}

%Custom Colors
\definecolor{lightergray}{RGB}{230, 230, 230}



% Beamer Template Options
\beamertemplatenavigationsymbolsempty
\setbeamertemplate{footline}[frame number]
\setbeamercolor{structure}{fg=black}
\setbeamercolor{footline}{fg=black}
\setbeamercolor{title}{fg=black}
\setbeamercolor{frametitle}{fg=black}
\setbeamercolor{item}{fg=black}
\setbeamercolor{}{fg=black}
\setbeamercolor{bibliography item}{fg=black}
\setbeamercolor*{bibliography entry title}{fg=black}
\setbeamercolor{alerted text}{fg=focus}
\setbeamertemplate{items}[square]
\setbeamertemplate{enumerate items}[default]
\captionsetup[figure]{labelfont={color=black},font={color=black}}
\captionsetup[table]{labelfont={color=black},font={color=black}}

\setbeamertemplate{bibliography item}{\insertbiblabel}

%tcolor boxes
\newtcolorbox{samplecode}[2][]{
  colback=mint, colframe=darkmint, coltitle=white,
  fontupper = \ttfamily\scriptsize, fonttitle= \bfseries\scriptsize,
  boxrule = 0mm, arc = 0mm,
  boxsep = 1.3mm, left = 0mm, right = 0mm, top = 0.5mm, bottom = 0mm, middle=0mm,
  #1,title=#2}
  
\newtcolorbox{keytakeaway}[2][]{
  colback=softred, colframe=red, coltitle=white,
  fontupper = \scriptsize, fonttitle= \bfseries\scriptsize,
  boxrule = 0mm, arc = 0mm,
  boxsep = 1.3mm, left = 0mm, right = 0mm, top = 0.5mm, bottom = 0mm, middle=0mm,
  #1,title=#2}

\newtcolorbox{exercise}[2][]{
  colback=brightanthracite, colframe=anthracite, coltitle=white,
  fontupper = \scriptsize, fonttitle= \bfseries\scriptsize,
  boxrule = 0mm, arc = 0mm,
  boxsep = 1.3mm, left = 0mm, right = 0mm, top = 0.5mm, bottom = 0mm, middle=0mm,
  #1,title=#2}



% Link Icon Command 
\newcommand{\link}{%
    \tikz[x=1.2ex, y=1.2ex, baseline=-0.05ex]{%
        \begin{scope}[x=1ex, y=1ex]
            \clip (-0.1,-0.1)
                --++ (-0, 1.2)
                --++ (0.6, 0)
                --++ (0, -0.6)
                --++ (0.6, 0)
                --++ (0, -1);
            \path[draw,
                line width = 0.5,
                rounded corners=0.5]
                (0,0) rectangle (1,1);
        \end{scope}
        \path[draw, line width = 0.5] (0.5, 0.5)
            -- (1, 1);
        \path[draw, line width = 0.5] (0.6, 1)
            -- (1, 1) -- (1, 0.6);
        }
    }

% Other commands
\newcommand\tab[1][0.5cm]{\hspace*{#1}} % for code boxes


% Read Git Data from Github Actions Workflow
% Defaults to gitinfo2 for local builds
\IfFileExists{gitInfo.txt}
	{\input{gitInfo.txt}}
	{
		\newcommand{\gitRelease}{(Local Release)}
		\newcommand{\gitSHA}{\gitHash}
		\newcommand{\gitDate}{\gitAuthorIsoDate}
	}

% Custom Titlepage
\defbeamertemplate*{title page}{customized}[1][]
{
  \vspace{-0cm}\hfill\includegraphics[width=2.5cm]{../config/logo_cif}
  \includegraphics[width=1.9cm]{../config/seal_wwz}
  \\ \vspace{2em}
  \usebeamerfont{title}\textbf{\inserttitle}\par
  \usebeamerfont{title}\usebeamercolor[fg]{title}\insertsubtitle\par  \vspace{1.5em}
  \small\usebeamerfont{author}\insertauthor\par
  \usebeamerfont{author}\insertinstitute\par \vspace{2em}
  \usebeamercolor[fg]{titlegraphic}\inserttitlegraphic
    \tiny \noindent \texttt{Release Ver.: \gitRelease}\\ 
    \texttt{Version Hash: \gitSHA}\\
    \texttt{Version Date: \gitDate}\\ \vspace{1em}
    
    
    \iffalse
  \link \href{https://github.com/cifunibas/Bitcoin-Blockchain-Cryptoassets/blob/main/slides/intro.pdf}
  {Get most recent version}\\
  \link \href{https://github.com/cifunibas/Bitcoin-Blockchain-Cryptoassets/blob/main/slides/intro.pdf}
  {Watch video lecture}\\ 
  
  \fi
  
  \vspace{1em}
  License: \texttt{Creative Commons Attribution-NonCommercial-ShareAlike 4.0 International}\\\vspace{2em}
  \includegraphics[width = 1.2cm]{../config/license}
}


% tikzlibraries
\usetikzlibrary{decorations.pathreplacing}
\usetikzlibrary{decorations.markings}
\usetikzlibrary{positioning}
\usetikzlibrary{calc}
\captionsetup{font=footnotesize}

%%%%%%%%%%%%%%%%%%%%%%%%%%%%%%%%%%%%%%%%%%%%%%
%%%%%%%%%%%%%%%%%%%%%%%%%%%%%%%%%%%%%%%%%%%%%%
\begin{document}

\thispagestyle{empty}
\begin{frame}[noframenumbering]
	\titlepage
\end{frame}


%%%
\begin{frame}{Protocol Layer}


\scalebox{0.7}{

\begin{tikzpicture}
  \input{../assets/figures/defi_stack_protocol_layer_asset_management.tex}
\end{tikzpicture}

}

\end{frame}
%%%	

%%%
\begin{frame}{Introduction}

Asset Management in the traditional financial system: \\ \vspace{1em}
	
	\begin{itemize}
		\item<1-> Requires trust in a third party 
		\item<2-> Often intransparent with incentives to use information asymmetry to own advantage
		\item<3-> Gambling for redemption (hidden actions)
		\item<4-> Airdrops
	\end{itemize}
	
\vspace{1em}	

DeFi Asset Management: \\ \vspace{1em}

	\begin{itemize}
		\item<1-> Non-custodial
		\item<2-> Transparent
		\item<3-> No discrimination
	\end{itemize}

\end{frame}

%%%	

%%%
\begin{frame}{Token Basket}

\begin{figure}
	\begin{tikzpicture}
		%icons
	\uncover<1->{
		\node(alice) at (-4.5,0) {\includegraphics[scale=0.1]{../assets/images/avatar.png}};
		\node at (-4.5,-1) {Alice};
		\node(SC) at (4.7,0) {\includegraphics[scale=0.12]{../assets/images/smart_contract.png}}; 
		\node at (4.7,-1) {Token Basket};
	
%arrows
		\draw [->, ultra thick] (-3,0) -- (3,0) node [midway,above] {mint token};
		\draw [->, ultra thick] (-3,0.75) -- (3,0.75) node [midway,above] {red token};
	}
	\uncover<2->{
		\draw [<-, ultra thick] (-3,-0.75) -- (3,-0.75) node [midway,above] {grey basket token};
	}
%circles
	\uncover<1->{
		\draw[fill=mint, draw=black, thick] (-3,0) circle (6pt);
		\draw[fill=red, draw=black, thick] (-3,0.75) circle (6pt);
	}
	\uncover<2->{
		\draw[fill=lightergray, draw=black, thick] (3,-0.75) circle (6pt);
	}

	\end{tikzpicture}	
\end{figure}

\end{frame}
%%%	


%%%
\begin{frame}{Vault Active vs Passive Rebalancing}
	\begin{minipage}{0.4\textwidth}
		\begin{center}
			\includegraphics[scale=0.18]{../assets/images/manager.png}
		\end{center}
		\begin{center}
		Actively \\ \vspace{1em}
		Manager trades according to predefined rule set
\vspace{0.5em}		
			\begin{scriptsize}
				\begin{itemize}
					\item<1-> Diversification
					\item<2-> Max. Exposure
					\item<3-> Whitelist for Exchanges / Platforms
					\item<4->  Asset universe
					\item<5->  ...
				\end{itemize}		
			\end{scriptsize}
		\vspace{0.5em}Change of rule set requires investor approval \\
		\end{center}
		

	\end{minipage}
	\hfill
	\begin{minipage}{0.4\textwidth}
		\begin{center}
			\includegraphics[scale=0.107]{../assets/images/document.png}
		\end{center}
		\begin{center}
		Passively 	\\ \vspace{1em}
		Rebalancing according to fixed rule set \vspace{0.5em}		
			\begin{scriptsize}
				\begin{itemize}
					\item<1-> Fixed ratio on \$ ratio
					\item<2-> Index investing
					\item<3-> Flexible weights (e.g. according to marketcap)
					\item<4->  Advanced strategies
					\item<4->  ...
				\end{itemize}
			\end{scriptsize}
	 	\vspace{0.5em}Problem: Often requires Oracles (on- or off-chain)
		\end{center}
	\end{minipage}

\end{frame}
%%%


%%%
\begin{frame}{Divisibility and F-NFT}

	
	\vspace{2em}
	\begin{tikzpicture}
		%icons
	\uncover<1->{
		\node(bob) at (-4.5,0) {\includegraphics[scale=0.1]{../assets/images/avatar.png}};
		\node at (-4.5,-1) {Bob};
		}
	\uncover<2->{	
		\node(SC) at (4.7,0) {\includegraphics[scale=0.12]{../assets/images/smart_contract.png}}; 
		\node at (4.7,-1) {Vault};
		
%arrows

		\draw [->, ultra thick] (-2.6,0.5) -- (3,0.5) node [midway,above] {hasoshi ERC-721};
	}
	\uncover<3->{
		\draw [<-, ultra thick] (-3,-0.5) -- (2.6,-0.5) node [midway,above] {HAS token};
	}
%circles
	\uncover<1->{


	\node(Hasoshi) at (-3,0.5) {\includegraphics[scale=0.055]{../assets/images/art_hasoshi.png}};	
	}
	
	\uncover<3->{
		\draw[fill=lightergray, draw=black, thick] (3,-0.5) circle (6pt);		
	\node(Hasoshi) at (3,-0.5) {\includegraphics[scale=0.055]{../assets/images/CLC_4.png}};	
	
		\draw[fill=lightergray, draw=black, thick] (3.3,-0.1) circle (6pt);	
		\node(Hasoshi) at (3.3,-0.1) {\includegraphics[scale=0.055]{../assets/images/CLC_4.png}};	
		
			\draw[fill=lightergray, draw=black, thick] (3.5,-0.7) circle (6pt);	
			\node(Hasoshi) at (3.5,-0.7) {\includegraphics[scale=0.055]{../assets/images/CLC_4.png}};	
	}
	\end{tikzpicture}	

\end{frame}
%%%	

%%%	
\begin{frame}{Divisibility and F-NFT}

	
	\vspace{2em}
	\begin{tikzpicture}
		
%Alice Name + Avatar
	\uncover<1->{ 
		\node(alice) at (0,4.5) {\includegraphics[scale=0.1]{../assets/images/avatar.png}};
		\node at (0,3.5) {Alice};
	}

%Alice NFT

	\uncover<1-2>{
		
		\node(Hasoshi) at (1,5) {\includegraphics[scale=0.055]{../assets/images/art_hasoshi.png}};				
	}
		
% Alice Hasoshi Token	

	\uncover<2-2>{

		\draw[fill=lightergray, draw=black, thick] (-1.9,4.8) circle (6pt);		
		\node(Hasoshi) at (-1.9,4.8) {\includegraphics[scale=0.055]{../assets/images/CLC_4.png}};	
	
		\draw[fill=lightergray, draw=black, thick] (-1.4,5.15) circle (6pt);	
		\node(Hasoshi) at (-1.4,5.15) {\includegraphics[scale=0.055]{../assets/images/CLC_4.png}};	
		
		\draw[fill=lightergray, draw=black, thick] (-1.4,4.45) circle (6pt);	
		\node(Hasoshi) at (-1.4,4.45) {\includegraphics[scale=0.055]{../assets/images/CLC_4.png}};	
			
	}
		
		
%Bob & Charlie Avatar & Text		
		
	\uncover<1->{
		
		\node(bob) at (-3.5,0) {\includegraphics[scale=0.1]{../assets/images/avatar.png}};
		\node at (-3.5,-1) {Bob};

		
		\node(charlie) at (3.5,0) {\includegraphics[scale=0.1]{../assets/images/avatar.png}};
		\node at (3.5,-1) {Charlie};	
	
	}
		
		
%arrows

	\uncover<3->{

		\draw [->, ultra thick] (-1,3.2) -- (-2.8,1.2) node [midway,left] {1 fHAS \hspace{0.5cm} };
	
		\draw [->, ultra thick] (1,3.2) -- (2.8,1.2) node [, midway,right] { \hspace{0.5cm}    2 fHAS };
	
	}

% split NFTS of Charlie & Bob
	\uncover<3->{

		\node(bob_Hasoshi) at (-4.2,0.5) {\includegraphics[scale=0.055]{../assets/images/art_hasoshi_ONE_THIRD.png}
	};	
	
		\node(charlie_Hasoshi) at (4.3,0.5) {\includegraphics[scale=0.055]{../assets/images/art_hasoshi_TWO_THIRDs.png}};		
	
	}
	


% Coins Bob & Charlie

	\uncover<3->{

%%%%% Coins Bob


		\draw[fill=lightergray, draw=black, thick] (-3.47,-1.6) circle (6pt);		
	\node(Hasoshi) at (-3.47,-1.6) {\includegraphics[scale=0.055]{../assets/images/CLC_4.png}};	
	
%%%% Coins Charlie

		\draw[fill=lightergray, draw=black, thick] (3.83,-1.6) circle (6pt);	
		\node(Hasoshi) at (3.83,-1.6) {\includegraphics[scale=0.055]{../assets/images/CLC_4.png}};	
		
			\draw[fill=lightergray, draw=black, thick] (3.23,-1.6) circle (6pt);	
			\node(Hasoshi) at (3.23,-1.6) {\includegraphics[scale=0.055]{../assets/images/CLC_4.png}};	
}

	\end{tikzpicture}	

\end{frame}
%%%	


%%%	
\begin{frame}{Benefits of F-NFTs}

\begin{minipage}{0.2\textwidth}
			\begin{center}
				\includegraphics[height=3em]{../assets/images/unlocked_padlock.png}
				
			\end{center}
		\end{minipage}
		\hspace{1 em}
		\begin{minipage}{0.72\textwidth}
			Accessible to more investors / \\ Lower cost of entry)

		\end{minipage}
	
		\pause
		\vspace{2 em}
		\begin{minipage}{0.2\textwidth}
			\begin{center}
				\includegraphics[height=3em]{../assets/images/faucet.png}
			\end{center}
		\end{minipage}
		\hspace{1 em}
		\begin{minipage}{0.72\textwidth}
			Higher liquidity \\

		\end{minipage}
	
		\pause
		\vspace{2 em}
		\begin{minipage}{0.2\textwidth}
			\begin{center}
				\includegraphics[height=3em]{../assets/images/exchange.png}
			\end{center}
		\end{minipage}
		\hspace{1 em}
		\begin{minipage}{0.72\textwidth}
		Better price discovery for the underlying asset 
		\end{minipage}


\end{frame}
%%%	






%%%
\begin{frame}{Redemption Schemes}

	
\end{frame}
%%%	


%%%
\begin{frame}{Yield Farming}

Yieldfarming Strategies \\ \vspace{1em}

	\begin{itemize}
	

		\item<1-> Providing liquidity on DEX
		\item<2-> Lending / providing assets on lending platform
		\item<3-> Arbitrage opportunities between protocols
		\item<4-> Exploting bonuses (e.g. liquidity / usage incentives)
		\item<4->...
	\end{itemize}


	
\end{frame}
%%%	



%%%
\begin{frame}{Yield Farming}


\begin{minipage}{0.2\textwidth}
			\begin{center}
				\includegraphics[height=3em]{../assets/images/strategy.png}
				
			\end{center}
		\end{minipage}
		\hspace{1 em}
		\begin{minipage}{0.72\textwidth}
			Use leverage and / or combine multiple strategies \\ ... to increase yield

		\end{minipage}
		\pause
		\vspace{2 em}
		
		

		\begin{minipage}{0.2\textwidth}
			\begin{center}
				\includegraphics[height=3em]{../assets/images/exchange.png}
			\end{center}
		\end{minipage}
		\hspace{1 em}
		\begin{minipage}{0.72\textwidth}
			APY / APR increase
		\end{minipage}
	
		\pause
		\vspace{2 em}
		\begin{minipage}{0.2\textwidth}
			\begin{center}
				\includegraphics[height=3em]{../assets/images/warning.png}
			\end{center}
		\end{minipage}
		\hspace{1 em}
		\begin{minipage}{0.72\textwidth}
		Higher exposure and liquidity risk 
		\end{minipage}
		
		
		
		
\end{frame}
%%%	



\begin{frame}{Yield Farming Agregators}



		\begin{minipage}{0.2\textwidth}
			\begin{center}
				\includegraphics[height=3em]{../assets/images/diversification.png}
			\end{center}
		\end{minipage}
		\hspace{1 em}
		\begin{minipage}{0.72\textwidth}
		Protocol combining multiple strategies / assets 
		\end{minipage}
		\pause
		\vspace{2 em}
		
		\begin{minipage}{0.2\textwidth}
			\begin{center}
				\includegraphics[height=3em]{../assets/images/warning.png}
			\end{center}
		\end{minipage}
		\hspace{1 em}
		\begin{minipage}{0.72\textwidth}
		Increase in wrapping complexity. 
		\end{minipage}
		\pause
		\vspace{3 em}			
		
		
			\link \href{https://arxiv.org/pdf/2012.09306}{Decentralized Finance, Centralized Ownership? An Iterative Mapping Process to Measure Protocol Token Distribution (Nadler, 2020}

\end{frame}


%%%
\begin{frame}{Generation of Rewards}
\vspace{1em}


\begin{figure}[H]
  \begin{minipage}[t]{.25\textwidth}
  	\center 
    \includegraphics[width=0.7\textwidth]{../assets/images/database-usage.png}
    \vspace{0.8em}\vspace{0.8em}
    Protocol Usage $\&$ Fees
  \end{minipage}
  \hfill
  \begin{minipage}[t]{.25\textwidth}
  \center 
    \includegraphics[width=0.7\textwidth]{../assets/images/reserved.png}
	\vspace{0.8em}\vspace{0.8em}    
    Predetermined allocation at token creation
  \end{minipage}
  \hfill
  \begin{minipage}[t]{.25\textwidth}
  \center 
    \includegraphics[width=0.7\textwidth]{../assets/images/inflation.png}
    \vspace{0.8em}\vspace{0.8em}
    Extension of Token Supply
  \end{minipage}  
\end{figure}




	
\end{frame}
%%%	


%%%%%%%%%%%%%%%%%%%%%%%%%%%%%%%%%%%%%%%
%%%%%%%%%%%%%%%%%%%%%%%%%%%%%%%%%%%%%%%
%%%%%%%%%%%%%%%%%%%%%%%%%%%%%%%%%%%%%%%
%%%%%%%%%%%%%%%%%%%%%%%%%%%%%%%%%%%%%%%





%%%
\begin{frame}{Airdrop Motivation}
	\vspace{3em}

\hspace*{1cm}
\begin{minipage}[c]{1\textwidth}
  \center
	\begin{tikzpicture}

%Gossip + Airdrop + Crowd 

		\node(gossip) at (-2,1.2) {\includegraphics[scale=0.1]{../assets/images/gossip_bubble.png}};

		\node(airdrop) at (-1.37,1.83) {\includegraphics[scale=0.05]{../assets/images/airdrop.png}};
		
		\scalebox{0.7}{
			\begin{tikzpicture}
  %3people
		\node at (0,0.75) {\includegraphics[scale=0.1]{../assets/images/avatar.png}};
		\node at (-0.7,-0.75) {\includegraphics[scale=0.1]{../assets/images/avatar.png}};
		\node at (0.7,-0.75) {\includegraphics[scale=0.1]{../assets/images/avatar.png}};
			\end{tikzpicture}
		}
	\end{tikzpicture}
  \end{minipage}
  \vspace{-2em}
  \begin{figure}[H]
  \begin{minipage}[t]{.25\textwidth}
  	\center 
    \includegraphics[width=0.5\textwidth]{../assets/images/attention.png}
    \vspace{0.8em}\vspace{0.8em}
    Attention
  \end{minipage}
  \hfill
  \begin{minipage}[t]{.25\textwidth}
  \center 
    \scalebox{0.4}{
			\begin{tikzpicture}
  %crowd

1st row

		\node at (0.7,2.25) {\includegraphics[scale=0.1]{../assets/images/avatar.png}};

		\node at (-0.7,2.25) {\includegraphics[scale=0.1]{../assets/images/avatar.png}};





%2nd row
		\node at (1.4,0.75) {\includegraphics[scale=0.1]{../assets/images/avatar.png}};

		\node at (-1.4,0.75) {\includegraphics[scale=0.1]{../assets/images/avatar.png}};
		
		\node at (0,0.75)
		{\includegraphics[scale=0.1]{../assets/images/avatar.png}};
		
		
		
		
%3rd row		

		 
		\node at (0.7,-0.75) {\includegraphics[scale=0.1]{../assets/images/avatar.png}};
		
		\node at (-0.7,-0.75)
{\includegraphics[scale=0.1]{../assets/images/avatar.png}};				
		
		\node at (2.1,-0.75) {\includegraphics[scale=0.1]{../assets/images/avatar.png}};
		
		\node at (-2.1,-0.75) {\includegraphics[scale=0.1]{../assets/images/avatar.png}};
			\end{tikzpicture}
		}
	\vspace{0.8em}\vspace{0.8em}    
    User acquisition
  \end{minipage}
  \hfill
  \begin{minipage}[t]{.25\textwidth}
  \center 
    \includegraphics[width=0.5\textwidth]{../assets/images/usage.png}
    \vspace{0.8em}\vspace{0.8em}
    Protocol usage
  \end{minipage}  
\end{figure}
		
	
\end{frame}
%%%	


%%%
\begin{frame}{Airdrop Problems}
	\vspace{1.5em}
	\scalebox{0.66}{
	
		\begin{tikzpicture}
			%Airdrop based on
	\uncover<1->{ 
		\node(airdrop) at (-9,6) {\includegraphics[scale=0.1]{../assets/images/airdrop.png}};
	}
	\uncover<2->{	
		\node at (-7.,5.5) {based on:};
		\node(airdrop) at (-5,6) {\includegraphics[scale=0.13]{../assets/images/question-mark .png}};

	}

%Agent
	\uncover<3->{ 
		\node at (0,4.5) {\includegraphics[scale=0.2]{../assets/images/fraud.png}};	
	}
	\uncover<3->{ 
		\node at (3.2,4.2) {Airdrop farming};
	}


%Arrows

	\uncover<3->{

		\draw [-, ultra thick] (-1,3.2) -- (-2.8,1.2) node [midway,left] {} ;
		
	}
	
	\uncover<4->{
		\draw [-, ultra thick] (1,3.2) -- (2.8,1.2) node [, midway,right] {};
	
	}


		
%Multiple Accounts		
		
	\uncover<3->{
		\node at (-4.5,1) {\includegraphics[scale=0.08]{../assets/images/suspect.png}};
		\node at (-4.5,0) {\tiny 0x4423...EE55};
		
		\node at (-3,-0.5) {\includegraphics[scale=0.08]{../assets/images/suspect.png}};
		\node at (-3,-1.5) {\tiny 0xA823...CE3B};
				
		\node at (-5,-1) {\includegraphics[scale=0.08]{../assets/images/suspect.png}};
		\node at (-5,-2) {\tiny 0x610A...6d63};
	}
	
	\uncover<3->{
			\node at (-3.8,-3) {Sybil attack};
	}
		
%% Transactions
	\uncover<4->{
\node at (4.5,1) {
	\scalebox{0.3}{
		\begin{tikzpicture}[scale=1, every node/.style={scale=1}]
			%Define Coordinates

%Files and mintgreen Arrows	
\coordinate (1) at (-0.5,1);
\coordinate (2) at (1.6,0);
\coordinate (3) at (-0.1,-0.7);

%Transactions

\node (state)at(1)  [scale=0.5]{\includegraphics{../assets/images/transactions.png}};
\node (state)at (2) [scale=0.5] {\includegraphics{../assets/images/transactions.png}};
\node (state)at (3) [scale=0.5] {\includegraphics{../assets/images/transactions.png}};

%Box

\draw[dashed, ultra thick] (-1.5,2) -- (2.7,2);
\draw[dashed, ultra thick] (-1.5,2) -- (-1.5,-1.7);
\draw[dashed, ultra thick] (-1.5,-1.7) -- (2.7,-1.7);
\draw[dashed, ultra thick] (2.7,2) -- (2.7,-1.7);

		\end{tikzpicture}
		}
};

\node at (2.65,0) {
	\scalebox{0.3}{
		\begin{tikzpicture}[scale=1, every node/.style={scale=1}]
			%Define Coordinates

%Files and mintgreen Arrows	
\coordinate (1) at (-0.5,1);
\coordinate (2) at (1.6,0);
\coordinate (3) at (-0.1,-0.7);

%Transactions

\node (state)at(1)  [scale=0.5]{\includegraphics{../assets/images/transactions.png}};
\node (state)at (2) [scale=0.5] {\includegraphics{../assets/images/transactions.png}};
\node (state)at (3) [scale=0.5] {\includegraphics{../assets/images/transactions.png}};

%Box

\draw[dashed, ultra thick] (-1.5,2) -- (2.7,2);
\draw[dashed, ultra thick] (-1.5,2) -- (-1.5,-1.7);
\draw[dashed, ultra thick] (-1.5,-1.7) -- (2.7,-1.7);
\draw[dashed, ultra thick] (2.7,2) -- (2.7,-1.7);

		\end{tikzpicture}
		}
};

\node at (4.8,-1.2) {
	\scalebox{0.3}{
		\begin{tikzpicture}[scale=1, every node/.style={scale=1}]
			%Define Coordinates

%Files and mintgreen Arrows	
\coordinate (1) at (-0.5,1);
\coordinate (2) at (1.6,0);
\coordinate (3) at (-0.1,-0.7);

%Transactions

\node (state)at(1)  [scale=0.5]{\includegraphics{../assets/images/transactions.png}};
\node (state)at (2) [scale=0.5] {\includegraphics{../assets/images/transactions.png}};
\node (state)at (3) [scale=0.5] {\includegraphics{../assets/images/transactions.png}};

%Box

\draw[dashed, ultra thick] (-1.5,2) -- (2.7,2);
\draw[dashed, ultra thick] (-1.5,2) -- (-1.5,-1.7);
\draw[dashed, ultra thick] (-1.5,-1.7) -- (2.7,-1.7);
\draw[dashed, ultra thick] (2.7,2) -- (2.7,-1.7);

		\end{tikzpicture}
		}
};
}
\uncover<4->{
		\node at (4,-3) {Multiple transactions};
}


%% Transactions Text
\uncover<4->{
\node at (4.5,0.2) {\tiny Transactions};
\node at (2.65,-0.8) {\tiny Transactions};
\node at (4.8,-2) {\tiny Transactions};
}





%% Insider Knowledge



\uncover<5->{


\node at (-9,3.5){\includegraphics[scale=0.08]{../assets/images/warning_yellow.png}};

\node at (-6.35,3.5){Insider knowledge};
}


		\end{tikzpicture}
	}	
	
\end{frame}
%%%	

%%%%%%%%%%%%%%%%%%%%%%%%%%%%%%%%%%%%%%%
%%%%%%%%%%%%%%%%%%%%%%%%%%%%%%%%%%%%%%%
%%%%%%%%%%%%%%%%%%%%%%%%%%%%%%%%%%%%%%%
%%%%%%%%%%%%%%%%%%%%%%%%%%%%%%%%%%%%%%%

%%%
\begin{frame}{Yield Farming}

 \vspace{1em}

	\begin{enumerate}
		\item<1-> Diversification of Assets and Risks
		\item<2-> Reduction of Gas and Transaction Fees
		\item<3-> Management of Assets
	\end{enumerate}
	
\vspace{1em}


There are two types of Token baskets: \vspace{1em}
	
	\begin{itemize}
		\item<4-> actively managed
		\item<5-> passively managed
	\end{itemize}
	

\end{frame}
%%%	



%%%
\begin{frame}{Tt}

Concepts in Crypto Asset Management: \\ \vspace{1em}
	
	\begin{itemize}
		\item<1-> Token Basket 
		\item<2-> Yield Farming
		\item<3-> Distribution concepts
		\item<4-> Airdrops
	\end{itemize}

\end{frame}
%%%	

%%%
\begin{frame}{Asset Management}

This is how lists can be used in the template. \\ \vspace{1em}

Some of the main features of this template:
	
	\begin{itemize}
		\item<1-> stay focused
		\item<2-> save ink
		\item<3-> do not get distracted
	\end{itemize}

	\vspace{1em}	
	\uncover<4->{The same works for Enumerate environments:}

	\begin{enumerate}
		\item<5-> This is the first point
		\item<6-> and the second one
		\item<7-> Last but not least, the third one
	\end{enumerate}
	
\end{frame}
%%%	



%%%
\begin{frame}{Tables}
	\begin{table}
		\begin{tabular}{ll}
			A & B\\
			C & D
		\end{tabular}
		\caption{This is a table}
		\label{tbl:simpletable}
	\end{table}
\end{frame}

%%%


%%%
\begin{frame}{Box Conventions}
	
	\begin{columns}[T]
		\begin{column}{0.45\textwidth}
			\vspace{-1em}
			\begin{samplecode}{Basic Contract Structure (Solidity)}
				Pragma solidity  \textasciicircum 0.4.19; \
				contract HelloWorld \{ \\
				\color{softanthracite} /* This is where you write your code */
			\end{samplecode}			
		\end{column} %\hfill

		\begin{column}{0.5\textwidth}
			\textbf{Sample Code Box}\\
			Contains code snippets in Solidity or pseudo code.
		\end{column}
	\end{columns}

\vspace{2.5em}

	\begin{columns}[T]
		\begin{column}{0.45\textwidth}
			\vspace{-0.8em}
			\begin{keytakeaway}{Transaction Definitions}
				A transaction is a message send by an externally owned account.
			\end{keytakeaway}			
		\end{column}
		
		\begin{column}{0.5\textwidth}
			\textbf{Key Take-Away}\\
			Highlights important concepts and definitions.
		\end{column}
	\end{columns}

\vspace{2.5em}

	\begin{columns}[T]
		\begin{column}{0.45\textwidth}
			\vspace{-0.9em}
			\begin{exercise}{Exercise 1}
				Get Ropsten Ether from the faucet and deploy your first smart contract on Ropsten Testnet.
			\end{exercise}					
		\end{column}
		\begin{column}{0.5\textwidth}
		\textbf{Exercises}\\
			Things for you to try out.
		\end{column}
	\end{columns}

\end{frame}

%%%


%%%
\begin{frame}{Code Sample}
	\begin{samplecode}{Minimum viable token}
		\begin{lstlisting}[language=Solidity]
contract MyToken {
	mapping (address => uint256) public balanceOf;
	
	function MyToken(
		uint256 initialSupply
		) {
		balanceOf[msg.sender] = initialSupply;       
	}
	
	function transfer(address _to, uint256 _value) {
		require(balanceOf[msg.sender] >= _value);           
		require(balanceOf[_to] + _value >= balanceOf[_to]); // Check for overflows
		balanceOf[msg.sender] -= _value;                    
		balanceOf[_to] += _value;                           
	}
}
\end{lstlisting}
	\end{samplecode}
\end{frame}
%%%


%%%
\begin{frame}{Equations}

Some in-line math can be used like this $y=x^2+2x-5$. \\ \vspace{1em}

However, in many cases it makes sense to use numbered equations. The $\&$-sign can be used to set anchor points, e.g. around the ``$=$'' for multiple equations. This is shown below in Equation \eqref{eq:firststep} and Equation \eqref{eq:secondstep}.
	\begin{align}
		y + 2x &= x^2+2x-5 \label{eq:firststep}\\
		y &= x^2-5 \label{eq:secondstep}
	\end{align}
\end{frame}
%%%

%%%
\begin{frame}{Figures}
	\begin{figure}
		\center
		\includegraphics[width=0.8\textwidth]{../config/logo_cif}	
		\caption{Center for Innovative Finance Logo}
		\label{fig:logo}
	\end{figure}
\end{frame}
%%%

%%%
\begin{frame}{Links}
		Links are signaled with the respective icon, which is set up in the config file and can be called with the respective command.\\
		
		\link \href{https://github.com/cifunibas}{Repository: CIF on GitHub}.
		
\end{frame}
%%%


%%%
\begin{frame}{Labels and (Non-Bib) References}
	In some cases in can make sense to reference to a figure, table or equation on a different slide. Use \texttt{$\backslash$pageref\{$<$label$>$\}} if you want to highlight, that there is a related resource on slide \pageref{fig:logo}.\\ \vspace{1em}
	
	Do - under no circumstances - hardcode references.	
\end{frame}
%%%

%%%
\begin{frame}{Bib-References}
		Read the Bitcoin Whitepaper, \cite{nakamotoBitcoin2008}.
\end{frame}
%%%

%%%
\begin{frame}%[allowframebreaks]
\frametitle{References and Recommended Reading}
	\bibliographystyle{amsplain}
	\bibliography{../assets/bib/refs}
\end{frame}
%%%



\end{document}