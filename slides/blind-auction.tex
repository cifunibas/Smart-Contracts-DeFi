% Choose one to switch between slides and handout
\documentclass[]{beamer}
%\documentclass[handout]{beamer}

% Video Meta Data
\title{Smart Contracts and Decentralized Finance}
\subtitle{Hashing and Complex Types}
\author{Prof. Dr. Fabian Schär}
\institute{University of Basel}

% Config File
% Packages
\usepackage[utf8]{inputenc}
\usepackage{hyperref}
\usepackage{gitinfo2}
\usepackage{tikz}
 \usetikzlibrary{calc}
\usepackage{amsmath}
\usepackage{mathtools}
\usepackage{bibentry}
\usepackage{xcolor}
\usepackage{colortbl} % Add colour to LaTeX tables
\usepackage{caption}
\usepackage[export]{adjustbox}
\usepackage{pgfplots} \pgfplotsset{compat = 1.17}
\usepackage{makecell}
\usepackage{fancybox}
\usepackage{ragged2e}
\usepackage{fontawesome}
\usepackage{seqsplit}
\usepackage{tabularx}
\usepackage{tcolorbox}
\usepackage{booktabs} % use instead  \hline in tables

% Color Options
\definecolor{highlight}{rgb}{0.65,0.84,0.82}
\definecolor{focus}{rgb}{0.72, 0, 0}
\definecolor{lightred}{rgb}{0.8,0.5,0.5}
\definecolor{midgray}{RGB}{190,195,200}

 %UniBas Main Colors
\definecolor{mint}{RGB}{165,215,210}
\definecolor{anthracite}{RGB}{45,55,60}
\definecolor{red}{RGB}{210,5,55}

 %UniBas Color Palette (for graphics)
\definecolor{strongmint}{RGB}{30,165,165}
\definecolor{darkmint}{RGB}{0,110,110}
\definecolor{softanthracite}{RGB}{140,145,150}
\definecolor{brightanthracite}{RGB}{190,195,200}
\definecolor{softred}{RGB}{235,130,155}

%Custom Colors
\definecolor{lightergray}{RGB}{230, 230, 230}



% Beamer Template Options
\beamertemplatenavigationsymbolsempty
\setbeamertemplate{footline}[frame number]
\setbeamercolor{structure}{fg=black}
\setbeamercolor{footline}{fg=black}
\setbeamercolor{title}{fg=black}
\setbeamercolor{frametitle}{fg=black}
\setbeamercolor{item}{fg=black}
\setbeamercolor{}{fg=black}
\setbeamercolor{bibliography item}{fg=black}
\setbeamercolor*{bibliography entry title}{fg=black}
\setbeamercolor{alerted text}{fg=focus}
\setbeamertemplate{items}[square]
\setbeamertemplate{enumerate items}[default]
\captionsetup[figure]{labelfont={color=black},font={color=black}}
\captionsetup[table]{labelfont={color=black},font={color=black}}

\setbeamertemplate{bibliography item}{\insertbiblabel}

%tcolor boxes
\newtcolorbox{samplecode}[2][]{
  colback=mint, colframe=darkmint, coltitle=white,
  fontupper = \ttfamily\scriptsize, fonttitle= \bfseries\scriptsize,
  boxrule = 0mm, arc = 0mm,
  boxsep = 1.3mm, left = 0mm, right = 0mm, top = 0.5mm, bottom = 0mm, middle=0mm,
  #1,title=#2}
  
\newtcolorbox{keytakeaway}[2][]{
  colback=softred, colframe=red, coltitle=white,
  fontupper = \scriptsize, fonttitle= \bfseries\scriptsize,
  boxrule = 0mm, arc = 0mm,
  boxsep = 1.3mm, left = 0mm, right = 0mm, top = 0.5mm, bottom = 0mm, middle=0mm,
  #1,title=#2}

\newtcolorbox{exercise}[2][]{
  colback=brightanthracite, colframe=anthracite, coltitle=white,
  fontupper = \scriptsize, fonttitle= \bfseries\scriptsize,
  boxrule = 0mm, arc = 0mm,
  boxsep = 1.3mm, left = 0mm, right = 0mm, top = 0.5mm, bottom = 0mm, middle=0mm,
  #1,title=#2}



% Link Icon Command 
\newcommand{\link}{%
    \tikz[x=1.2ex, y=1.2ex, baseline=-0.05ex]{%
        \begin{scope}[x=1ex, y=1ex]
            \clip (-0.1,-0.1)
                --++ (-0, 1.2)
                --++ (0.6, 0)
                --++ (0, -0.6)
                --++ (0.6, 0)
                --++ (0, -1);
            \path[draw,
                line width = 0.5,
                rounded corners=0.5]
                (0,0) rectangle (1,1);
        \end{scope}
        \path[draw, line width = 0.5] (0.5, 0.5)
            -- (1, 1);
        \path[draw, line width = 0.5] (0.6, 1)
            -- (1, 1) -- (1, 0.6);
        }
    }

% Other commands
\newcommand\tab[1][0.5cm]{\hspace*{#1}} % for code boxes


% Read Git Data from Github Actions Workflow
% Defaults to gitinfo2 for local builds
\IfFileExists{gitInfo.txt}
	{\input{gitInfo.txt}}
	{
		\newcommand{\gitRelease}{(Local Release)}
		\newcommand{\gitSHA}{\gitHash}
		\newcommand{\gitDate}{\gitAuthorIsoDate}
	}

% Custom Titlepage
\defbeamertemplate*{title page}{customized}[1][]
{
  \vspace{-0cm}\hfill\includegraphics[width=2.5cm]{../config/logo_cif}
  \includegraphics[width=1.9cm]{../config/seal_wwz}
  \\ \vspace{2em}
  \usebeamerfont{title}\textbf{\inserttitle}\par
  \usebeamerfont{title}\usebeamercolor[fg]{title}\insertsubtitle\par  \vspace{1.5em}
  \small\usebeamerfont{author}\insertauthor\par
  \usebeamerfont{author}\insertinstitute\par \vspace{2em}
  \usebeamercolor[fg]{titlegraphic}\inserttitlegraphic
    \tiny \noindent \texttt{Release Ver.: \gitRelease}\\ 
    \texttt{Version Hash: \gitSHA}\\
    \texttt{Version Date: \gitDate}\\ \vspace{1em}
    
    
    \iffalse
  \link \href{https://github.com/cifunibas/Bitcoin-Blockchain-Cryptoassets/blob/main/slides/intro.pdf}
  {Get most recent version}\\
  \link \href{https://github.com/cifunibas/Bitcoin-Blockchain-Cryptoassets/blob/main/slides/intro.pdf}
  {Watch video lecture}\\ 
  
  \fi
  
  \vspace{1em}
  License: \texttt{Creative Commons Attribution-NonCommercial-ShareAlike 4.0 International}\\\vspace{2em}
  \includegraphics[width = 1.2cm]{../config/license}
}


% tikzlibraries
\usetikzlibrary{decorations.pathreplacing}
\usetikzlibrary{decorations.markings}
\usetikzlibrary{positioning}
\usetikzlibrary{calc}
\captionsetup{font=footnotesize}

%%%%%%%%%%%%%%%%%%%%%%%%%%%%%%%%%%%%%%%%%%%%%%
%%%%%%%%%%%%%%%%%%%%%%%%%%%%%%%%%%%%%%%%%%%%%%
\begin{document}

\thispagestyle{empty}
\begin{frame}[noframenumbering]
	\titlepage
\end{frame}

%%%
\begin{frame}{Sealed Bid Auctions}

	\textbf{Sealed bid auction:} Bidders submit sealed/secret bids so that no bidder knows the bid of any other participant.\\
	
	\vspace{1em}
	
	\begin{columns}[T]
		\begin{column}{0.45\textwidth}
			\uncover<2->{		
				\textbf{Open auction:}
				\begin{figure}
					\begin{tikzpicture}

	\scriptsize
	% Alice
	\node (alice) at (-2.5, 0) {\includegraphics[height= 0.125\textheight]{../assets/images/avatar.png}};
	\node[above =-.05cm of alice] (name1) {Alice};
	\node[below= -.05cm of alice](valuation) {Valuation: $v_a$};

	% Bob	
	\node (bob) at (0, 0) {\includegraphics[height= 0.125\textheight]{../assets/images/avatar.png}};
	\node[above =-.05cm of bob] (name2) {Bob};
	\node[below= -.05cm of bob](valuation) {Valuation: $v_b$};	

	% Sequential bidding
	\node (optimal) at (-1.25, -1.5) {Bid as long as $p \leq v$};
	
\end{tikzpicture}	
				\end{figure}
			}
		\end{column}
		\begin{column}{0.45\textwidth}
			\uncover<3->{		
				\textbf{Sealed bid auction:}
				\begin{figure}
					\begin{tikzpicture}

	\scriptsize
	% Alice
	\node (alice) at (-2.5, 0) {\includegraphics[height= 0.125\textheight]{../assets/images/avatar.png}};
	\node[above =-.05cm of alice] (name1) {Alice};
	\node[below= -.05cm of alice](valuation) {Valuation: $v_a$};

	% Bob	
	\node (bob) at (0, 0) {\includegraphics[height= 0.125\textheight]{../assets/images/avatar.png}};
	\node[above =-.05cm of bob] (name2) {Bob};
	\node[below= -.05cm of bob](valuation) {Valuation: $v_b$};
	
	% Simultaneous bidding
	\node[text width = 3.8cm] (optimal) at (-1.25, -2) {Bid based on own valuation and expectation about other participants' valuations.};
	
\end{tikzpicture}
				\end{figure}
			}
		\end{column}
	\end{columns}
	
	\vspace{1em}
	
	\uncover<4->{
		\begin{keytakeaway}{Bid and Reveal Phases}
			In sealed bid auctions, participants bid quasi-simultaneously, i.e., the auction process consists of two phases. First, bidders submit their bids, and second, the bids are revealed and the highest bidder is determined.
		\end{keytakeaway}
	}
\end{frame}
%%%

%%%
\begin{frame}{Problem 1: Sealed Bids}

	\textbf{Problem:} Transaction data on public blockchains/blockchain networks is transparent.
		
	\begin{figure}
		\input{../assets/figures/transparency.tex}
	\end{figure}
		
	\uncover<2->{
		\textbf{Solution: Commit and reveal}\\
		Send a hash of the bid during the bidding phase. Send the unencrypted value during the reveal phase.

		\begin{figure}
			\begin{tikzpicture}
	\scriptsize
	
	\node[label = above:{Alice}] (alice) at (-3,0) {\includegraphics[height = 0.15\textheight]{../assets/images/avatar.png}};
	\node[label = above:{\texttt{AuctionContract}}] (sc) at (5,0) {\includegraphics[height = 0.15\textheight]{../assets/images/smart_contract.png}};
	
	\uncover<3->{
		\node[below = -.03cm of alice] {\texttt{H(}``I bid 3 ETH."\texttt{)}};
	}
	
	\uncover<4->{
		\draw[->] ($(alice)+(2,0.4)$) node[left]{\textbf{Commit}} -- ($(sc)+(-2,0.4)$) node[right]{\textbf{Stores}};
		\filldraw[fill=black!20,dotted,thick] (-0.5,0.1) -- (2.5,0.1) -- (2.5,0.8) -- (-0.5,0.8) -- (-0.5,0.1) node[midway, right]{\texttt{0xb680ae657719a9...}};
	}
	
	\uncover<4>{
		\node[below = -.03cm of sc] {Alice \texttt{=> 0xb680...}};
	}
	
	\uncover<5->{
		\draw[->] ($(alice)+(2,-0.4)$) node[left]{\textbf{Reveal}} -- ($(sc)+(-2,-0.4)$) node[right]{\textbf{Checks}};
		\filldraw[fill=black!20,dotted,thick] (-0.5,-0.1) -- (2.5,-0.1) -- (2.5,-0.8) -- (-0.5,-0.8) -- (-0.5,-0.1) node[midway, right]{``I bid 3 ETH."};
	}
	
	\uncover<5>{
		\node[below = .25cm of sc] {\texttt{H(}``I bid 3 ETH."\texttt{)== 0xb680...}};
	
	}

\end{tikzpicture}

		\end{figure}
	}
	
\end{frame}
%%%

%%%
\begin{frame}{Problem 2: Binding Auction}

	\textbf{Problem:} Ensure that the highest bidder pays if they win.
	
	\vspace{1em}
	
	\uncover<2->{
		\textbf{Separate problems:}
		\begin{itemize}
			\item<2-> No way to force the bidder to pay later $\rightarrow$ value must be deposited at the time of the bid.
			\item<3-> Value transfers are visible on-chain / in the network $\rightarrow$ how can we ensure that bids are still secret?
		\end{itemize}
	}
	
	\vspace{1em}
	
	\uncover<4->{
		\textbf{Solution:} Allow for ''fake" bids. During the reveal phase, check:\\
	}
	
	\begin{itemize}
		\item<5-> Deposit $<$ Bid: Bid invalid, full refund
		\item<6-> Deposit $==$ Bid: Bid valid, no refund.
		\item<7-> Deposit $>$ Bid: Bid valid, excess deposit (Deposit - Bid) refunded.
	\end{itemize}

	\uncover<8->{
		$\rightarrow$ Additionally, allow the bidder to secretly state if their bid is fake or legitimate.
	}
	
\end{frame}
%%%

%%%
\begin{frame}{Hashing the Bid}

	\textbf{Hashing in Solidity:} \texttt{\genkey{keccak256}(\typesunits{bytes}) \genkey{returns} (\typesunits{bytes32})} can be used for any arbitrary input.\\
	
	\vspace{1em}
	
	\uncover<2->{
		\textbf{Creating a bytes array from any variables:} \texttt{abi.encodePacked()} combines all variables in a single bytes array without padding or extending them.\\
	}
	
	\vspace{1em}
	
	\uncover<3->{
		\textbf{What we will hash:}
		\begin{enumerate}
			\item<3-> Value of the bid
			\item<4-> Indicator whether the bid is fake or real
			\item<5-> Secret (salt) to prevent guessing
		\end{enumerate}
	}
	
\end{frame}
%%%

%%%
\begin{frame}{Strings}

	\textbf{Problem of hashing the bid:} Others can brute-force the hash.
	
	\begin{samplecode}{Pseudo-code}
		
		hash\_from\_alice = 0xb680ae657719a9...\\
		
		for(x in 1:100) \{ \\
		\phantom{....}hash = H("I bid x ETH.") \\
		\phantom{....}print(hash == hash\_from\_alice) \\
		\}
		
	\end{samplecode}
	
	\vspace{1em}
	
	\uncover<2->{
	\textbf{Solution:} add additional arbitrary information to the input.
	
		\begin{itemize}
			\item In our example, we use a string because it is intuitive and can be used similarly to a password.
			\item A string stores text and is written with single (' ') or double quotes ('' ").
			\item Unicode strings can be used by prefixing \texttt{unicode}, e.g., \texttt{unicode"Secret \includegraphics[height = 0.8\baselineskip]{../assets/images/earth_emoji.png}"}.
		\end{itemize}
	}

\end{frame}
%%%

%%%
\begin{frame}{Hashing the Bid}

	\textbf{Create a pure function to generate a sealed bid:}
	
	\begin{lstlisting}[language=Solidity]
contract SealedBidAuction {
	function generateSealedBid(uint _bidAmount, bool _isLegit, string memory _secret) public pure returns (bytes32 sealedBid) {
			sealedBid = keccak256(abi.encodePacked(_bidAmount, _isLegit, _secret));
	}
}
\end{lstlisting}
	
	No trace of a pure/view function call is stored on-chain.\\
	
\end{frame}
%%%

%%%
\begin{frame}[allowframebreaks]{Creating the New Contract}

	\textbf{Procedure:}
	
	\begin{itemize}
		\item Keep the basics from the simple auction contract.
		\item Split the auction into two periods by setting an end time for both periods.	
	\end{itemize}
	
	\begin{lstlisting}[language=Solidity]
// SPDX-License-Identifier: MIT
pragma solidity ^0.8.9;

contract SealedBidAuction {
  // Auction parameters
  address public immutable beneficiary;
  uint public biddingEnd;
  uint public revealEnd;
  
  // State of the auction
  uint public highestBid;
  address public highestBidder;
  bool public hasEnded;
  
  // Allowed withdrawals of previous bids
  mapping(address => uint) public pendingReturns;
  
  event AuctionEnded(address winner, uint amount);

  constructor (address _beneficiary, uint _durationBiddingMinutes, uint _durationRevealMinutes) 	{
    beneficiary = _beneficiary;
    biddingEnd = block.timestamp + _durationBiddingMinutes * 1 minutes;
    revealEnd = biddingEnd + _durationRevealMinutes * 1 minutes;
  }
  
  function withdraw() external returns (uint amount) {
    amount = pendingReturns[msg.sender];
    if (amount > 0) {
      pendingReturns[msg.sender] = 0;
      payable(msg.sender).transfer(amount);
    }
  }
  
  function generateSealedBid(uint _bidAmount, _bool isLegit, string memory _secret) public pure returns (bytes32 sealedBid) {
			sealedBid = keccak256(abi.encodePacked(_bidAmount, _isLegit, _secret));
	}
}
	
\end{lstlisting}


\end{frame}
%%%

%%%
\begin{frame}{Structs}

	\textbf{Purpose:}
	
	\begin{itemize}
		\item Keep track of hashed sealed bids with the corresponding deposit amount.
		\item Complex user defined types with any number of properties.
	\end{itemize}
	
	\uncover<2->{
		\begin{lstlisting}[language=Solidity]
struct Bid {
	bytes32 sealedBid;
	uint deposit;
}
\end{lstlisting}
	}
	
	\uncover<3->{
		\texttt{Bid} can now be used as a variable type, e.g.,\\

		\begin{samplecode}{}
		Bid newBid = Bid(generateSealedBid(50e18, true, "secret"), 50e18);
		\end{samplecode}
	}
	
	\uncover<4->{
		\begin{keytakeaway}{Structs usage}
			\begin{itemize}
				\item Structs can be used in mappings and arrays.
				\item Structs can contain mappings and arrays.
			\end{itemize}
		\end{keytakeaway}
	}
	
\end{frame}
%%%

%%%
\begin{frame}{Arrays}

	\textbf{Idea:} use mapping to store one bid per address:
	
	\begin{samplecode}{}
		mapping(address => Bid) bids;
	\end{samplecode}
	
	\uncover<2->{
		\textbf{Problem:} what if users want to create multiple bids?
	
	$\rightarrow$ Use a variable sized list of elements that is enumerable: Dynamic Arrays.\\
	
	\vspace{1em}
	
	\textbf{Arrays in Solidity:}
	\begin{itemize}
			\item \texttt{T[<k>]}: Fixed size array of type \texttt{T} and length \texttt{k}.
			\item \texttt{T[]}: Dynamic size array of type \texttt{T}.
		\end{itemize}
		
	
	\begin{keytakeaway}{Array properties and methods}
		Both array types have the \texttt{.length()} property. Fixed arrays will return \texttt{k}, dynamic arrays the current length.\\
		Dynamic arrays have the \texttt{.push(<value>)} and \texttt{.pop()} methods to add or remove an element at the end of the array.
	\end{keytakeaway}
	}

\end{frame}
%%%

%%%
\begin{frame}{Store Bids per Address}

	Use a mapping to store a dynamic array (a variable size list) for each address:\\
	
	\begin{lstlisting}
struct Bid {
	bytes32 sealedBid;
	uint deposit;
}
mapping(address => Bid[]) bids;	
\end{lstlisting}
	
\end{frame}
%%%

%%%
\begin{frame}{Commit and Reveal Exercise}
	
	\begin{exercise}{Exercise 1}
		\textbf{Preparation:}
		\begin{itemize}
			\item Read the introduction for the \href{https://docs.ens.domains/contract-api-reference/.eth-permanent-registrar/controller}{\link Ethereum Name Service (ENS) \texttt{ETHRegistrarController}} and the \href{https://docs.ens.domains/contract-api-reference/.eth-permanent-registrar/controller\#calculate-commitment-hash}{\link \texttt{makeCommitment}} description.
			\item Check out the \texttt{makeCommitment} function in the ''Read Code" section of the deployed contract on \href{https://etherscan.io/address/0x283af0b28c62c092c9727f1ee09c02ca627eb7f5\#readContract}{\link etherscan}.
		\end{itemize}
		
		\textbf{Question 1:} What is the output of the function if you use
					
		\begin{itemize}
			\item \texttt{name: vitalik}
			\item \texttt{owner: 0xd8dA6BF26964aF9D7eEd9e03E53415D37aA96045}
			\item \texttt{secret: 0x6162636400000000000000000000000000000000000000000000000000000000}
		\end{itemize}
		
		as the input values?
					
	\end{exercise}	
\end{frame}
%%%

%%%
\begin{frame}{Commit and Reveal Exercise}

	\begin{exercise}{Exercise 1 (part 2)}
	
		\textbf{Question 2:} Assume someone created the commitment hash \texttt{0x5af80c257639b6180b3d8e91ad2fefa8006afe66bbc98f2e46384e7ccefbe823}. They used the \texttt{owner} and \texttt{secret} from \textbf{Question 1}. Which one of the following names did they want to register?
			
		\vspace{-0.5em}
		\begin{multicols}{2}
				\begin{itemize}
					\item[A.] vitalik
					\item[C.] mary
					\item[B.] aaron
					\item[D.] patricia
				\end{itemize}
			\end{multicols}
		
		\uncover<2->{
		\textbf{Question 3:} Assume someone created the commitment hash \texttt{0xedc18ec53ab6729380c138ef6ab3f04c1a73fec9540cfc09d5ea84ff27ebe796}. They used the \texttt{owner} from \textbf{Question 1}. You do not know the \texttt{secret} they used. Are you able to find out which one of the four names from \textbf{Question 2} they wanted to register?
		}
	\end{exercise}
	
%	\uncover<3->{
%		\begin{exercise}{Solution to question 3: very low probability of success (for verification purposes only)}
%			\begin{itemize}
%						\item \texttt{name: patricia}
%						\item \texttt{owner: 0xd8dA6BF26964aF9D7eEd9e03E53415D37aA96045}
%						\item \texttt{secret:\\
%					0x61626364000000009990000000000120000000034000005600000abcdefabcd}
%			\end{itemize}
%		\end{exercise}
%	}
\end{frame}
%%%

%%%
\begin{frame}{Commit and Reveal Exercise}

	\begin{exercise}{Exercise 2}
		\begin{enumerate}
			\item Create a new contract file named SealedBidAuction.sol
			\item Copy the SimpleAuction.sol code to the new file.
			\item Delete the \texttt{bid()} function and \texttt{auctionEnd()} function. Also delete any associated events.
			\item Add the \texttt{biddingEnd} and \texttt{revealEnd} variables and set them as part of the \texttt{constructor()}.
			\item Add the \texttt{generateSealedBid()} function.
			\item Deploy the contract and test the \texttt{generateSealedBid()} function. Note that we have not yet reimplemented the rest of the auction contract, i.e. the bidding and resolution part. 
		\end{enumerate}
	\end{exercise}
	
\end{frame}
%%%

\end{document}