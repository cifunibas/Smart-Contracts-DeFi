% Choose one to switch between slides and handout
\documentclass[]{beamer}
%\documentclass[handout]{beamer}

% Video Meta Data
\title{Smart Contracts and Decentralized Finance}
\subtitle{Functions and Modifiers}
\author{Prof. Dr. Fabian Schär}
\institute{University of Basel}

% Config File
% Packages
\usepackage[utf8]{inputenc}
\usepackage{hyperref}
\usepackage{gitinfo2}
\usepackage{tikz}
 \usetikzlibrary{calc}
\usepackage{amsmath}
\usepackage{mathtools}
\usepackage{bibentry}
\usepackage{xcolor}
\usepackage{colortbl} % Add colour to LaTeX tables
\usepackage{caption}
\usepackage[export]{adjustbox}
\usepackage{pgfplots} \pgfplotsset{compat = 1.17}
\usepackage{makecell}
\usepackage{fancybox}
\usepackage{ragged2e}
\usepackage{fontawesome}
\usepackage{seqsplit}
\usepackage{tabularx}
\usepackage{tcolorbox}
\usepackage{booktabs} % use instead  \hline in tables

% Color Options
\definecolor{highlight}{rgb}{0.65,0.84,0.82}
\definecolor{focus}{rgb}{0.72, 0, 0}
\definecolor{lightred}{rgb}{0.8,0.5,0.5}
\definecolor{midgray}{RGB}{190,195,200}

 %UniBas Main Colors
\definecolor{mint}{RGB}{165,215,210}
\definecolor{anthracite}{RGB}{45,55,60}
\definecolor{red}{RGB}{210,5,55}

 %UniBas Color Palette (for graphics)
\definecolor{strongmint}{RGB}{30,165,165}
\definecolor{darkmint}{RGB}{0,110,110}
\definecolor{softanthracite}{RGB}{140,145,150}
\definecolor{brightanthracite}{RGB}{190,195,200}
\definecolor{softred}{RGB}{235,130,155}

%Custom Colors
\definecolor{lightergray}{RGB}{230, 230, 230}



% Beamer Template Options
\beamertemplatenavigationsymbolsempty
\setbeamertemplate{footline}[frame number]
\setbeamercolor{structure}{fg=black}
\setbeamercolor{footline}{fg=black}
\setbeamercolor{title}{fg=black}
\setbeamercolor{frametitle}{fg=black}
\setbeamercolor{item}{fg=black}
\setbeamercolor{}{fg=black}
\setbeamercolor{bibliography item}{fg=black}
\setbeamercolor*{bibliography entry title}{fg=black}
\setbeamercolor{alerted text}{fg=focus}
\setbeamertemplate{items}[square]
\setbeamertemplate{enumerate items}[default]
\captionsetup[figure]{labelfont={color=black},font={color=black}}
\captionsetup[table]{labelfont={color=black},font={color=black}}

\setbeamertemplate{bibliography item}{\insertbiblabel}

%tcolor boxes
\newtcolorbox{samplecode}[2][]{
  colback=mint, colframe=darkmint, coltitle=white,
  fontupper = \ttfamily\scriptsize, fonttitle= \bfseries\scriptsize,
  boxrule = 0mm, arc = 0mm,
  boxsep = 1.3mm, left = 0mm, right = 0mm, top = 0.5mm, bottom = 0mm, middle=0mm,
  #1,title=#2}
  
\newtcolorbox{keytakeaway}[2][]{
  colback=softred, colframe=red, coltitle=white,
  fontupper = \scriptsize, fonttitle= \bfseries\scriptsize,
  boxrule = 0mm, arc = 0mm,
  boxsep = 1.3mm, left = 0mm, right = 0mm, top = 0.5mm, bottom = 0mm, middle=0mm,
  #1,title=#2}

\newtcolorbox{exercise}[2][]{
  colback=brightanthracite, colframe=anthracite, coltitle=white,
  fontupper = \scriptsize, fonttitle= \bfseries\scriptsize,
  boxrule = 0mm, arc = 0mm,
  boxsep = 1.3mm, left = 0mm, right = 0mm, top = 0.5mm, bottom = 0mm, middle=0mm,
  #1,title=#2}



% Link Icon Command 
\newcommand{\link}{%
    \tikz[x=1.2ex, y=1.2ex, baseline=-0.05ex]{%
        \begin{scope}[x=1ex, y=1ex]
            \clip (-0.1,-0.1)
                --++ (-0, 1.2)
                --++ (0.6, 0)
                --++ (0, -0.6)
                --++ (0.6, 0)
                --++ (0, -1);
            \path[draw,
                line width = 0.5,
                rounded corners=0.5]
                (0,0) rectangle (1,1);
        \end{scope}
        \path[draw, line width = 0.5] (0.5, 0.5)
            -- (1, 1);
        \path[draw, line width = 0.5] (0.6, 1)
            -- (1, 1) -- (1, 0.6);
        }
    }

% Other commands
\newcommand\tab[1][0.5cm]{\hspace*{#1}} % for code boxes


% Read Git Data from Github Actions Workflow
% Defaults to gitinfo2 for local builds
\IfFileExists{gitInfo.txt}
	{\input{gitInfo.txt}}
	{
		\newcommand{\gitRelease}{(Local Release)}
		\newcommand{\gitSHA}{\gitHash}
		\newcommand{\gitDate}{\gitAuthorIsoDate}
	}

% Custom Titlepage
\defbeamertemplate*{title page}{customized}[1][]
{
  \vspace{-0cm}\hfill\includegraphics[width=2.5cm]{../config/logo_cif}
  \includegraphics[width=1.9cm]{../config/seal_wwz}
  \\ \vspace{2em}
  \usebeamerfont{title}\textbf{\inserttitle}\par
  \usebeamerfont{title}\usebeamercolor[fg]{title}\insertsubtitle\par  \vspace{1.5em}
  \small\usebeamerfont{author}\insertauthor\par
  \usebeamerfont{author}\insertinstitute\par \vspace{2em}
  \usebeamercolor[fg]{titlegraphic}\inserttitlegraphic
    \tiny \noindent \texttt{Release Ver.: \gitRelease}\\ 
    \texttt{Version Hash: \gitSHA}\\
    \texttt{Version Date: \gitDate}\\ \vspace{1em}
    
    
    \iffalse
  \link \href{https://github.com/cifunibas/Bitcoin-Blockchain-Cryptoassets/blob/main/slides/intro.pdf}
  {Get most recent version}\\
  \link \href{https://github.com/cifunibas/Bitcoin-Blockchain-Cryptoassets/blob/main/slides/intro.pdf}
  {Watch video lecture}\\ 
  
  \fi
  
  \vspace{1em}
  License: \texttt{Creative Commons Attribution-NonCommercial-ShareAlike 4.0 International}\\\vspace{2em}
  \includegraphics[width = 1.2cm]{../config/license}
}


% tikzlibraries
\usetikzlibrary{decorations.pathreplacing}
\usetikzlibrary{decorations.markings}
\usetikzlibrary{positioning}
\usetikzlibrary{calc}
\captionsetup{font=footnotesize}

%%%%%%%%%%%%%%%%%%%%%%%%%%%%%%%%%%%%%%%%%%%%%%
%%%%%%%%%%%%%%%%%%%%%%%%%%%%%%%%%%%%%%%%%%%%%%
\begin{document}

\thispagestyle{empty}
\begin{frame}[noframenumbering]
	\titlepage
\end{frame}
%%%

%%%
\begin{frame}{Functions}
	\textbf{Functions are executable parts of a smart contract.}
		
	\uncover<2->{
		\begin{samplecode}{Function Example}
			\begin{lstlisting}[language=Solidity]
	function <functionName>(<parameters>) <modifiers> returns (<return variables>) {
		<function body>
	}
\end{lstlisting}
		\end{samplecode}
	}
	
	\scriptsize
	\begin{columns}[T]
		\begin{column}{0.48\textwidth}
		
		\uncover<3->{
			\texttt{<functionName>}\\
			\begin{itemize}
				\item Pick and set a unique name for your function	
			\end{itemize}
		}
		
		\uncover<4->{
			\texttt{<parameters>}\\
			\begin{itemize}
				\item Used in the \texttt{<function body>}
				\item Must be different from state variables	
			\end{itemize}
		}
		
		\end{column}
		\begin{column}{0.48\textwidth}
		
		\uncover<5->{
			\texttt{<modifiers>}\\
			\begin{itemize}
				\item Accessibility modifiers
				\item State permission modifiers
				\item Special modifiers
				\item Custom modifiers	
			\end{itemize}
		}
		
		\uncover<6->{
			\texttt{<return variables>}
			\begin{itemize}
				\item Optional
				\item Multiple return variables possible	
			\end{itemize}
		}

		\end{column}
	\end{columns}
\end{frame}
%%%

%%%
\begin{frame}{Accessibility Modifiers}
	\textbf{Accessibility modifiers state which accounts can access a function or variable.}
	
	\begin{itemize}
		\item<2-> They are explicitly required for all functions.
		\item<2-> They can be defined for variables, but default to \genkey{private} if omitted. 
	\end{itemize}
	
	\uncover<3->{
	\textbf{Who can access which functions/variables?}
		\begin{table}
			\begin{tabular}{lp{1.6cm}p{1.6cm}p{1.6cm}p{1.6cm}}
			\hline
				\rowcolor{highlight}
				\textbf{Keyword} & \textbf{EOA} & \textbf{This contract} & \textbf{Inheriting contract} & \textbf{External contract}\\
				\genkey{private} & No & Yes & No & No \\
				\genkey{internal} & No & Yes & Yes & No \\
				\genkey{external} & Yes & No & No & Yes \\
				\genkey{public} & Yes & Yes & Yes & Yes \\
				\hline
			\end{tabular}
		\end{table}
	}
	
	\uncover<4->{
	\begin{keytakeaway}{Additional information}
		\begin{itemize}
			\item \genkey{external} can only be used for functions.
			\item Declaring a variable as \genkey{public} will create a getter function with the same name.
			\item \genkey{private} does not mean the variable is invisible.
		\end{itemize}
	\end{keytakeaway}
	}
\end{frame}
%%%

%%%
\begin{frame}{State Permission Modifiers}
	\textbf{State permission modifiers define which functions can read from or modify (write to) the state.}
	\uncover<2->{
		\begin{table}
			\begin{tabular}{lll}
				\hline
				\rowcolor{highlight}
				\textbf{Keyword} & \textbf{Read} & \textbf{Write}\\
				\genkey{pure} & No & No\\
				\genkey{view} & Yes & No\\
				\textit{$<$omitted$>$} & Yes & Yes\\
				\hline
			\end{tabular}
		\end{table}
	}
\end{frame}
%%%

%%%
\begin{frame}{Special and Custom Modifiers}
\textbf{Special modifiers for functions:}
	\begin{itemize}
		\item<1-> \genkey{payable}: allows a function to receive ETH as part of the transaction.
		\item<2-> \genkey{virtual} \& \genkey{override}: used for inheritance (\textcolor{focus}{more on inheritance later!}).
		\item<3-> \genkey{constant} \& \genkey{immutable}: disallow changing the value of a variable during the contract's lifetime.
		\begin{itemize}
			\item<3->{Difference: \genkey{immutable} can be set during contract deployment}	
		\end{itemize}
		\item<4-> Custom (user-defined) modifiers (\textcolor{focus}{more on this later!})
	\end{itemize}
\end{frame}
%%%

%%%
\begin{frame}{Update of the Auction Contract}
	\begin{exercise}{Exercise:}
		\begin{enumerate}
			\item Make sure the variables \texttt{beneficiary}, \texttt{highestBid}, \texttt{highestBidder} and \texttt{hasEnded} are \genkey{public}.
			\item Create a function that allows any EOA or external contract to set the \texttt{beneficiary}.
			\item Self-check: how can you get the values of the previous variables?
		\end{enumerate}
	\end{exercise}
\end{frame}
%%%

%%%
\begin{frame}{Update the Auction Contract}
	\begin{samplecode}{Solution}
		\begin{lstlisting}[language=Solidity,escapechar=|]
contract SimpleAuction {
  // Auction parameters
  address public beneficiary;|\label{line:benef}|
  
  // State of the auction
  uint public highestBid;
  address public highestBidder;
  bool public hasEnded;

  function initialize(address _beneficiary) external {
    beneficiary = _beneficiary;
  }
}
\end{lstlisting}
	\end{samplecode}
\end{frame}
%%%

%%%
\begin{frame}{Contract Constructors}

	\textbf{The \genkey{constructor} is a special function that is executed when the contract is deployed.}
	
	\begin{samplecode}{Constructor Example}
			\begin{lstlisting}[language=Solidity]
constructor(<parameters>) {
  <constructor body>
}
\end{lstlisting}
	\end{samplecode}
	
\end{frame}

\begin{frame}{Immutable \texttt{beneficiary}}

	\begin{exercise}{Exercise:}
		\begin{enumerate}
			\item Remove \genkey{function} \texttt{initialize(}\typesunits{address}\texttt{\_beneficiary)} \genkey{external} \texttt{\{\dots\}} from the auction contract.
			\item Use the \genkey{constructor} to set the \texttt{beneficiary} at the time of contract deployment for the contract's entire lifetime.
		\end{enumerate}
	\end{exercise}
		
	\uncover<2->{
	\textbf{Solution:}
	\begin{samplecode}{Constructor for the Auction Contract}
			\begin{lstlisting}[language=Solidity]
constructor(address _beneficiary) {
  beneficiary = _beneficiary;
}
\end{lstlisting}
	\end{samplecode}
	
	\begin{samplecode}{Line \ref{line:benef} in the previous contract}
		address public immutable beneficiary;
	\end{samplecode}
	}
	
\end{frame}
%%%

%%%
\begin{frame}{Current State of the Auction Contract}
	\begin{samplecode}{After Integrating the Constructor into \texttt{SimpleAuction}}
		\begin{lstlisting}[language=Solidity]
// SPDX-License-Identifier: MIT
pragma solidity ^0.8.9;

contract SimpleAuction {
  // Auction parameters
  address public immutable beneficiary;
  
  // State of the auction
  uint public highestBid;
  address public highestBidder;
  bool public hasEnded;

  constructor (address _beneficiary) {
    beneficiary = _beneficiary;
  }
}
\end{lstlisting}
	\end{samplecode}
\end{frame}
%%%

\end{document}