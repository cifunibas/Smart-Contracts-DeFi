% Choose one to switch between slides and handout
\documentclass[]{beamer}
%\documentclass[handout]{beamer}

% Video Meta Data
\title{Smart Contracts and Decentralized Finance Applications}
\subtitle{Gas and Fees}
\author{Prof. Dr. Fabian Schär}
\institute{University of Basel}

% Config File
% Packages
\usepackage[utf8]{inputenc}
\usepackage{hyperref}
\usepackage{gitinfo2}
\usepackage{tikz}
 \usetikzlibrary{calc}
\usepackage{amsmath}
\usepackage{mathtools}
\usepackage{bibentry}
\usepackage{xcolor}
\usepackage{colortbl} % Add colour to LaTeX tables
\usepackage{caption}
\usepackage[export]{adjustbox}
\usepackage{pgfplots} \pgfplotsset{compat = 1.17}
\usepackage{makecell}
\usepackage{fancybox}
\usepackage{ragged2e}
\usepackage{fontawesome}
\usepackage{seqsplit}
\usepackage{tabularx}
\usepackage{tcolorbox}
\usepackage{booktabs} % use instead  \hline in tables

% Color Options
\definecolor{highlight}{rgb}{0.65,0.84,0.82}
\definecolor{focus}{rgb}{0.72, 0, 0}
\definecolor{lightred}{rgb}{0.8,0.5,0.5}
\definecolor{midgray}{RGB}{190,195,200}

 %UniBas Main Colors
\definecolor{mint}{RGB}{165,215,210}
\definecolor{anthracite}{RGB}{45,55,60}
\definecolor{red}{RGB}{210,5,55}

 %UniBas Color Palette (for graphics)
\definecolor{strongmint}{RGB}{30,165,165}
\definecolor{darkmint}{RGB}{0,110,110}
\definecolor{softanthracite}{RGB}{140,145,150}
\definecolor{brightanthracite}{RGB}{190,195,200}
\definecolor{softred}{RGB}{235,130,155}

%Custom Colors
\definecolor{lightergray}{RGB}{230, 230, 230}



% Beamer Template Options
\beamertemplatenavigationsymbolsempty
\setbeamertemplate{footline}[frame number]
\setbeamercolor{structure}{fg=black}
\setbeamercolor{footline}{fg=black}
\setbeamercolor{title}{fg=black}
\setbeamercolor{frametitle}{fg=black}
\setbeamercolor{item}{fg=black}
\setbeamercolor{}{fg=black}
\setbeamercolor{bibliography item}{fg=black}
\setbeamercolor*{bibliography entry title}{fg=black}
\setbeamercolor{alerted text}{fg=focus}
\setbeamertemplate{items}[square]
\setbeamertemplate{enumerate items}[default]
\captionsetup[figure]{labelfont={color=black},font={color=black}}
\captionsetup[table]{labelfont={color=black},font={color=black}}

\setbeamertemplate{bibliography item}{\insertbiblabel}

%tcolor boxes
\newtcolorbox{samplecode}[2][]{
  colback=mint, colframe=darkmint, coltitle=white,
  fontupper = \ttfamily\scriptsize, fonttitle= \bfseries\scriptsize,
  boxrule = 0mm, arc = 0mm,
  boxsep = 1.3mm, left = 0mm, right = 0mm, top = 0.5mm, bottom = 0mm, middle=0mm,
  #1,title=#2}
  
\newtcolorbox{keytakeaway}[2][]{
  colback=softred, colframe=red, coltitle=white,
  fontupper = \scriptsize, fonttitle= \bfseries\scriptsize,
  boxrule = 0mm, arc = 0mm,
  boxsep = 1.3mm, left = 0mm, right = 0mm, top = 0.5mm, bottom = 0mm, middle=0mm,
  #1,title=#2}

\newtcolorbox{exercise}[2][]{
  colback=brightanthracite, colframe=anthracite, coltitle=white,
  fontupper = \scriptsize, fonttitle= \bfseries\scriptsize,
  boxrule = 0mm, arc = 0mm,
  boxsep = 1.3mm, left = 0mm, right = 0mm, top = 0.5mm, bottom = 0mm, middle=0mm,
  #1,title=#2}



% Link Icon Command 
\newcommand{\link}{%
    \tikz[x=1.2ex, y=1.2ex, baseline=-0.05ex]{%
        \begin{scope}[x=1ex, y=1ex]
            \clip (-0.1,-0.1)
                --++ (-0, 1.2)
                --++ (0.6, 0)
                --++ (0, -0.6)
                --++ (0.6, 0)
                --++ (0, -1);
            \path[draw,
                line width = 0.5,
                rounded corners=0.5]
                (0,0) rectangle (1,1);
        \end{scope}
        \path[draw, line width = 0.5] (0.5, 0.5)
            -- (1, 1);
        \path[draw, line width = 0.5] (0.6, 1)
            -- (1, 1) -- (1, 0.6);
        }
    }

% Other commands
\newcommand\tab[1][0.5cm]{\hspace*{#1}} % for code boxes


% Read Git Data from Github Actions Workflow
% Defaults to gitinfo2 for local builds
\IfFileExists{gitInfo.txt}
	{\input{gitInfo.txt}}
	{
		\newcommand{\gitRelease}{(Local Release)}
		\newcommand{\gitSHA}{\gitHash}
		\newcommand{\gitDate}{\gitAuthorIsoDate}
	}

% Custom Titlepage
\defbeamertemplate*{title page}{customized}[1][]
{
  \vspace{-0cm}\hfill\includegraphics[width=2.5cm]{../config/logo_cif}
  \includegraphics[width=1.9cm]{../config/seal_wwz}
  \\ \vspace{2em}
  \usebeamerfont{title}\textbf{\inserttitle}\par
  \usebeamerfont{title}\usebeamercolor[fg]{title}\insertsubtitle\par  \vspace{1.5em}
  \small\usebeamerfont{author}\insertauthor\par
  \usebeamerfont{author}\insertinstitute\par \vspace{2em}
  \usebeamercolor[fg]{titlegraphic}\inserttitlegraphic
    \tiny \noindent \texttt{Release Ver.: \gitRelease}\\ 
    \texttt{Version Hash: \gitSHA}\\
    \texttt{Version Date: \gitDate}\\ \vspace{1em}
    
    
    \iffalse
  \link \href{https://github.com/cifunibas/Bitcoin-Blockchain-Cryptoassets/blob/main/slides/intro.pdf}
  {Get most recent version}\\
  \link \href{https://github.com/cifunibas/Bitcoin-Blockchain-Cryptoassets/blob/main/slides/intro.pdf}
  {Watch video lecture}\\ 
  
  \fi
  
  \vspace{1em}
  License: \texttt{Creative Commons Attribution-NonCommercial-ShareAlike 4.0 International}\\\vspace{2em}
  \includegraphics[width = 1.2cm]{../config/license}
}


% tikzlibraries
\usetikzlibrary{decorations.pathreplacing}
\usetikzlibrary{decorations.markings}
\usetikzlibrary{positioning}
\usetikzlibrary{calc}
\captionsetup{font=footnotesize}

%%%%%%%%%%%%%%%%%%%%%%%%%%%%%%%%%%%%%%%%%%%%%%
%%%%%%%%%%%%%%%%%%%%%%%%%%%%%%%%%%%%%%%%%%%%%%
\begin{document}

\thispagestyle{empty}
\begin{frame}[noframenumbering]
	\titlepage
\end{frame}

%%%	
\begin{frame}{The Halting Problem and its Implications}
\begin{columns}
	\begin{column}{0.3\textwidth}
		\center
		\includegraphics[width=\textwidth , frame]{../assets/images/alan-turing.png}
	\end{column}
	\begin{column}{0.7\textwidth}
		In 1936, Alan Turing proved that it's impossible for a (turing) machine to inspect any code that you give it, and correctly tell you whether the code will halt or run forever.\\
		\vspace{1em}
		\link \href{https://www.cs.virginia.edu/~robins/Turing_Paper_1936.pdf}{Online PDF}\\
		\vspace{1em}	
		\uncover<2->{Implication: Infinite loops or resource intensive scripts are a potential attack vector and must be dealt with.}\\
	\end{column}
\end{columns}
\uncover<3->{
	\begin{alertblock}{Bitcoin and Ethereum Deal With This Problem in Different Ways}
		\begin{itemize}
			\item Bitcoin Script is a restricted scripting language.
			\item Ethereum has a Turing complete instruction set. It charges a small fee (gas) for each computation step.
		\end{itemize}
	\end{alertblock}
}
\end{frame}
%%%

%%%
%% TODO: Flexible block size limit erklären --> evtl. separates slide für EIP-1559 related changes
\begin{frame}{Ethereum Gas Fees: The Basics}
	\begin{itemize}
		\item Every EVM operation consumes a pre-defined amount of gas. These values cannot be changed by user. Some examples\footnote{\href{https://docs.google.com/spreadsheets/d/1m89CVujrQe5LAFJ8-YAUCcNK950dUzMQPMJBxRtGCqs}{Full list} \link}:
			\begin{columns}
				\uncover<2->{
				\begin{column}{0.08\textwidth} %% TODO: is there a better way?
				\end{column}
				\begin{column}{0.25\textwidth}
					\begin{exampleblock}{Addition}
						\begin{center}
							\includegraphics[width=2em]{../assets/images/plus.png}
						\end{center}
						$\rightarrow$ 3 gas units 
					\end{exampleblock}				
				\end{column}
				}
				\uncover<3->{
				\begin{column}{0.25\textwidth}
					\begin{exampleblock}{Multiplication}
						\begin{center}
							\includegraphics[width=2em]{../assets/images/multiply.png}
						\end{center}
						$\rightarrow$ 5 gas units 
					\end{exampleblock}
				\end{column}
				}
				\uncover<4->{
				\begin{column}{0.25\textwidth}
					\begin{exampleblock}{Store 256 bits}
						\begin{center}
							\includegraphics[width=2em]{../assets/images/database.png}
						\end{center}
						$\rightarrow$ 20k gas units 
					\end{exampleblock}	
				\end{column}
				}
				\begin{column}{0.02\textwidth} %% TODO: is there a better way?
				\end{column}
			\end{columns}
		\uncover<5->{
		\hspace{1em}
		\item Every transaction has a user-specified \textbf{gas price}, i.e. the price in $\mu ETH$ or GWEI the user is willing to pay per unit of gas.
		}
		\uncover<6->{
		\item Every transaction has a user-specified \textbf{gas limit}, i.e. the maximum number of gas units that can be in this transaction.
		\hspace{-1em}
		\begin{itemize}
			\item 21000 required to send ETH
			\item Contract execution varies
		\end{itemize}
		}
	\end{itemize} 
\end{frame}
%%%

%%%
%TODO: Add Gas price pre and post EIP-1559
%%%

%%%
% TODO: remainingGas ist etwas verwirrend (mal in ETH, mal nicht)
\begin{frame}{Understanding the Ethereum Fee Process}
\begin{table}
	\tiny
	{\renewcommand{\arraystretch}{2}%
  \center
  \begin{tabular}[]{m{0.19\textwidth} | m{0.39\textwidth}| m{0.35\textwidth}}
  		\hline \hline
		\textbf{Trigger} & \textbf{Consequence} & \textbf{remainingGas in ETH}\\
		\hline
			Transaction initiation
			& User sets \textbf{gasLimit} and \textbf{gasPrice} and places an amount in GWEI equal to the product of the two in \textbf{transactionEscrow}. \newline \newline \textbf{transactionEscrow = gasLimit $\cdot$ gasPrice}
			& \textbf{remainingGas = gasLimit $\cdot$ gasPrice}\\
		\hline
			For each operation
			& The \textbf{remainingGas} is decreased by operation’s gas consumption (\textbf{opCost}).
			& \textbf{remainingGas = remainingGas - opCost}\\
		\hline
			If \textbf{remainingGas} $\leq$ 0
			& If \textbf{remainingGas} is 0 or insufficient and there are operations remaining, the transaction is reverted (but still included in the Blockchain and \textbf{gasLimit $\cdot$ gasPrice} paid as fee).
			& \textbf{remainingGas} $\leq$ 0\\
		\hline
			If all operations are successful
			& State changes get applied and any unused ETH returned to the sender of the transaction. \newline \newline
			\textbf{Refund = remainingGas $\cdot$ gasPrice} \newline \newline % TODO: remove gasPrice?
			\textbf{Fee = (gasLimit - remainingGas) $\cdot$ gasPrice} % TODO: remove gasPrice?
			& Is multiplied with \textbf{gasPrice} and then refunded to the transaction originator’s address\\% TODO: remove "multiplied with gasPrice and then"?
		\hline
		\multicolumn{3}{r}{All values in GWEI = ETH $\cdot$ 10$^{-9}$}\\
  \end{tabular}}
\end{table}
\end{frame}
%%%

%%%
\begin{frame}{A Gas Consumption Example}
	\begin{center}
	 	\begin{tabular}{m{0.2\textwidth} m{0.3\textwidth}}
	 		\includegraphics[width=2cm]{../assets/images/gas-pump.png} &
			\parbox{\textwidth}{gasLimit: 100 \\
    		remainingGas: \only<1>{100}\textcolor{red}{\only<2>{97}\only<3>{92}\only<4->{62}} \\
    		gasPrice: x}
 		\end{tabular}
	\begin{tikzpicture}
		\begin{footnotesize}
	\uncover<1->{
		\node (avatar1) at (-4.5, 3) {\includegraphics[width=.15\textwidth]{../assets/images/avatar.png}};
		\node (start) at (-4.5, 0) [circle,fill=mint,minimum size=1.2cm, label={south,mint:+100}] {Start};
		\draw[->, dotted, very thick] (avatar1) -- (start) node[midway, right, mint] (text1) {100 $\cdot$ x GWEI};
	}
	\uncover<2->{
		\node (add) at (-2.5,0) [rectangle,fill=mint,minimum width=1.9cm,minimum height=0.6cm, label={south,red:-3}] {ADD};
	}
	\uncover<3->{
		\node (mul) at (0,0) [rectangle,fill=mint,minimum width=1.9cm,minimum height=0.6cm, label={south,red:-5}] {MUL};
	}
	\uncover<4->{
		\node (sha3base) at (2.5,0) [rectangle,fill=mint,minimum width=1.9cm,minimum height=0.6cm,label={south,red:-30}] {SHA3BASE};
		\node (rect) at (-3.7,-0.1) [draw,dotted,thick,minimum width=7.4cm,minimum height=1.5cm, right,label=south:EVM Operations]{};
	}
	\uncover<5->{
		\node (end) at (4.5, 0) [circle,fill=mint,minimum size=1.2cm] {End};
		\node (avatar2) at (4.5, 3) {\includegraphics[width=.15\textwidth]{../assets/images/avatar.png}};
		\draw[->, dotted, very thick] (end) -- (avatar2) node[midway, left, mint] (text2) {62 $\cdot$ x GWEI};
	}
\end{footnotesize}
	\end{tikzpicture}
	\end{center}
\end{frame}
%%%

%%%
\begin{frame}{Essential Tools: Ethereum Gas Tracker}
	\begin{figure}
	\centering
	\begin{minipage}{.45\textwidth}
  		\centering
  		\includegraphics[width=0.9\textwidth]{../assets/images/etherscan.png}
  		\caption*{\footnotesize \href{https://etherscan.io/}{\link Etherscan}}
	\end{minipage}
	\begin{minipage}{.45\textwidth}
  		\centering
  		\includegraphics[width=0.9\textwidth]{../assets/images/eth-gas-station.png}
  		\caption*{\footnotesize \href{https://www.Ethgasstation.info}{\link Eth Gas Station}}
	\end{minipage}
	\end{figure}
\end{frame}
%%%

%%%
\begin{frame}{Advantages of the Gas System}
	\textbf{Fees help to secure the network:}
	\begin{itemize}
		\item Stops execution of…
		\begin{itemize}
			\item Badly written code
			\item Infinite loops
		\end{itemize}
		\item Reduce spam transactions
		\item Establish a market for priority access to network
	\end{itemize}
	
\begin{block}{Exercise 3} % TODO: An EIP-1559 anpassen
\begin{itemize}
	\item Use your Metamask account to issue an ETH transaction with a gasPrice of 100 GWEI. How large do you expect the fee to be considering a gasLimit of 21000?
	\item Use your Metamask account to issue an ETH transaction with a gasPrice of 0 GWEI and a gasLimit 0f 21000. What do you expect to happen?
	\item Use your Metamask account to issue an ETH transaction with a gasPrice of 100 GWEI and a gasLimit of 1000000. What do you expect to happen?
	\item Use your Metamask account to issue an ETH transaction with a gasPrice of 100 GWEI and a gasLimit of 1000. What do you expect to happen?
\end{itemize}
\end{block}
\end{frame}
%%%

%%%
\begin{frame}{Gasprice Market - Some Interesting Observations}
	% TODO: Add gas price chart
\end{frame}
%%%

\end{document}