% Choose one to switch between slides and handout
%\documentclass[]{beamer}
\documentclass[handout]{beamer}

% Video Meta Data
\title{Smart Contracts and Decentralized Finance}
\subtitle{Global Variables, Transfers, and Events}
\author{Prof. Dr. Fabian Schär}
\institute{University of Basel}

% Config File
% Packages
\usepackage[utf8]{inputenc}
\usepackage{hyperref}
\usepackage{gitinfo2}
\usepackage{tikz}
 \usetikzlibrary{calc}
\usepackage{amsmath}
\usepackage{mathtools}
\usepackage{bibentry}
\usepackage{xcolor}
\usepackage{colortbl} % Add colour to LaTeX tables
\usepackage{caption}
\usepackage[export]{adjustbox}
\usepackage{pgfplots} \pgfplotsset{compat = 1.17}
\usepackage{makecell}
\usepackage{fancybox}
\usepackage{ragged2e}
\usepackage{fontawesome}
\usepackage{seqsplit}
\usepackage{tabularx}
\usepackage{tcolorbox}
\usepackage{booktabs} % use instead  \hline in tables

% Color Options
\definecolor{highlight}{rgb}{0.65,0.84,0.82}
\definecolor{focus}{rgb}{0.72, 0, 0}
\definecolor{lightred}{rgb}{0.8,0.5,0.5}
\definecolor{midgray}{RGB}{190,195,200}

 %UniBas Main Colors
\definecolor{mint}{RGB}{165,215,210}
\definecolor{anthracite}{RGB}{45,55,60}
\definecolor{red}{RGB}{210,5,55}

 %UniBas Color Palette (for graphics)
\definecolor{strongmint}{RGB}{30,165,165}
\definecolor{darkmint}{RGB}{0,110,110}
\definecolor{softanthracite}{RGB}{140,145,150}
\definecolor{brightanthracite}{RGB}{190,195,200}
\definecolor{softred}{RGB}{235,130,155}

%Custom Colors
\definecolor{lightergray}{RGB}{230, 230, 230}



% Beamer Template Options
\beamertemplatenavigationsymbolsempty
\setbeamertemplate{footline}[frame number]
\setbeamercolor{structure}{fg=black}
\setbeamercolor{footline}{fg=black}
\setbeamercolor{title}{fg=black}
\setbeamercolor{frametitle}{fg=black}
\setbeamercolor{item}{fg=black}
\setbeamercolor{}{fg=black}
\setbeamercolor{bibliography item}{fg=black}
\setbeamercolor*{bibliography entry title}{fg=black}
\setbeamercolor{alerted text}{fg=focus}
\setbeamertemplate{items}[square]
\setbeamertemplate{enumerate items}[default]
\captionsetup[figure]{labelfont={color=black},font={color=black}}
\captionsetup[table]{labelfont={color=black},font={color=black}}

\setbeamertemplate{bibliography item}{\insertbiblabel}

%tcolor boxes
\newtcolorbox{samplecode}[2][]{
  colback=mint, colframe=darkmint, coltitle=white,
  fontupper = \ttfamily\scriptsize, fonttitle= \bfseries\scriptsize,
  boxrule = 0mm, arc = 0mm,
  boxsep = 1.3mm, left = 0mm, right = 0mm, top = 0.5mm, bottom = 0mm, middle=0mm,
  #1,title=#2}
  
\newtcolorbox{keytakeaway}[2][]{
  colback=softred, colframe=red, coltitle=white,
  fontupper = \scriptsize, fonttitle= \bfseries\scriptsize,
  boxrule = 0mm, arc = 0mm,
  boxsep = 1.3mm, left = 0mm, right = 0mm, top = 0.5mm, bottom = 0mm, middle=0mm,
  #1,title=#2}

\newtcolorbox{exercise}[2][]{
  colback=brightanthracite, colframe=anthracite, coltitle=white,
  fontupper = \scriptsize, fonttitle= \bfseries\scriptsize,
  boxrule = 0mm, arc = 0mm,
  boxsep = 1.3mm, left = 0mm, right = 0mm, top = 0.5mm, bottom = 0mm, middle=0mm,
  #1,title=#2}



% Link Icon Command 
\newcommand{\link}{%
    \tikz[x=1.2ex, y=1.2ex, baseline=-0.05ex]{%
        \begin{scope}[x=1ex, y=1ex]
            \clip (-0.1,-0.1)
                --++ (-0, 1.2)
                --++ (0.6, 0)
                --++ (0, -0.6)
                --++ (0.6, 0)
                --++ (0, -1);
            \path[draw,
                line width = 0.5,
                rounded corners=0.5]
                (0,0) rectangle (1,1);
        \end{scope}
        \path[draw, line width = 0.5] (0.5, 0.5)
            -- (1, 1);
        \path[draw, line width = 0.5] (0.6, 1)
            -- (1, 1) -- (1, 0.6);
        }
    }

% Other commands
\newcommand\tab[1][0.5cm]{\hspace*{#1}} % for code boxes


% Read Git Data from Github Actions Workflow
% Defaults to gitinfo2 for local builds
\IfFileExists{gitInfo.txt}
	{\input{gitInfo.txt}}
	{
		\newcommand{\gitRelease}{(Local Release)}
		\newcommand{\gitSHA}{\gitHash}
		\newcommand{\gitDate}{\gitAuthorIsoDate}
	}

% Custom Titlepage
\defbeamertemplate*{title page}{customized}[1][]
{
  \vspace{-0cm}\hfill\includegraphics[width=2.5cm]{../config/logo_cif}
  \includegraphics[width=1.9cm]{../config/seal_wwz}
  \\ \vspace{2em}
  \usebeamerfont{title}\textbf{\inserttitle}\par
  \usebeamerfont{title}\usebeamercolor[fg]{title}\insertsubtitle\par  \vspace{1.5em}
  \small\usebeamerfont{author}\insertauthor\par
  \usebeamerfont{author}\insertinstitute\par \vspace{2em}
  \usebeamercolor[fg]{titlegraphic}\inserttitlegraphic
    \tiny \noindent \texttt{Release Ver.: \gitRelease}\\ 
    \texttt{Version Hash: \gitSHA}\\
    \texttt{Version Date: \gitDate}\\ \vspace{1em}
    
    
    \iffalse
  \link \href{https://github.com/cifunibas/Bitcoin-Blockchain-Cryptoassets/blob/main/slides/intro.pdf}
  {Get most recent version}\\
  \link \href{https://github.com/cifunibas/Bitcoin-Blockchain-Cryptoassets/blob/main/slides/intro.pdf}
  {Watch video lecture}\\ 
  
  \fi
  
  \vspace{1em}
  License: \texttt{Creative Commons Attribution-NonCommercial-ShareAlike 4.0 International}\\\vspace{2em}
  \includegraphics[width = 1.2cm]{../config/license}
}


% tikzlibraries
\usetikzlibrary{decorations.pathreplacing}
\usetikzlibrary{decorations.markings}
\usetikzlibrary{positioning}
\usetikzlibrary{calc}
\captionsetup{font=footnotesize}

%%%%%%%%%%%%%%%%%%%%%%%%%%%%%%%%%%%%%%%%%%%%%%
%%%%%%%%%%%%%%%%%%%%%%%%%%%%%%%%%%%%%%%%%%%%%%
\begin{document}

\thispagestyle{empty}
\begin{frame}[noframenumbering]
	\titlepage
\end{frame}

%%%
\begin{frame}{Time on the Blockchain}

\begin{minipage}{0.3\textwidth}
	\begin{figure}
		\center
		\includegraphics[width= 2.2cm]{../assets/images/time.png}	
	\end{figure}
\end{minipage}
\begin{minipage}{0.65\textwidth}
	\textbf{Goal:} Contract users must be able to set an end time for their auction.\\
	
	\color{darkmint}\textbf{How does a contract keep track of time?}
\end{minipage}

\vspace{2em}

\uncover<2->{

Solidity offers \globvar{global variables}.

\vspace{1em}

\textbf{Example:} \globvar{block.timestamp} gives us the time that the miner sets in the block where our transaction is included.\\

\vspace{1em}

\textbf{Security consideration:} This information can be influenced by the miner.
}
	
\end{frame}
%%%


%%%
\begin{frame}{Other Block Variables }

\begin{itemize}
	\item \globvar{block.basefee}  returns current block’s base fee.
	\item \globvar{block.chainid}  returns current chain id.
	\item \globvar{block.coinbase}  returns the miner address of the current block.
	\item \globvar{block.difficulty}  returns current block difficulty.
	\item \globvar{block.gaslimit}  returns current block gaslimit.
	\item \globvar{block.number} returns the current block number.
\end{itemize}

\vspace{1.5em}

\link \href{https://docs.soliditylang.org/en/v0.8.9/units-and-global-variables.html}{Solidity Documentation: Block and Transaction Properties}.

	
\end{frame}
%%%


%%%
\begin{frame}{Other Global Variables}

\textbf{Frequently used}
\begin{itemize}
	\item \globvar{msg.sender} sender address of the message (current call).\\
		\textit{This is not necessarily the origin of the transaction, it can be the address of another contract that made an internal call.}
	\item \globvar{msg.value} ether amount in wei sent with the message.
\end{itemize}

\vspace{0.5em}

\uncover<2->{

\textbf{Others}
\begin{itemize}
	\item \globvar{msg.data} complete calldata.
	\item \globvar{msg.sig} first four bytes of the calldata indicating the function that is called.
	\item \globvar{tx.origin} address of the \textbf{original} sender of the transaction.
	\item \globvar{tx.gasprice} gas price of the transaction.
	\item \globvar{blockhash(blockNumber)} hash of the indicated block.
\end{itemize}

\vspace{0.5em}

\link \href{https://docs.soliditylang.org/en/v0.8.9/cheatsheet.html}{Solidity Documentation: Cheatsheet | Global Variables}.

}
	
\end{frame}
%%%


%%%
\begin{frame}{Units}

Solidity supports \textbf{time} and \textbf{monetary} units natively.

\vspace{0.5 em}

\uncover<2->{

\begin{samplecode}{Ether Units}
	\begin{lstlisting}[language=Solidity]
uint amount = 1 ether;
assert(amount == 1e18); // Will Pass

\end{lstlisting}
\end{samplecode}

Most common are \typesunits{wei} $(1)$,  \typesunits{gwei} $(10^9)$, and  \typesunits{ether} $(10^{18})$.
}

\vspace{1.5em}

\uncover<3->{

Time suffixes can be used to convert to seconds. Supported are \typesunits{seconds}, \typesunits{minutes}, \typesunits{hours}, \typesunits{days}, \typesunits{weeks}.
\begin{samplecode}{Time Units}
	\begin{lstlisting}[language=Solidity]
uint twoHours = 2 hours;
assert(twoHours == 2 * 60 * 60); // Will Pass

\end{lstlisting}
\end{samplecode} 
%
%$\Rightarrow$ months or years are \textbf{not} supported.
}

\end{frame}
%%%

%%%
\begin{frame}{Suffixes and Variables}

\textbf{Keep in mind:} Usage of variables with suffixes is disallowed.

\vspace{1.5em}
\begin{samplecode}{Hours to Seconds}
	\begin{lstlisting}[language=Solidity]
uint numberOfHours = 2;
uint twoHours = numberOfHours hours; // Not allowed
uint twoHours = numberOfHours * 1 hours; // Use this instead

\end{lstlisting}
\end{samplecode}

\end{frame}
%%%


%%%
\begin{frame}{Add Duration to the Auction}

\begin{samplecode}{Auction Parameters}
	\begin{lstlisting}[language=Solidity]
// Auction parameters
address public beneficiary;
uint public endTime; // As UNIX timestamp

constructor (address _beneficiary, uint durationMinutes) {
	beneficiary = _beneficiary;
	endTime = block.timestamp + durationMinutes * 1 minutes;
}
\end{lstlisting}
\end{samplecode} 
	
\end{frame}
%%%


%%%
\begin{frame}{Function Restrictions and Requirements}

\vspace{1 em}
Often, functions in a contract need to be \textbf{restricted} in some way.

\vspace{0.5em}

\textbf{Goal:} \texttt{bid()} should only be callable during an active auction.

\uncover<2->{
\vspace{1.5 em}

Solidity offers \genkey{require}\texttt{(<condition>, <error message>);}:
\begin{itemize}
	\item if \texttt{<condition>} evaluates to \texttt{false}, the \textbf{whole} transaction is \textbf{reverted.}
	\item  \texttt{<error message>} is optional but improves usability.
\end{itemize}

\begin{samplecode}{Restricted bid() Function}
	\begin{lstlisting}[language=Solidity]
function bid() public {
  require(block.timestamp < endTime, 'Auction Ended');
}

\end{lstlisting}
\end{samplecode}
}
	
\end{frame}
%%

%
%%%
\begin{frame}{Building the bid() function}

\vspace{1em}

\textbf{Accept and store valid bids:}
\begin{itemize}
	\item Add \genkey{payable} modifier to be able to receive ether.\\
	\textit{$\Rightarrow$ All ether received are in full control of the contract.}
	\item Check if bid (\globvar{msg.value}) is higher than previous valid bid.
	\item Store the new highest bid and bidder (\globvar{msg.sender}).
\end{itemize}

\vspace{1em}
\begin{samplecode}{bid() Function}
	\begin{lstlisting}[language=Solidity]
function bid() public payable {
  require(block.timestamp < endTime, 'Auction Ended');
  require(msg.value > highestBid, 'Bid too small');
  highestBid = msg.value;
  highestBidder = msg.sender;
}

\end{lstlisting}
\end{samplecode}


\end{frame}
%%%


%%%
\begin{frame}{Refunding the Outbidden}

\textbf{Goal:} Whenever a new valid bid is received, the previous bidder is refunded.\\

\vspace{1em}

\textbf{Remember:} A smart contract is unable to initiate transactions by itself.

\vspace{2em}
\uncover<2->{
\textbf{Options:}

\begin{minipage}{0.18\textwidth}
	\begin{figure}
		\center
		\includegraphics[width= 2.2cm]{../assets/images/refund.png}	
	\end{figure}
\end{minipage}
\begin{minipage}{0.8\textwidth}
	\begin{enumerate}
		\item Refund as part of the new bid transaction.\\
			\uncover<3->{$\rightarrow$ \textit{Passing execution to another address introduces significant security risk.}}
			\vspace{0.5em}
		\item Allow the user to withdraw his invalidated bids.\\
		\uncover<3->{\color{darkmint}\textbf{$\rightarrow$ Preferred option}}
	\end{enumerate}
\end{minipage}


}

\end{frame}
%%%


%%%
\begin{frame}{Mappings}

\textbf{Goal:} Keep track of how much ether a previous bidder can withdraw.

\vspace{1em}

\uncover<2->{
\emph{Mappings} store \texttt{key => value} pairs, similar to a hash map or dictionary in other programming languages:
\begin{itemize}
	\item The key's keccak256 hash points to the storage slot for the value.
	\item Mappings span the full storage space of $2^{256}$ slots.
	\item Declaration: \texttt{mapping(keyType => valueType) <visibility> <variable name>;}
\end{itemize}
}

\vspace{1em}

\uncover<3->{
\textbf{Visibility:} If a mapping is declared as \genkey{public}, it automatically has a getter function in form of \texttt{<variable name>(<keyType> key) returns (<valueType>)}.
}
\end{frame}
%%%


%%%
\begin{frame}{Store withdrawable funds per address}

\begin{samplecode}{pendingReturns Mapping}
	\begin{lstlisting}[language=Solidity]
contract SimpleAuction {
  // Allowed withdrawals of previous bids
  mapping(address => uint) pendingReturns;
  
  function bid() public payable {
    require(block.timestamp < endTime, 'Auction ended');
    require(msg.value > highestBid, 'Bid too low');
    if (highestBid != 0) {
      pendingReturns[highestBidder] += highestBid;
    }
    highestBid = msg.value;
    highestBidder = msg.sender;
  }
}

\end{lstlisting}
\end{samplecode}

\textbf{$\Rightarrow$ Using \texttt{+=} prevents overwriting previous unclaimed refunds.}

\end{frame}
%%%


%%%
\begin{frame}{More on Mappings}

Mappings are very powerful data structures with some drawbacks:
\begin{itemize}
	\item	Lack of length property.
	\item	Can not be enumerated or returned.
	\item	Can not be cleared easily.
\end{itemize}

\uncover<2->{
\vspace{1em}
\color{darkmint} $\Rightarrow$ \textbf{No straightforward way to reconstruct storage state}.\color{black}

\vspace{0.5em}
Separately storing all keys used can be a workaround. 
}

\uncover<3->{
\vspace{1.5em}
Mappings can be \textbf{nested}.\\
\texttt{mapping(address => mapping(uint => bool))} will store a \texttt{uint => bool} mapping for each address.
}

\end{frame}
%%%


%%%
\begin{frame}{Sending Ether from a Contract to an Address}

\textbf{Goal:} Return ether to outbidden user as part of \texttt{withdraw()}.

\uncover<2->{
\vspace{1.5em}

For security reasons, Solidity differentiates between \genkey{payable} and non-payable addresses.

\vspace{1.5em}
To be able to send ether to an address, it needs to be explicitly declared as \genkey{payable:}\\ \vspace{1em}

\texttt{address payable payableAddress = payable(normalAddress)};
 }

\end{frame}
%%%

%%%
\begin{frame}{The Three Options of a Contract to Transfer Ether}

\begin{enumerate}
	\item 	\texttt{address payable.transfer(<amount>)}\\
	Forwards 2300 gas and tries to transfer \texttt{<amount>} ether to the target address. Transfer failure reverts the whole transaction.
	\vspace{0.5em}
	\item<2->	\texttt{address payable.send(<amount>)}\\
	Same as \texttt{transfer} but does not revert on failure. Returns \texttt{true} or \texttt{false} depending success.
	\vspace{0.5em}
	\item<3->	\texttt{address payable.call $\{$value: <amount> $\}$("")}\\
	Low level call to the address. Forwards all remaining gas or an exact value, if so specified via \texttt{$\{$value: <amount>, gas: <gas amount>.$\}$}
\end{enumerate}

\vspace{1.5em}
\uncover<4->{
For simplicity and security reasons, we are using \texttt{transfer}. But there are good reasons to use call instead. Especially to ensure forward-compatibility regarding gas costs.
}

\end{frame}
%%%

%%%
\begin{frame}{Building the withdraw() function}

\begin{samplecode}{withdraw() Function}
	\begin{lstlisting}[language=Solidity]
function withdraw() external returns (uint amount) {
  amount = pendingReturns[msg.sender];
  if (amount > 0) {
    pendingReturns[msg.sender] = 0;
	payable(msg.sender).transfer(amount);
  }
  // optional: return amount;
}

\end{lstlisting}
\end{samplecode}

\begin{itemize}
	\item 	Add a returns statement to the function to show the amount of ether withdrawn.
	\item<2->  Specify the name of the return value to avoid declaring the variable and an explicit return statement. 
	\item<3->	To prevent re-entrancy attacks, update the balance \color{focus}before \color{black} transfer.
\end{itemize}

\end{frame}
%%%

%%%
\begin{frame}{Ending the Auction}

\textbf{Goal:} After endTime, the highest bid is to be transferred to the beneficiary.

\vspace{1.5em}

\textbf{Remember:} Smart contracts can not do anything by themselves.

\uncover<2->{
\vspace{0.5em}
$\Rightarrow$ \texttt{auctionEnd()} to close the auction and transfer the bid is needed.


\vspace{1.5em}
\textbf{Structuring guidelines} for functions interacting with other addresses:
\begin{enumerate}
	\item Check all conditions.
	\item Apply all internal state changes.
	\item Interact with other addresses.
\end{enumerate}
}

\end{frame}
%%%

%%%
\begin{frame}{The auctionEnd() Function}


\begin{samplecode}{auctionEnd() Function}
	\begin{lstlisting}[language=Solidity]
function auctionEnd() external {
  // 1. Check all conditions
  require(!hasEnded, 'Auction already ended');
  require(block.timestamp >= endTime, 'Wait for auction to end');
  
  // 2. Apply all internal state changes
  hasEnded = true;
  
  // 3. Interact with other addresses
  payable(beneficiary).transfer(highestBid);
}
\end{lstlisting}
\end{samplecode}

\end{frame}
%%%

%%%
\begin{frame}{Events}

\begin{minipage}{0.65\textwidth}
\textbf{The Blockchain is fully transparent. Does this mean that one can obtain the full history of bids?}

\uncover<2->{
\vspace{1 em}

\textit{Yes, using a lot of resources: Access an archive node and call the state at every block to check the bid variable. }
}
\end{minipage}
\begin{minipage}{0.3\textwidth}
	\begin{figure}
		\center
		\includegraphics[width= 2.2cm]{../assets/images/event.png}	
	\end{figure}
\end{minipage}

\uncover<3->{
\vspace{1.5em}
\textbf{Should the bid of each user be stored permanently then?}
}

\vspace{1 em}

\uncover<4->{
\textit{No, this would be very expensive and causes state bloat.}
}

\uncover<5->{
\vspace{1.5em}

\textbf{Solution:} Write data into logs.
\begin{itemize}
	\item	\textbf{Cheap:} It is stored in a different place than the state.
	\item	\textbf{Verifiable:} It is part of the block hash.
	\item	\textbf{Downside:} It cannot be accessed as part of a transaction.
\end{itemize}
}

\end{frame}
%%%

%%%
\begin{frame}{Declaring Events}

\textbf{Declaration:}
\begin{itemize}
	\item \texttt{event <EventName> (<parameterType> <indexed> <parameterName>, ...)}
	\item	Once emitted, the corresponding log can be queried from a full node.  
\end{itemize}

\uncover<2->{
\vspace{0.5 em}

\textbf{Indexing:}
\begin{itemize}
	\item Indexed parameters can be used to filter logs.
	\item Slightly more expensive to emit.
	\item Maximum of three indexed parameters per log.
\end{itemize}
}

\uncover<3->{
\vspace{0.5 em}

\textbf{Emission:} \texttt{emit <EventName> (<value>,... );}
}

\uncover<4->{
\vspace{1 em}

$\Rightarrow$ Most node clients offer functionality to subscribe to events and get notified whenever a specific contract emits a specified event.
}

\end{frame}
%%%

%%%
\begin{frame}{Building Events for the Auction}

\begin{samplecode}{NewBid and AuctionEnded Events}
	\begin{lstlisting}[language=Solidity]
// Events
event NewBid(address indexed bidder, uint amount);
event AuctionEnded(address winner, uint amount);

function bid() public payable { // ...
  highestBid = msg.value;
  highestBidder = msg.sender;
  emit NewBid(msg.sender, msg.value);
}
function auctionEnd() external { 
  // 1. Check all conditions...
  // 2. Apply all internal state changes
  hasEnded = true;
  emit AuctionEnded(highestBidder, highestBid);
  // 3. Interact with other addresses
  payable(beneficiary).transfer(highestBid);
}

\end{lstlisting}
\end{samplecode}

\end{frame}
%%%

%%%
\begin{frame}{Update of the Auction Contract}
	\begin{exercise}{Exercise 1:}
	\begin{enumerate}
		\item Add the code snippets shown in this slide deck to the SimpleAuction contract.
		\item Deploy the contract, and interact with it using different addresses.
	\end{enumerate}
	\end{exercise}
\end{frame}
%%%


\end{document}
