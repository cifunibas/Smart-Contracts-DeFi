% Choose one to switch between slides and handout
\documentclass[]{beamer}
%\documentclass[handout]{beamer}

% Video Meta Data
\title{Smart Contracts and Decentralized Finance Applications}
\subtitle{Intro: Welcome to the Course}
\author{Prof. Dr. Fabian Schär}
\institute{University of Basel}

% Config File
% Packages
\usepackage[utf8]{inputenc}
\usepackage{hyperref}
\usepackage{gitinfo2}
\usepackage{tikz}
 \usetikzlibrary{calc}
\usepackage{amsmath}
\usepackage{mathtools}
\usepackage{bibentry}
\usepackage{xcolor}
\usepackage{colortbl} % Add colour to LaTeX tables
\usepackage{caption}
\usepackage[export]{adjustbox}
\usepackage{pgfplots} \pgfplotsset{compat = 1.17}
\usepackage{makecell}
\usepackage{fancybox}
\usepackage{ragged2e}
\usepackage{fontawesome}
\usepackage{seqsplit}
\usepackage{tabularx}
\usepackage{tcolorbox}
\usepackage{booktabs} % use instead  \hline in tables

% Color Options
\definecolor{highlight}{rgb}{0.65,0.84,0.82}
\definecolor{focus}{rgb}{0.72, 0, 0}
\definecolor{lightred}{rgb}{0.8,0.5,0.5}
\definecolor{midgray}{RGB}{190,195,200}

 %UniBas Main Colors
\definecolor{mint}{RGB}{165,215,210}
\definecolor{anthracite}{RGB}{45,55,60}
\definecolor{red}{RGB}{210,5,55}

 %UniBas Color Palette (for graphics)
\definecolor{strongmint}{RGB}{30,165,165}
\definecolor{darkmint}{RGB}{0,110,110}
\definecolor{softanthracite}{RGB}{140,145,150}
\definecolor{brightanthracite}{RGB}{190,195,200}
\definecolor{softred}{RGB}{235,130,155}

%Custom Colors
\definecolor{lightergray}{RGB}{230, 230, 230}



% Beamer Template Options
\beamertemplatenavigationsymbolsempty
\setbeamertemplate{footline}[frame number]
\setbeamercolor{structure}{fg=black}
\setbeamercolor{footline}{fg=black}
\setbeamercolor{title}{fg=black}
\setbeamercolor{frametitle}{fg=black}
\setbeamercolor{item}{fg=black}
\setbeamercolor{}{fg=black}
\setbeamercolor{bibliography item}{fg=black}
\setbeamercolor*{bibliography entry title}{fg=black}
\setbeamercolor{alerted text}{fg=focus}
\setbeamertemplate{items}[square]
\setbeamertemplate{enumerate items}[default]
\captionsetup[figure]{labelfont={color=black},font={color=black}}
\captionsetup[table]{labelfont={color=black},font={color=black}}

\setbeamertemplate{bibliography item}{\insertbiblabel}

%tcolor boxes
\newtcolorbox{samplecode}[2][]{
  colback=mint, colframe=darkmint, coltitle=white,
  fontupper = \ttfamily\scriptsize, fonttitle= \bfseries\scriptsize,
  boxrule = 0mm, arc = 0mm,
  boxsep = 1.3mm, left = 0mm, right = 0mm, top = 0.5mm, bottom = 0mm, middle=0mm,
  #1,title=#2}
  
\newtcolorbox{keytakeaway}[2][]{
  colback=softred, colframe=red, coltitle=white,
  fontupper = \scriptsize, fonttitle= \bfseries\scriptsize,
  boxrule = 0mm, arc = 0mm,
  boxsep = 1.3mm, left = 0mm, right = 0mm, top = 0.5mm, bottom = 0mm, middle=0mm,
  #1,title=#2}

\newtcolorbox{exercise}[2][]{
  colback=brightanthracite, colframe=anthracite, coltitle=white,
  fontupper = \scriptsize, fonttitle= \bfseries\scriptsize,
  boxrule = 0mm, arc = 0mm,
  boxsep = 1.3mm, left = 0mm, right = 0mm, top = 0.5mm, bottom = 0mm, middle=0mm,
  #1,title=#2}



% Link Icon Command 
\newcommand{\link}{%
    \tikz[x=1.2ex, y=1.2ex, baseline=-0.05ex]{%
        \begin{scope}[x=1ex, y=1ex]
            \clip (-0.1,-0.1)
                --++ (-0, 1.2)
                --++ (0.6, 0)
                --++ (0, -0.6)
                --++ (0.6, 0)
                --++ (0, -1);
            \path[draw,
                line width = 0.5,
                rounded corners=0.5]
                (0,0) rectangle (1,1);
        \end{scope}
        \path[draw, line width = 0.5] (0.5, 0.5)
            -- (1, 1);
        \path[draw, line width = 0.5] (0.6, 1)
            -- (1, 1) -- (1, 0.6);
        }
    }

% Other commands
\newcommand\tab[1][0.5cm]{\hspace*{#1}} % for code boxes


% Read Git Data from Github Actions Workflow
% Defaults to gitinfo2 for local builds
\IfFileExists{gitInfo.txt}
	{\input{gitInfo.txt}}
	{
		\newcommand{\gitRelease}{(Local Release)}
		\newcommand{\gitSHA}{\gitHash}
		\newcommand{\gitDate}{\gitAuthorIsoDate}
	}

% Custom Titlepage
\defbeamertemplate*{title page}{customized}[1][]
{
  \vspace{-0cm}\hfill\includegraphics[width=2.5cm]{../config/logo_cif}
  \includegraphics[width=1.9cm]{../config/seal_wwz}
  \\ \vspace{2em}
  \usebeamerfont{title}\textbf{\inserttitle}\par
  \usebeamerfont{title}\usebeamercolor[fg]{title}\insertsubtitle\par  \vspace{1.5em}
  \small\usebeamerfont{author}\insertauthor\par
  \usebeamerfont{author}\insertinstitute\par \vspace{2em}
  \usebeamercolor[fg]{titlegraphic}\inserttitlegraphic
    \tiny \noindent \texttt{Release Ver.: \gitRelease}\\ 
    \texttt{Version Hash: \gitSHA}\\
    \texttt{Version Date: \gitDate}\\ \vspace{1em}
    
    
    \iffalse
  \link \href{https://github.com/cifunibas/Bitcoin-Blockchain-Cryptoassets/blob/main/slides/intro.pdf}
  {Get most recent version}\\
  \link \href{https://github.com/cifunibas/Bitcoin-Blockchain-Cryptoassets/blob/main/slides/intro.pdf}
  {Watch video lecture}\\ 
  
  \fi
  
  \vspace{1em}
  License: \texttt{Creative Commons Attribution-NonCommercial-ShareAlike 4.0 International}\\\vspace{2em}
  \includegraphics[width = 1.2cm]{../config/license}
}


% tikzlibraries
\usetikzlibrary{decorations.pathreplacing}
\usetikzlibrary{decorations.markings}
\usetikzlibrary{positioning}
\usetikzlibrary{calc}
\captionsetup{font=footnotesize}

%%%%%%%%%%%%%%%%%%%%%%%%%%%%%%%%%%%%%%%%%%%%%%
%%%%%%%%%%%%%%%%%%%%%%%%%%%%%%%%%%%%%%%%%%%%%%
\begin{document}

\thispagestyle{empty}
\begin{frame}[noframenumbering]
	\titlepage
\end{frame}

%%%
\begin{frame}{Content Overview}

\begin{enumerate}
	\item<1-> \textbf{Blockchain Fundamentals}\\
		Recap of basic blockchain building blocks.
		\vspace{0.2em}
	\item<2-> \textbf{Ethereum Basics}\\
		The specifics of the Ethereum as Smart Contract platform.
		\vspace{0.2em}
	\item<3-> \textbf{Smart Contract Programming}\\
		Hands-on introduction to the basics of Smart Contract programming with emphasis on exercises.
		\vspace{0.2em}
	\item<4-> \textbf{Tokenization}\\
		Motivation behind Tokenization and overview of most prominent token standards.
		\vspace{0.2em}
	\item<5-> \textbf{Decentralized Finance Applications}\\
			Overview of the DeFi ecosystem, key components and major protocol types.
\end{enumerate}

\end{frame}
%%%	

%%%
\begin{frame}{Why Smart Contracts?}

Vending Machines can be seen as predecessors of Smart Contracts \cite{NS:94,NS:97}. Agreements are automated and breach of contract is costly

\vspace{0.5em}

\begin{minipage}{0.4\textwidth}
	\vspace{0.5em}
	\centering
	\includegraphics[width=2.5cm]{../assets/images/vendingmachine.png}
\end{minipage}
\begin{minipage}{0.55\textwidth}
	\begin{exampleblock}{Simple Vending Machine (Pseudo Code)}
	 	\texttt{
			if(coin $>=$ price) \{ \\
				\tab dispenseBeverage(); \\
				\tab returnChange(coin - price);\\
			\} else \{ \\
				\tab print("insufficient funds"); \\
			\}
		}
	\end{exampleblock}	
\end{minipage}

\vspace{1.0em}
\textbf{But:}\\
\begin{itemize}
	\item Trust based! Closed source $\rightarrow$ Contract is not observable.
	\item Execution environment, i.e., hardware under control of seller.
\end{itemize}

\vspace{1.0em}
\textbf{$\Rightarrow$ Smart Contracts on public Blockchains address this.}


\end{frame}
%%%

%%%
\begin{frame}{Why Decentralized Finance (DeFi) Applications?}

\begin{figure}[t]
	\centering	
	\resizebox{0.8\textwidth}{!}{
	\begin{tikzpicture}[scale=1.0, every node/.style={scale=1.0}]
			\input{../assets/figures/defi_stack.tex}
	\end{tikzpicture}}
	\caption{The DeFi Stack \cite{FS:21}}
\end{figure}

\vspace{-1.0em}

\begin{itemize}
	\item Most mature and diverse smart contract-based ecosystem.
	\item Relevancy of applications from an economics perspective.
\end{itemize}

\end{frame}
%%%

%%%
\begin{frame}{Interdisciplinaray Approach}

	\uncover<1->{
		\begin{figure}[h]
  			\center
			\input{../assets/figures/interdiciplinarity.tex}
		\end{figure}
	}
	\vspace{1em}
Public Blockchains can only be fully understood, when they are studied from various perspectives. This is the reason why this course uses an \color{focus} \textbf{interdisciplinary} \color{black} approach.	


\end{frame}
%%%

%%%
\begin{frame}{Programming and Computer Science Exposure}

\textbf{Programming Languages}
\vspace{1em}

\begin{minipage}{0.32\textwidth}
	\begin{figure}[t]
		\begin{tikzpicture}[scale=1.0, every node/.style={scale=1.0}]
				

% Logos
\node at (0,1) {\includegraphics[height = 1.2cm]{../assets/images/logo_solidity.png}};



% Labels
\draw (0,0) node  {solidity};

		\end{tikzpicture}
	\end{figure}
\end{minipage}
\vspace{2em}

\textbf{Tools}
\vspace{1em}

\begin{minipage}{0.95\textwidth}
	\begin{figure}[t]	
		\begin{tikzpicture}[scale=1.0, every node/.style={scale=1.0}]
				

% Logos
\node at (0,2) {\includegraphics[height = 1.2 cm]{../assets/images/logo_atom.png}};
\node at (1.8, 2) {\includegraphics[height = 1.2 cm]{../assets/images/logo_vscode.png}};
\node at (3.6,2) {\includegraphics[height = 1.2 cm]{../assets/images/logo_remix.png}};
\node at (5.4,2) {\includegraphics[height = 1.2 cm]{../assets/images/logo_github.png}};
\node at (7.2,2) {\includegraphics[height = 1.2 cm]{../assets/images/logo_ganache.png}};
\node at (9,2) {\includegraphics[height = 1.2 cm]{../assets/images/logo_truffle.png}};
\node at (10.8,2) {\includegraphics[height = 1.2 cm]{../assets/images/logo_metamask.png}};


% Labels
\footnotesize{
	\draw (0,1.1) node [minimum height = 1cm] {Atom};
	\draw (1.8, 1.1) node [minimum height = 1cm] {VS Code};
	\draw (3.6,1.1) node [minimum height = 1cm] {Remix};
	\draw (5.4,1.1) node [minimum height = 1cm] {Github};
	\draw (7.2,1.1) node [minimum height = 1cm] {Ganache};
	\draw (9,1.1) node [minimum height = 1cm] {Truffle};
	\draw (10.8,1.1) node [minimum height = 1cm] {Metamask};			
}

% Groups

\draw [decorate,decoration={brace,amplitude=5pt,mirror},xshift=0 pt,yshift=0pt] (-0.8,0.8) -- (4.4,0.8) node [midway, below, align = center, yshift = -0.25cm, text width = 3.6 cm] {Integrated Development Environments};

\draw [decorate,decoration={brace,amplitude=5pt,mirror},xshift=0 pt,yshift=0pt] (4.6,0.8) -- (6.2,0.8) node [midway, below, align = center, yshift = -0.25cm, text width = 1.8 cm] {Version Control};

\draw [decorate,decoration={brace,amplitude=5pt,mirror},xshift=0 pt,yshift=0pt] (6.4,0.8) -- (9.8,0.8) node [midway, below, align = center, yshift = -0.25cm, text width = 2 cm] {Blockchain Management};

\draw [decorate,decoration={brace,amplitude=5pt,mirror},xshift=0 pt,yshift=0pt] (10,0.8) -- (11.6,0.8) node [midway, below, align = center, yshift = -0.25cm, text width = 3.6 cm] {Testing};


		\end{tikzpicture}
	\end{figure}
\end{minipage}



\end{frame}
%%%

%%%
\begin{frame}{Some Lecture Conventions}

	\begin{columns}[T]
		\begin{column}{0.15\textwidth}
			
		\end{column} %\hfill
		\begin{column}{0.5\textwidth}
			\textbf{Sample Code Box}\
			Contains code snippets in R, Solidity \& pseudo code.
		\end{column}
	\end{columns}

\vspace{2.0em}

	\begin{columns}[T]
		\begin{column}{0.45\textwidth}
			\vspace{-1.0em}
			\begin{alertblock}{Transaction Definitions}
				A transaction is a message send by an externally owned account.
			\end{alertblock}		
		\end{column}
		\begin{column}{0.5\textwidth}
			\textbf{Key Take-Away}\\
			Highlights important concepts and definitions.
		\end{column}
	\end{columns}

\vspace{2.0em}

	\begin{columns}[T]
		\begin{column}{0.45\textwidth}
			\vspace{-1.0em}
			\begin{block}{Exercise 1}
				Get Ropsten Ether from the faucet and deploy your first smart contract on Ropsten Testnet.
			\end{block}		
		\end{column}
		\begin{column}{0.5\textwidth}
		\textbf{Exercises}\\
			Things for you to try out.
		\end{column}
	\end{columns}

\end{frame}
%%%

%%%
\begin{frame}{Part of Multi-Course Series}

Blockchain courses have been part of the University of Basel's curriculum since 2017.

\vspace{1.5em}

\begin{columns}
	\begin{column}{0.35 \textwidth}
		\uncover<1->{
			\includegraphics[width = 4cm]{../config/logo_cif}
		}
	\end{column}
	\begin{column}{0.6 \textwidth}	
			\begin{itemize}
			\item<2-> This is a University graduate-/ master-level course
			\item<3-> It is part of a series of courses
			\item<4-> Second course to switch to open lecture format
		\end{itemize}
	\end{column}	
\end{columns}

%\vspace{2em}
%
%\uncover<5->{
%	$\rightarrow$ There will be more open lecture courses.
%}

\end{frame}
%%%

%%%
\begin{frame}{Three Options to Take This Course}

The goal of our open lectures is to make teaching resources freely available. There are \color{focus} \textbf{three options} \color{black} for taking this course:\vspace{1em}

\begin{table}\footnotesize
	\begin{tabular}{lcccc}
	\hline \hline
									& Videos 		& Platform 		& Assignments 	& ECTS 	\\ \cline{2-5}
		YouTube 	 				& $\checkmark$	& 				& 				& 		\\
		Cryptolectures.io 			& $\checkmark$	& $\checkmark$	& $\checkmark$	&		\\
		University of Basel			& $\checkmark$	& $\checkmark$	& $\checkmark$	& $\checkmark$	\\
		\hline \hline
	\end{tabular}
\end{table} \vspace{2em}

\uncover<2->{
\link \href{https://www.youtube.com/channel/UCOA52m4BOqtI8cHIx4zJAWg}
  {YouTube Channel}  \\
\link \href{https://www.cryptolectures.io}{Cryptolectures.io} \\
\link \href{https://www.unibas.ch/en/Studies/Application-Admission.html}{University of Basel - General Information}
}

\end{frame}
%%%

%%%
\begin{frame}{Meet the Open Crypto Lectures Team}
	\begin{columns}[T]
		\begin{column}{0.31\textwidth}
			\center \textbf{Professor}
			\begin{table}\small
				\begin{tabular}{c}
					Fabian Schär\\
					\href{https://linkedin.com/in/fabian-schaer/}{\faLinkedinSquare}\ \href{https://twitter.com/chainomics}{\faTwitterSquare}\\
				\end{tabular}
			\end{table}
		\end{column}
		\begin{column}{0.31\textwidth}
			\center \textbf{PhD Candidates}
			\begin{table}\small
				\begin{tabular}{c}
					Tobias Bitterli\\
					\href{https://linkedin.com/in/tobiasbitterli/}{\faLinkedinSquare}\ \href{https://twitter.com/tobias_bitterli}{\faTwitterSquare}\\
					\vspace{0.5em}\\
					Mitchell Goldberg\\
					\href{https://linkedin.com/in/mitchell-goldberg/}{\faLinkedinSquare}\ \href{https://twitter.com/golmit_crypto}{\faTwitterSquare}\\
					\vspace{0.5em}\\
					Matthias Nadler\\
					\href{https://linkedin.com/in/mat-nadler/}{\faLinkedinSquare}\ \href{https://twitter.com/mat_nadler}{\faTwitterSquare}\\
					\vspace{0.5em}\\
					Katrin Schuler\\
					\href{https://linkedin.com/in/kmschuler/}{\faLinkedinSquare}\ \href{https://twitter.com/Katatcrypt}{\faTwitterSquare}\\
				\end{tabular}
			\end{table}
		\end{column}
		\begin{column}{0.31\textwidth}
			\center \textbf{Student Assistants}
			\begin{table}\small
				\begin{tabular}{c}
					Lorenz Geering\\
					\href{https://linkedin.com/in/lorenz-geering-770359a8/}{\faLinkedinSquare}\ \href{https://twitter.com/lorenz_geering}{\faTwitterSquare}\\
					\vspace{0.5em}\\
					Pirmin Özdemir\\
					\href{https://www.linkedin.com/in/pirmin-\%C3\%B6zdemir-539845159/}{\faLinkedinSquare}\ \href{https://twitter.com/Pirmin15}{\faTwitterSquare}\\
					\vspace{0.5em}\\
					Jonas Ruchti\\
					\href{https://linkedin.com/in/jonas-ruchti-a29042221}{\faLinkedinSquare}\ \href{https://twitter.com/jonas_ruchti}{\faTwitterSquare}\\
					\vspace{0.5em}\\
					Dario Thürkauf\\
					\href{https://linkedin.com/in/dario-thuerkauf/}{\faLinkedinSquare} \href{https://twitter.com/dario_thuerkauf}{\faTwitterSquare}\\
				\end{tabular}
			\end{table}
		\end{column}
	\end{columns}
\end{frame}
%%%

%%%
\begin{frame}{Information for University of Basel Students}

	

	
\uncover<1->{
	\textbf{Group Project} (40\% of final grade)
	\begin{itemize}
		\item Smart Contract programming project.
		\item Groups of 2-4 students.
		\item Date: see course directory or welcome email.
	\end{itemize}
}

\vspace{1em}

\uncover<2->{
\textbf{Exam} (60\% of final grade)
	\begin{itemize}
		\item 90 Minutes
		\item Closed book
		\item T/F, MC, Numbers and Text/Figure Boxes ??
		\item You may use a non-programmable calculator (\link \href{https://wwz.unibas.ch/en/studies/examinations/use-of-materials-and-aids/}{Rules})
	\end{itemize}
	}

\end{frame}
%%%

%%%
\begin{frame}%[allowframebreaks]
\frametitle{References and Recommended Reading}
	\bibliographystyle{amsplain}
	\bibliography{../assets/bib/refs}
\end{frame}
%%%



\end{document}