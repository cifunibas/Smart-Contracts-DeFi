% Choose one to switch between slides and handout
%\documentclass[]{beamer}
\documentclass[handout]{beamer}

% Video Meta Data
\title{Smart Contracts and Decentralized Finance}
\subtitle{Intro: Welcome to the Course}
\author{Prof. Dr. Fabian Schär}
\institute{University of Basel}

% Config File
% Packages
\usepackage[utf8]{inputenc}
\usepackage{hyperref}
\usepackage{gitinfo2}
\usepackage{tikz}
 \usetikzlibrary{calc}
\usepackage{amsmath}
\usepackage{mathtools}
\usepackage{bibentry}
\usepackage{xcolor}
\usepackage{colortbl} % Add colour to LaTeX tables
\usepackage{caption}
\usepackage[export]{adjustbox}
\usepackage{pgfplots} \pgfplotsset{compat = 1.17}
\usepackage{makecell}
\usepackage{fancybox}
\usepackage{ragged2e}
\usepackage{fontawesome}
\usepackage{seqsplit}
\usepackage{tabularx}
\usepackage{tcolorbox}
\usepackage{booktabs} % use instead  \hline in tables

% Color Options
\definecolor{highlight}{rgb}{0.65,0.84,0.82}
\definecolor{focus}{rgb}{0.72, 0, 0}
\definecolor{lightred}{rgb}{0.8,0.5,0.5}
\definecolor{midgray}{RGB}{190,195,200}

 %UniBas Main Colors
\definecolor{mint}{RGB}{165,215,210}
\definecolor{anthracite}{RGB}{45,55,60}
\definecolor{red}{RGB}{210,5,55}

 %UniBas Color Palette (for graphics)
\definecolor{strongmint}{RGB}{30,165,165}
\definecolor{darkmint}{RGB}{0,110,110}
\definecolor{softanthracite}{RGB}{140,145,150}
\definecolor{brightanthracite}{RGB}{190,195,200}
\definecolor{softred}{RGB}{235,130,155}

%Custom Colors
\definecolor{lightergray}{RGB}{230, 230, 230}



% Beamer Template Options
\beamertemplatenavigationsymbolsempty
\setbeamertemplate{footline}[frame number]
\setbeamercolor{structure}{fg=black}
\setbeamercolor{footline}{fg=black}
\setbeamercolor{title}{fg=black}
\setbeamercolor{frametitle}{fg=black}
\setbeamercolor{item}{fg=black}
\setbeamercolor{}{fg=black}
\setbeamercolor{bibliography item}{fg=black}
\setbeamercolor*{bibliography entry title}{fg=black}
\setbeamercolor{alerted text}{fg=focus}
\setbeamertemplate{items}[square]
\setbeamertemplate{enumerate items}[default]
\captionsetup[figure]{labelfont={color=black},font={color=black}}
\captionsetup[table]{labelfont={color=black},font={color=black}}

\setbeamertemplate{bibliography item}{\insertbiblabel}

%tcolor boxes
\newtcolorbox{samplecode}[2][]{
  colback=mint, colframe=darkmint, coltitle=white,
  fontupper = \ttfamily\scriptsize, fonttitle= \bfseries\scriptsize,
  boxrule = 0mm, arc = 0mm,
  boxsep = 1.3mm, left = 0mm, right = 0mm, top = 0.5mm, bottom = 0mm, middle=0mm,
  #1,title=#2}
  
\newtcolorbox{keytakeaway}[2][]{
  colback=softred, colframe=red, coltitle=white,
  fontupper = \scriptsize, fonttitle= \bfseries\scriptsize,
  boxrule = 0mm, arc = 0mm,
  boxsep = 1.3mm, left = 0mm, right = 0mm, top = 0.5mm, bottom = 0mm, middle=0mm,
  #1,title=#2}

\newtcolorbox{exercise}[2][]{
  colback=brightanthracite, colframe=anthracite, coltitle=white,
  fontupper = \scriptsize, fonttitle= \bfseries\scriptsize,
  boxrule = 0mm, arc = 0mm,
  boxsep = 1.3mm, left = 0mm, right = 0mm, top = 0.5mm, bottom = 0mm, middle=0mm,
  #1,title=#2}



% Link Icon Command 
\newcommand{\link}{%
    \tikz[x=1.2ex, y=1.2ex, baseline=-0.05ex]{%
        \begin{scope}[x=1ex, y=1ex]
            \clip (-0.1,-0.1)
                --++ (-0, 1.2)
                --++ (0.6, 0)
                --++ (0, -0.6)
                --++ (0.6, 0)
                --++ (0, -1);
            \path[draw,
                line width = 0.5,
                rounded corners=0.5]
                (0,0) rectangle (1,1);
        \end{scope}
        \path[draw, line width = 0.5] (0.5, 0.5)
            -- (1, 1);
        \path[draw, line width = 0.5] (0.6, 1)
            -- (1, 1) -- (1, 0.6);
        }
    }

% Other commands
\newcommand\tab[1][0.5cm]{\hspace*{#1}} % for code boxes


% Read Git Data from Github Actions Workflow
% Defaults to gitinfo2 for local builds
\IfFileExists{gitInfo.txt}
	{\input{gitInfo.txt}}
	{
		\newcommand{\gitRelease}{(Local Release)}
		\newcommand{\gitSHA}{\gitHash}
		\newcommand{\gitDate}{\gitAuthorIsoDate}
	}

% Custom Titlepage
\defbeamertemplate*{title page}{customized}[1][]
{
  \vspace{-0cm}\hfill\includegraphics[width=2.5cm]{../config/logo_cif}
  \includegraphics[width=1.9cm]{../config/seal_wwz}
  \\ \vspace{2em}
  \usebeamerfont{title}\textbf{\inserttitle}\par
  \usebeamerfont{title}\usebeamercolor[fg]{title}\insertsubtitle\par  \vspace{1.5em}
  \small\usebeamerfont{author}\insertauthor\par
  \usebeamerfont{author}\insertinstitute\par \vspace{2em}
  \usebeamercolor[fg]{titlegraphic}\inserttitlegraphic
    \tiny \noindent \texttt{Release Ver.: \gitRelease}\\ 
    \texttt{Version Hash: \gitSHA}\\
    \texttt{Version Date: \gitDate}\\ \vspace{1em}
    
    
    \iffalse
  \link \href{https://github.com/cifunibas/Bitcoin-Blockchain-Cryptoassets/blob/main/slides/intro.pdf}
  {Get most recent version}\\
  \link \href{https://github.com/cifunibas/Bitcoin-Blockchain-Cryptoassets/blob/main/slides/intro.pdf}
  {Watch video lecture}\\ 
  
  \fi
  
  \vspace{1em}
  License: \texttt{Creative Commons Attribution-NonCommercial-ShareAlike 4.0 International}\\\vspace{2em}
  \includegraphics[width = 1.2cm]{../config/license}
}


% tikzlibraries
\usetikzlibrary{decorations.pathreplacing}
\usetikzlibrary{decorations.markings}
\usetikzlibrary{positioning}
\usetikzlibrary{calc}
\captionsetup{font=footnotesize}

%%%%%%%%%%%%%%%%%%%%%%%%%%%%%%%%%%%%%%%%%%%%%%
%%%%%%%%%%%%%%%%%%%%%%%%%%%%%%%%%%%%%%%%%%%%%%
\begin{document}

\thispagestyle{empty}
\begin{frame}[noframenumbering]
	\titlepage
\end{frame}

%%%
\begin{frame}{Content Overview}
\small
\begin{enumerate}
	\item<1-> \textbf{Introduction and Blockchain Fundamentals}\\
		Recap of basic blockchain building blocks
		\vspace{0.2em}
	\item<2-> \textbf{Ethereum Basics}\\
		The specifics of the Ethereum as Smart Contract platform
		\vspace{0.2em}
	\item<3-> \textbf{Smart Contract Programming}\\
		Hands-on introduction to the basics of Smart Contract programming with emphasis on exercises
	\item<4-> \textbf{DeFi Overview}\\
		A short introduction to Decentralized Finance
		\vspace{0.2em}
	\item<5-> \textbf{DeFi Asset Layer}\\
		Tokenization from an economics and technological point of view
		\vspace{0.2em}
	\item<6-> \textbf{DeFi Protocol Layer}\\
			Overview of the DeFi ecosystem, key components and major protocol types.
	\item<7-> \textbf{Discussion and Outlook}\\
			Discussion, outlook and some advanced topics
\end{enumerate}

\end{frame}
%%%	

%%%
\begin{frame}{Why Smart Contracts?}

Vending Machines can be seen as predecessors of Smart Contracts \cite{NS:94,NS:97}. Agreements are automated and breach of contract is costly.

\vspace{0.5em}

\begin{minipage}{0.4\textwidth}
	\vspace{0.5em}
	\centering
	\includegraphics[width=2.5cm]{../assets/images/vendingmachine.png}
\end{minipage}
\begin{minipage}{0.55\textwidth}
	\begin{samplecode}{Simple Vending Machine (Pseudo Code)}
		if(coin $>=$ price) \{ \\
			\tab dispenseBeverage(); \\
			\tab returnChange(coin - price);\\
		\} else \{ \\
			\tab print("insufficient funds"); \\
		\}
	\end{samplecode}	
\end{minipage}

\uncover<2->{
\vspace{1.0em}
\textbf{But:}\\
\begin{itemize}
	\item Trust based! Closed source $\rightarrow$ Contract is not observable.
	\item Execution environment, i.e., hardware is under control of seller.
\end{itemize}
}

\uncover<3->{
\vspace{1.0em}
\textbf{$\Rightarrow$ Smart contracts on public blockchains address this.}
}

\end{frame}
%%%

%%%
\begin{frame}{Popular Smart Contract Blockchains}

Blockchains based on the Ethereum Virtual Machine (EVM):

\vspace{0.2em}

\begin{center}
\begin{tikzpicture}[scale=1, every node/.style={scale=1}]
	\begin{footnotesize}
	\coordinate (1) at (-4, 3);
	\coordinate (2) at (-2, 3);
	\coordinate (3) at (0, 3);
	\coordinate (4) at (2, 3);
	\coordinate (5) at (4, 3);
	
	\node (a) at (1) {\includegraphics[height = 0.1\textheight]{../assets/images/ethereum}};
	\node[below = 3pt] at (a.south) {Ethereum};

	\node (b) at (2) {\includegraphics[height = 0.10\textheight]{../assets/images/bsc}};
	\node[below = 3pt] at (b.south) {BSC};
	
	\node (c) at (3) {\includegraphics[height = 0.10\textheight]{../assets/images/polygon}};
	\node[below = 3pt] at (c.south) {Polygon};
	
	\node (d) at (4) {\includegraphics[height = 0.10\textheight]{../assets/images/tron}};
	\node[below = 3pt] at (d.south) {Tron};
	
	\node (e) at (5) {\includegraphics[height = 0.10\textheight]{../assets/images/etc}};
	\node[below = 3pt] at (e.south) {ETC};
	
%	\node (e) at (5) {\includegraphics[height = 0.10\textheight]{../assets/images/arbitrum}};
%	\node[below = 3pt] at (e.south) {Arbitrum};
	
\end{footnotesize}

\end{tikzpicture}
\end{center}

\vspace{1em}



Non-EVM-based Blockchains:

\vspace{0.2em}

\begin{center}
\begin{tikzpicture}[scale=1, every node/.style={scale=1}]
	\input{../assets/figures/blockchainsNonEVM}
\end{tikzpicture}
\end{center}

\vspace{1em}
This is a non-exhaustive list that will likely age poorly.

\end{frame}
%%%	

%%%
\begin{frame}{Why Ethereum / EVM?}
	All EVM-based blockchains share the same smart contract development languages.
	
	\vspace{1 em}
	
	Ethereum is currently THE dominant platform:
	\begin{itemize}
		\item Market Cap
		\item Economic Activity
		\item Developer Activity
		\item Community
	\end{itemize}
	\vspace{1 em}
	Understanding Ethereum will help you to understand any protocol.
\end{frame}

%%%

%%%
\begin{frame}{Electric Capital Developer Report 2020}
	
	\begin{figure}
		\includegraphics[width=\textwidth]{../assets/images/electric_capital}
		
		\caption{\footnotesize  \href{https://github.com/electric-capital/developer-reports}{\link Electric Capital Developer Report 2020, p. 46} \cite{EC20}} 
	\end{figure}


\end{frame}

%%%

%%%
\begin{frame}{Why Decentralized Finance (DeFi)?}

\begin{figure}[t]
	\centering	
	\resizebox{0.8\textwidth}{!}{
	\begin{tikzpicture}[scale=1.0, every node/.style={scale=1.0}]
			\input{../assets/figures/defi_stack.tex}
	\end{tikzpicture}}
	\caption{The DeFi Stack \cite{FS:21}}
\end{figure}

\vspace{-1.0em}

\begin{itemize}
	\item Most mature and diverse smart contract-based ecosystem
	\item Relevancy of applications from an economics perspective
\end{itemize}

\end{frame}
%%%

%%%
\begin{frame}{Interdisciplinaray Approach}

	\uncover<1->{
		\begin{figure}[h]
  			\center
			\input{../assets/figures/interdiciplinarity.tex}
		\end{figure}
	}
	\vspace{1em}
Public blockchains can only be fully understood, when they are studied from various perspectives. This is the reason why this course uses an \color{focus} \textbf{interdisciplinary} \color{black} approach.	


\end{frame}
%%%

%%%
\begin{frame}{Programming and Computer Science Exposure}

\textbf{Programming Language}
\vspace{1em}

\begin{minipage}{0.16\textwidth}
	\begin{figure}[t]
		\begin{tikzpicture}[scale=1.0, every node/.style={scale=1.0}]
				

% Logos
\node at (0,1) {\includegraphics[height = 1.2cm]{../assets/images/logo_solidity.png}};



% Labels
\draw (0,0) node  {solidity};

		\end{tikzpicture}
	\end{figure}
\end{minipage}
\vspace{2em}

\textbf{Tools}
\vspace{1em}

\begin{minipage}{0.95\textwidth}
	\begin{figure}[t]	
		\begin{tikzpicture}[scale=0.8, every node/.style={scale=0.8}]
				

% Logos
\node at (0,2) {\includegraphics[height = 1.2 cm]{../assets/images/logo_atom.png}};
\node at (1.8, 2) {\includegraphics[height = 1.2 cm]{../assets/images/logo_vscode.png}};
\node at (3.6,2) {\includegraphics[height = 1.2 cm]{../assets/images/logo_remix.png}};
\node at (5.4,2) {\includegraphics[height = 1.2 cm]{../assets/images/logo_github.png}};
\node at (7.2,2) {\includegraphics[height = 1.2 cm]{../assets/images/logo_ganache.png}};
\node at (9,2) {\includegraphics[height = 1.2 cm]{../assets/images/logo_truffle.png}};
\node at (10.8,2) {\includegraphics[height = 1.2 cm]{../assets/images/logo_metamask.png}};


% Labels
\footnotesize{
	\draw (0,1.1) node [minimum height = 1cm] {Atom};
	\draw (1.8, 1.1) node [minimum height = 1cm] {VS Code};
	\draw (3.6,1.1) node [minimum height = 1cm] {Remix};
	\draw (5.4,1.1) node [minimum height = 1cm] {Github};
	\draw (7.2,1.1) node [minimum height = 1cm] {Ganache};
	\draw (9,1.1) node [minimum height = 1cm] {Truffle};
	\draw (10.8,1.1) node [minimum height = 1cm] {Metamask};			
}

% Groups

\draw [decorate,decoration={brace,amplitude=5pt,mirror},xshift=0 pt,yshift=0pt] (-0.8,0.8) -- (4.4,0.8) node [midway, below, align = center, yshift = -0.25cm, text width = 3.6 cm] {Integrated Development Environments};

\draw [decorate,decoration={brace,amplitude=5pt,mirror},xshift=0 pt,yshift=0pt] (4.6,0.8) -- (6.2,0.8) node [midway, below, align = center, yshift = -0.25cm, text width = 1.8 cm] {Version Control};

\draw [decorate,decoration={brace,amplitude=5pt,mirror},xshift=0 pt,yshift=0pt] (6.4,0.8) -- (9.8,0.8) node [midway, below, align = center, yshift = -0.25cm, text width = 2 cm] {Blockchain Management};

\draw [decorate,decoration={brace,amplitude=5pt,mirror},xshift=0 pt,yshift=0pt] (10,0.8) -- (11.6,0.8) node [midway, below, align = center, yshift = -0.25cm, text width = 3.6 cm] {Testing};


		\end{tikzpicture}
	\end{figure}
\end{minipage}



\end{frame}
%%%

%%%
\begin{frame}{Some Lecture Conventions}

	\begin{columns}[T]
		\begin{column}{0.45\textwidth}
			\vspace{-1em}
			\begin{samplecode}{Basic Contract Structure (Solidity)}
				Pragma solidity  \textasciicircum 0.8.7; \
				contract HelloWorld \{ \\
				\color{softanthracite} /* Instructions */ \\
				\color{black} \}
			\end{samplecode}			
		\end{column} %\hfill

		\begin{column}{0.5\textwidth}
			\textbf{Sample Code Box}\\
			Contains code snippets in Solidity or pseudo code.
		\end{column}
	\end{columns}

\vspace{2.5em}

	\begin{columns}[T]
		\begin{column}{0.45\textwidth}
			\vspace{-0.8em}
			\begin{keytakeaway}{Key Takeaway}
				Transactions can be used to transfer value, interact with an existing smart contract and to deploy new smart contracts.
			\end{keytakeaway}			
		\end{column}
		
		\begin{column}{0.5\textwidth}
			\textbf{Key Take-Away}\\
			Highlights important concepts and definitions.
		\end{column}
	\end{columns}

\vspace{2.5em}

	\begin{columns}[T]
		\begin{column}{0.45\textwidth}
			\vspace{-0.9em}
			\begin{exercise}{Exercise 1}
				Get Ropsten Ether from the faucet and deploy your first smart contract on Ropsten Testnet.
			\end{exercise}					
		\end{column}
		\begin{column}{0.5\textwidth}
		\textbf{Exercises}\\
			Things for you to try out.
		\end{column}
	\end{columns}

\end{frame}
%%%

%%%
\begin{frame}{Part of Multi-Course Series}

Blockchain courses have been part of the University of Basel's curriculum since 2017.

\vspace{1.5em}

\begin{columns}
	\begin{column}{0.35 \textwidth}
		\uncover<1->{
			\includegraphics[width = 4cm]{../config/logo_cif}
		}
	\end{column}
	\begin{column}{0.6 \textwidth}	
			\begin{itemize}
			\item<2-> This is a University graduate-/ master-level course.
			\item<3-> It is part of a series of courses.
			\item<4-> It is the second course to switch to an open lecture format.
		\end{itemize}
	\end{column}	
\end{columns}

%\vspace{2em}
%
%\uncover<5->{
%	$\rightarrow$ There will be more open lecture courses.
%}

\end{frame}
%%%

%%%
\begin{frame}{Three Options to Take This Course}

The goal of our open lectures is to make teaching resources freely available. There are \color{focus} \textbf{three options} \color{black} for taking this course:\vspace{1em}

\begin{table}\footnotesize
	\begin{tabular}{lccccc}
	\hline \hline
									& Videos 		& Platform 		& Quizzes  		& Group Project	& ECTS 			\\ \cline{2-6}
		YouTube 	 				& $\checkmark$	& 				& 				& 				&				\\
		Cryptolectures.io 			& $\checkmark$	& $\checkmark$	& $\checkmark$	&				&				\\
		University of Basel			& $\checkmark$	& $\checkmark$	& $\checkmark$	& $\checkmark$	&$\checkmark$	\\
		\hline \hline
	\end{tabular}
\end{table} \vspace{2em}

\uncover<2->{
\link \href{https://www.youtube.com/channel/UCOA52m4BOqtI8cHIx4zJAWg}
  {YouTube Channel}  \\
\link \href{https://www.cryptolectures.io}{Cryptolectures.io} \\
\link \href{https://www.unibas.ch/en/Studies/Application-Admission.html}{University of Basel - General Information}
}

\end{frame}
%%%

%%%
\begin{frame}{Welcome to the Frontier...}

The topic is in great demand and will help you to fast track your professional finance career. That being said, there will be \color{focus} some hiccups and moving parts... \color{black} \vspace{1em} 
	\begin{itemize}
		\item Ropsten $\rightarrow$ Goerli (potentially even $\rightarrow$ Sepolia).
		\item Frontend changes for the tools we are using. 
		\item New consensus protocol and switch to \texttt{EIP-1559}.
	\end{itemize}	
	
	\vspace{2em}
	
If you have any questions, have found a bug or want to support the open lectures project, feel free to approach us at any time. \\

\vspace{1em}
Also: opportunity to join optional talks/seminars.
	
\end{frame}
%%%

%%%
\begin{frame}{Meet the Open Crypto Lectures Team}
	\begin{columns}[T]
		\begin{column}{0.31\textwidth}
			\center \textbf{\small{Instructor}}
			\begin{table}\small
				\begin{tabular}{c}
					Fabian Schär\\
					\href{https://linkedin.com/in/fabian-schaer/}{\faLinkedinSquare}\ \href{https://twitter.com/fschaer}{\faTwitterSquare}\\
				\end{tabular}
			\end{table}
		\end{column}
		\begin{column}{0.31\textwidth}
			\center \textbf{\small{Teaching Assistants}}
			\begin{table}\small
				\begin{tabular}{c}
					%Tobias Bitterli\\
					%\href{https://linkedin.com/in/tobiasbitterli/}{\faLinkedinSquare}\ \href{https://twitter.com/tobias_bitterli}{\faTwitterSquare}\\
					Mitchell Goldberg\\
					\href{https://linkedin.com/in/mitchell-goldberg/}{\faLinkedinSquare}\ \href{https://twitter.com/golmit_crypto}{\faTwitterSquare}\\
					\vspace{0.5em}\\
					Matthias Nadler\\
					\href{https://linkedin.com/in/mat-nadler/}{\faLinkedinSquare}\ \href{https://twitter.com/mat_nadler}{\faTwitterSquare}\\
					\vspace{0.5em}\\
					Katrin Schuler\\
					\href{https://linkedin.com/in/kmschuler/}{\faLinkedinSquare}\ \href{https://twitter.com/Katatcrypt}{\faTwitterSquare}\\
					\vspace{0.5em}\\
					Dario Thürkauf\\
					\href{https://linkedin.com/in/dario-thuerkauf/}{\faLinkedinSquare} \href{https://twitter.com/dario_thuerkauf}{\faTwitterSquare}\\
				\end{tabular}
			\end{table}
		\end{column}
		\begin{column}{0.31\textwidth}
			\center \textbf{\small{Student Assistants}}
			\begin{table}\small
				\begin{tabular}{c}
					Jonas Ruchti\\
					\href{https://linkedin.com/in/jonas-ruchti-a29042221}{\faLinkedinSquare}\ \href{https://twitter.com/jonas_ruchti}{\faTwitterSquare}\\
				\end{tabular}
			\end{table}
		\end{column}
	\end{columns}
\end{frame}
%%%

%%%
\begin{frame}{Grading}

\uncover<1->{
	\textbf{Group Project} (40\% of final grade)
	\begin{itemize}
		\item Smart Contract programming project
		\item Groups of 2-4 students
	\end{itemize}
}

\vspace{1em}

\uncover<2->{
\textbf{Exam} (60\% of final grade)
	\begin{itemize}
		\item 90 minutes
		\item Closed book
		\item T/F, MC, numbers and text/figure boxes
		\item You may use a non-programmable calculator (\link \href{https://wwz.unibas.ch/en/studies/examinations/use-of-materials-and-aids/}{Rules})
	\end{itemize}
	}

\end{frame}
%%%

%%%
\begin{frame}{Important Dates}

\uncover<1->{
\textbf{Kickoff Event}\\
18th September 2023, 2:15pm - 4pm
\begin{itemize}
	\item General information, expectations, dates.
\end{itemize}
}
\vspace{0.5em}

\uncover<2->{
\textbf{Deadline for Team Applications}
2nd October 2023, 23:59:59.
\begin{itemize}
	\item Additional information: \hyperlink{TEAM}{\beamergotobutton{How to Form a Team}}
\end{itemize}
}
\vspace{0.5em}

\uncover<3->{
\textbf{Short Paper \& Slide Submission Deadline}\\
20th November 2023, 23:59:59.
\begin{itemize}
	\item Submission: \href{mailto:mitchell.goldberg@unibas.ch?subject=Short\%20Paper\%20and\%20Slides}{\texttt{mitchell.goldberg@unibas.ch}}
\end{itemize}
} 
\vspace{0.5em}

\uncover<4->{
\textbf{Seminar Days (Group Presentations)}\\
27th November 2023, 8:15am - 6:00pm\\
28th November 2023, 8:15am - noon\\
29th November 2023, 2:15pm - 6:00pm
\begin{itemize}
	\item Please block all three slots. If we do not need all of them (very likely), I will let you know in advance. 
\end{itemize}
}
\vspace{0.5em}

\uncover<6->{
\textbf{Exam}\\
Date tbd, will soon be available in \href{https://vorlesungsverzeichnis.unibas.ch/de/home?id=277159}{the course directory}.
}
\end{frame}
%%%

%%%
\begin{frame}{Our Expectations for Group Project}
	\begin{itemize}
		\item<1-> \color{focus}Design and develop a financial protocol\color{black}, e.g., simple exchange, lending market, token basket, governance/voting system, lottery, NFT fractionalizer...
		\item<2-> You are \color{focus} allowed to copy code \color{black} snippets, however, there are a few rules.
		\begin{enumerate}
			\item<3-> \color{focus} You must disclose \color{black} the \color{focus} extent \color{black} to which and the \color{focus} sources \color{black} from where you have copied code. Non-compliance $\rightarrow$ 1.0 and potentially ``Plagiatsverfahren''. \vspace{0.35em}
			\item<4-> There must be some \color{focus} creativity and effort \color{black} visible in your submission. A simple copy of another protocol is not sufficient.\vspace{0.35em}
			\item<5-> \color{focus} It's ok to fail! \color{black} If you are running into issues, show the steps you have taken to mitigate them. If the effort is clearly visible you can get a sufficient grade, even if your protocol is not running.
		\end{enumerate}
		\item<6-> Short paper $\leq$ 8 pages. 
		\item<7-> Presentation $\leq$ 15 minutes, followed by 10 minutes Q\&A.
	\end{itemize}
\end{frame}
%%%

%%%
\begin{frame}{How to Form a Team}
\label{TEAM}
	Talk to your colleagues and try to form groups - ideally 3-4 team members. Submit your group by sending an email to \href{mailto:mitchell.goldberg@unibas.ch?subject=Group\%20SCDF}{\texttt{mitchell.goldberg@unibas.ch}}. The email must contain:
	\begin{itemize}
		\item<2-> Full name of all group members
		\item<3-> Student ID (Matrikelnummer) of all group members
		\item<4-> Preliminary topic proposal
		\item<5-> All team members must be cc'ed.
	\end{itemize}

\vspace{1.5em}
\uncover<6->{
\color{focus}The deadline for this mail is October 2nd 2022, end of day. \color{black} Anyone who is officially enrolled in the class but did not submit a team application by this deadline, will be randomly matched with other inactive people. 
}
\end{frame}
%%%

%%%
\begin{frame}{Team Issues and Potential Conflicts}
\begin{itemize}
	\item<1-> In case of conflicts within your team, please try to find a \color{focus} reasonable solution\color{black}. 
	\item<2-> If no agreement can be reached, there is \color{focus} the option of a ``fork''\color{black}. Any progress up to that point belongs to all group members. 
	\item<3-> In any case, keep in mind that doing the work in your group is an \color{focus} excellent preparation for the exam\color{black}. 
	\item<4-> If a group \color{focus} has 2 or less members, it may merge with another group \color{black}. No group may have more than 5 members (rule applies to mergers). 
	\item<5-> Any mergers are \color{focus} s.t. the approval \color{black} of all involved parties and the professor. Forks are s.t. the approval of the professor.
\end{itemize}
\end{frame}
%%%


%%%
\begin{frame}{Contact and Resources}
\begin{itemize}
	\item<1-> Feel free to approach me in case of questions. To make sure that you will receive a prompt reply, use the email subject line "question unibas" (special high priority inbox). \\ $\rightarrow$ \href{mailto:f.schaer@unibas.ch?subject=question\%20unibas}{\texttt{f.schaer@unibas.ch}}
	\item<2-> For general questions regarding the course and very swift responses, you can send an email to the TAs. \\$\rightarrow$ \href{mailto:mitchell.goldberg@unibas.ch?subject=question\%20unibas}{\texttt{mitchell.goldberg@unibas.ch}}
	\item<3-> Some of the exercises require Goerli Testnet Ether. To get Goerli Eth, you can send an email, including your Testnet address to: \\
	$\rightarrow$ \href{mailto:dario.thuerkauf@unibas.ch?subject=Goerli\%20ETH}{\texttt{dario.thuerkauf@unibas.ch}}.
\end{itemize}
\end{frame}
%%%


%%%
\begin{frame}%[allowframebreaks]
\frametitle{References and Recommended Reading}
	\bibliographystyle{amsplain}
	\bibliography{../assets/bib/refs}
\end{frame}
%%%



\end{document}