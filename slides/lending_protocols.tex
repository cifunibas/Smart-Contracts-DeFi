% Choose one to switch between slides and handout
%\documentclass[]{beamer}
\documentclass[handout]{beamer}

% Video Meta Data
\title{Smart Contracts and Decentralized Finance}
\subtitle{Lending Protocols}
\author{Prof. Dr. Fabian Schär}
\institute{University of Basel}

% Config File
% Packages
\usepackage[utf8]{inputenc}
\usepackage{hyperref}
\usepackage{gitinfo2}
\usepackage{tikz}
 \usetikzlibrary{calc}
\usepackage{amsmath}
\usepackage{mathtools}
\usepackage{bibentry}
\usepackage{xcolor}
\usepackage{colortbl} % Add colour to LaTeX tables
\usepackage{caption}
\usepackage[export]{adjustbox}
\usepackage{pgfplots} \pgfplotsset{compat = 1.17}
\usepackage{makecell}
\usepackage{fancybox}
\usepackage{ragged2e}
\usepackage{fontawesome}
\usepackage{seqsplit}
\usepackage{tabularx}
\usepackage{tcolorbox}
\usepackage{booktabs} % use instead  \hline in tables

% Color Options
\definecolor{highlight}{rgb}{0.65,0.84,0.82}
\definecolor{focus}{rgb}{0.72, 0, 0}
\definecolor{lightred}{rgb}{0.8,0.5,0.5}
\definecolor{midgray}{RGB}{190,195,200}

 %UniBas Main Colors
\definecolor{mint}{RGB}{165,215,210}
\definecolor{anthracite}{RGB}{45,55,60}
\definecolor{red}{RGB}{210,5,55}

 %UniBas Color Palette (for graphics)
\definecolor{strongmint}{RGB}{30,165,165}
\definecolor{darkmint}{RGB}{0,110,110}
\definecolor{softanthracite}{RGB}{140,145,150}
\definecolor{brightanthracite}{RGB}{190,195,200}
\definecolor{softred}{RGB}{235,130,155}

%Custom Colors
\definecolor{lightergray}{RGB}{230, 230, 230}



% Beamer Template Options
\beamertemplatenavigationsymbolsempty
\setbeamertemplate{footline}[frame number]
\setbeamercolor{structure}{fg=black}
\setbeamercolor{footline}{fg=black}
\setbeamercolor{title}{fg=black}
\setbeamercolor{frametitle}{fg=black}
\setbeamercolor{item}{fg=black}
\setbeamercolor{}{fg=black}
\setbeamercolor{bibliography item}{fg=black}
\setbeamercolor*{bibliography entry title}{fg=black}
\setbeamercolor{alerted text}{fg=focus}
\setbeamertemplate{items}[square]
\setbeamertemplate{enumerate items}[default]
\captionsetup[figure]{labelfont={color=black},font={color=black}}
\captionsetup[table]{labelfont={color=black},font={color=black}}

\setbeamertemplate{bibliography item}{\insertbiblabel}

%tcolor boxes
\newtcolorbox{samplecode}[2][]{
  colback=mint, colframe=darkmint, coltitle=white,
  fontupper = \ttfamily\scriptsize, fonttitle= \bfseries\scriptsize,
  boxrule = 0mm, arc = 0mm,
  boxsep = 1.3mm, left = 0mm, right = 0mm, top = 0.5mm, bottom = 0mm, middle=0mm,
  #1,title=#2}
  
\newtcolorbox{keytakeaway}[2][]{
  colback=softred, colframe=red, coltitle=white,
  fontupper = \scriptsize, fonttitle= \bfseries\scriptsize,
  boxrule = 0mm, arc = 0mm,
  boxsep = 1.3mm, left = 0mm, right = 0mm, top = 0.5mm, bottom = 0mm, middle=0mm,
  #1,title=#2}

\newtcolorbox{exercise}[2][]{
  colback=brightanthracite, colframe=anthracite, coltitle=white,
  fontupper = \scriptsize, fonttitle= \bfseries\scriptsize,
  boxrule = 0mm, arc = 0mm,
  boxsep = 1.3mm, left = 0mm, right = 0mm, top = 0.5mm, bottom = 0mm, middle=0mm,
  #1,title=#2}



% Link Icon Command 
\newcommand{\link}{%
    \tikz[x=1.2ex, y=1.2ex, baseline=-0.05ex]{%
        \begin{scope}[x=1ex, y=1ex]
            \clip (-0.1,-0.1)
                --++ (-0, 1.2)
                --++ (0.6, 0)
                --++ (0, -0.6)
                --++ (0.6, 0)
                --++ (0, -1);
            \path[draw,
                line width = 0.5,
                rounded corners=0.5]
                (0,0) rectangle (1,1);
        \end{scope}
        \path[draw, line width = 0.5] (0.5, 0.5)
            -- (1, 1);
        \path[draw, line width = 0.5] (0.6, 1)
            -- (1, 1) -- (1, 0.6);
        }
    }

% Other commands
\newcommand\tab[1][0.5cm]{\hspace*{#1}} % for code boxes


% Read Git Data from Github Actions Workflow
% Defaults to gitinfo2 for local builds
\IfFileExists{gitInfo.txt}
	{\input{gitInfo.txt}}
	{
		\newcommand{\gitRelease}{(Local Release)}
		\newcommand{\gitSHA}{\gitHash}
		\newcommand{\gitDate}{\gitAuthorIsoDate}
	}

% Custom Titlepage
\defbeamertemplate*{title page}{customized}[1][]
{
  \vspace{-0cm}\hfill\includegraphics[width=2.5cm]{../config/logo_cif}
  \includegraphics[width=1.9cm]{../config/seal_wwz}
  \\ \vspace{2em}
  \usebeamerfont{title}\textbf{\inserttitle}\par
  \usebeamerfont{title}\usebeamercolor[fg]{title}\insertsubtitle\par  \vspace{1.5em}
  \small\usebeamerfont{author}\insertauthor\par
  \usebeamerfont{author}\insertinstitute\par \vspace{2em}
  \usebeamercolor[fg]{titlegraphic}\inserttitlegraphic
    \tiny \noindent \texttt{Release Ver.: \gitRelease}\\ 
    \texttt{Version Hash: \gitSHA}\\
    \texttt{Version Date: \gitDate}\\ \vspace{1em}
    
    
    \iffalse
  \link \href{https://github.com/cifunibas/Bitcoin-Blockchain-Cryptoassets/blob/main/slides/intro.pdf}
  {Get most recent version}\\
  \link \href{https://github.com/cifunibas/Bitcoin-Blockchain-Cryptoassets/blob/main/slides/intro.pdf}
  {Watch video lecture}\\ 
  
  \fi
  
  \vspace{1em}
  License: \texttt{Creative Commons Attribution-NonCommercial-ShareAlike 4.0 International}\\\vspace{2em}
  \includegraphics[width = 1.2cm]{../config/license}
}


% tikzlibraries
\usetikzlibrary{decorations.pathreplacing}
\usetikzlibrary{decorations.markings}
\usetikzlibrary{positioning}
\usetikzlibrary{calc}
\captionsetup{font=footnotesize}

%%%%%%%%%%%%%%%%%%%%%%%%%%%%%%%%%%%%%%%%%%%%%%
%%%%%%%%%%%%%%%%%%%%%%%%%%%%%%%%%%%%%%%%%%%%%%
\begin{document}

\thispagestyle{empty}
\begin{frame}[noframenumbering]
	\titlepage
\end{frame}

%%%
\begin{frame}{Introduction}
Besides exchanging assets, \textbf{borrowing} and \textbf{lending} are the most fundamental needs in a financial system. \\

\vspace{1em}
Due to openness and pseudonymity, DeFi loans are always \textbf{collateralized loans} (exception: flash loans).\\

\vspace{1em}
There are two main lending protocol archetypes in DeFi: 
\vspace{1em}
\begin{itemize}
  \item \textbf{Collateral Debt Protocols}
  \item \textbf{Collateral Debt Markets}
\end{itemize}
	
\end{frame}
%%%	


%%%
\begin{frame}{Some Terminology around Lending in DeFi}

\vspace{1em}

		
			\uncover<2-> {\textbf{Position}: Entirety of collateral and debt an address has within a protocol.}
			
\vspace{1em}

			
			\uncover<3->  {\textbf{Loan-to-Value} (\textit{LTV}): Percentage at which the value of a collateral asset is counted towards the borrowing capacity by a protocol.}
			
			\uncover<3->{ 	\begin{equation}
	  LTV \in [0,1] \label{eq:LTV} 
	\end{equation}}



\end{frame}
%%%	


%%%
\begin{frame}{Some Terminology around Lending in DeFi}

\vspace{1em}


	\uncover<4->{ \textbf{Borrowing capacity} (BC): Total amount a position may borrow via the protocol, given the value of the assets provided as collateral (c) and their respective LTVs}
	
	\uncover<4->{\begin{equation}
	  BC = \sum c_i LTV_i  \label{eq:BC} 
	\end{equation}}
		
$	\rightarrow$ New debt for a position can only be drawn as long as its outstanding debt (d) across all assets is below borrowing capacity, i.e.
\begin{equation}
	  BC > \sum d_i\label{eq:BC_restriction} 
	\end{equation}

	\vspace{1em}

\end{frame}
%%%

%%%
\begin{frame}{Some Terminology around Lending in DeFi}

\vspace{1em}

	\uncover<4->{{\normalsize \textbf{Liquidation threshold } (LT)}: Threshold value of debt in percentage of collateral value at which a position starts to become unhealthy and thus available for liquidation}
	
	\uncover<4-> {\begin{equation}
	  LTV \in [0,1] while LT \geq LTV   \label{eq:LT} 
	\end{equation}}
	


	\uncover<4->{\textbf{Health Factor } (HF): Relative value of collateral, adjusted for LT, to outstanding debt (d) of a position}

\begin{equation}
	  HF = \frac{\sum c_i LT_i}{\sum d_i} \label{eq:HF} 
\end{equation}

	\vspace{1em}

 $	\rightarrow$ A position with $HF \geq 1$ it is considered \textit{healthy}
 
 \vspace{1em}

 $	\rightarrow$ Collateral of \textit{unhealthy} positions is up for liquidation against repayment of (parts) of the position’s debt.  


\end{frame}
%%%


%%%
\begin{frame}{Some Terminology around Lending in DeFi}

\vspace{1em}

	\uncover<4->{{\normalsize \textbf{Liquidation threshold } (LT)}: Threshold value of debt in percentage of collateral value at which a position starts to become unhealthy and thus available for liquidation}
	
	\uncover<4-> {\begin{equation}
	  BC = \sum c_i LTV_i  \label{eq:BC} 
	\end{equation}}
	
$	\Rightarrow$ New debt for a position can only be drawn as long as its outstanding debt (d) across all assets is below borrowing capacity, i.e.
\begin{equation}
	  BC > \sum d_i\label{eq:BC_restriction} 
	\end{equation}

	\vspace{1em}

\end{frame}
%%%

%%%
\begin{frame}{Some Terminology around Lending in DeFi}

\vspace{1em}

	\uncover<4->{{\normalsize \textbf{Liquidation threshold } (LT)}: Threshold value of debt in percentage of collateral value at which a position starts to become unhealthy and thus available for liquidation}
	
	\uncover<4-> {\begin{equation}
	  BC = \sum c_i LTV_i  \label{eq:BC} 
	\end{equation}}
	
$	\Rightarrow$ New debt for a position can only be drawn as long as its outstanding debt (d) across all assets is below borrowing capacity, i.e.
\begin{equation}
	  BC > \sum d_i\label{eq:BC_restriction} 
	\end{equation}
	
	\vspace{1em}

\end{frame}
%%%




%%%
\begin{frame}{Collateral Debt Protocols (CDP)}

	
\end{frame}
%%%	


%%%
\begin{frame}{Collateral Debt Markets (CDM)}

	
\end{frame}
%%%	


%%%
\begin{frame}{Lender's Perspective}
Transfers assets to the protocol that then can be borrowed against interest $\rightarrow$  Only required in CDM protocols.

\vspace{1em}

Commonly receives a liquidity token for the assets made available, that

\vspace{1em}

\begin{itemize}
  \item Represents a transferable claim for the assets provided
  \item Is used by the protocol to account for interest earned by the lender.
  \item Is burnt against the provided assets plus interest, when the assets are withdrawn from the protocol
\end{itemize}


\end{frame}
%%%	


%%%
\begin{frame}{Lender's Perspective}

Interest received depends on the utilization of the respective asset by borrowers and the proportion kept by the protocol.


$\rightarrow $ Interest earned is NOT risk free: While commonly a reserve mechanism is in place to protect lenders in case of malfunctions or tail-events, they may not suffice, leaving lenders with liquidity tokens, that can not or only partially be reimbursed.


\end{frame}
%%%	



%%%
\begin{frame}{Comparison of two Liquidity Token setups}

	\begin{table}
		\begin{tabular}{ll}
			A & B\\
			C & D
		\end{tabular}
		\caption{This is a table}
		\label{tbl:simpletable}
	\end{table}
	
\end{frame}
%%%	


%%%
\begin{frame}{CDM: Utilization Depending Interest }

$if \;L_t > 0 : \; U_t = \dfrac{D_t}{L_t}$

\vspace{1em}

$if \;  U_t < U_{opt} : \; R_t = R_0 + \dfrac{U_t} {U_{opt}}  \, S_1$

\vspace{1em}

$if \; U_t \leq U_{opt} : \; R_t = R_0 + S_1 + \dfrac{U_t - U_{opt}}{U_{exc}}\, S_2$
	
\end{frame}
%%%	


%%%
\begin{frame}{Overcollateralization and Liquidation}

	
\end{frame}
%%%	


%%%
\begin{frame}{Liquidations: Two approaches }

As smart contracts, protocols cannot become active by themselves. They rely on arbitrageurs to spot and liquidate unhealthy positions via offering them a liquidation incentive. Two  liquidation approaches can be observed here:

\vspace{1em}

\textbf{Discount Sale} (Aave, Compound)

\begin{itemize}
\item Any unhealthy position’s collateral is offered at a discount from the current on-chain oracle price.
\item Liquidator repays (part of the) position’s debt in the respective debt asset and in return receives collateral assets at the discounted price for the repaid amount.
\item Pro: Atomic transaction, fast.
\item Con: Requires high on-chain oracle price accuracy at all times. 

\end{itemize}

	
\end{frame}
%%%	



%%%
\begin{frame}{Liquidations: Two approaches }

\vspace{1em}

\textbf{Tend – Dent Auction} (MakerDAO)

\begin{itemize}
\item English auction of an unhealthy position’s collateral that can be started by anyone submitting the first bid.
\item Ends when either the maximum auction length is reached or the grace period since the last valid bid. Winning bidder can then release the collateral bought.


\item In the tend phase, bidders compete by offering to pay larger proportions of the outstanding debt in exchange for the collateral.
\item If a bidder offers to pay the debt in full, the auction enters the dent phase, where bidders can compete further by accepting a lower proportion of the collateral in exchange for repayment of all the debt.

\item Pro: Independent of on-chain oracles

\item Con: Time-consuming, costly process for bidders (multiple transactions), no lower bound in collateral sale price.
\end{itemize}

	
\end{frame}
%%%	


%%%
\begin{frame}{Flash Loans Overview}

Flash loans are a \textbf{concept unique to DeFi} and are the only loans in the ecosystem that \textbf{do not require collateral}.

\vspace{1em}

In a flash loan, the funds are borrowed, used, and paid back \textbf{all in one transaction}.

\vspace{1em}

Instead of collateral, lenders rely on \textbf{transaction atomicity}: If principal plus fee is not paid back in the same transaction, all is reverted as if the funds were never granted.

\vspace{1em}

The transaction is still part of the blockchain but is considered failed and does not lead to any state changes except for nonce update and transaction fees. 



	
\end{frame}
%%%	


%%%
\begin{frame}{Flash Loans Overview}
For a flash loan of $x$ with interest $\rho$ and / or a flat fee $\varphi$ to work,

\vspace{1em}
\begin{itemize}
\item If lending protocol is a CDM, it must have x liquid available:  $x \leq LP_{x^{'}}$ (Flash -minting loosens restriction)


\item  $ x(1+\rho) + \varphi$ must be returned to the protocol at the end of the transaction.


\end{itemize}

	
\end{frame}
%%%	


%%%
\begin{frame}{Example of Arbitrage with Flash Loan}

$\Pi = max (x^{*}-[x \,(1+\rho) + \varphi ],0) - \epsilon$
	
\end{frame}
%%%	


%%%
\begin{frame}{Summarizing Flash Loans}

	
\end{frame}
%%%	


%%%
\begin{frame}{Bib-References}
		Read the Bitcoin Whitepaper, \cite{nakamotoBitcoin2008}.
\end{frame}
%%%

%%%
\begin{frame}%[allowframebreaks]

References must include AAVE, Compund and MAKER Docs
Gervais DeFi Liq Paper, Fabian's Flashloan explainer

\frametitle{References and Recommended Reading}
	\bibliographystyle{amsplain}
	\bibliography{../assets/bib/refs}
\end{frame}
%%%



\end{document}