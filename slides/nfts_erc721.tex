% Choose one to switch between slides and handout
\documentclass[]{beamer}
%\documentclass[handout]{beamer}

% Video Meta Data
\title{Smart Contracts and Decentralized Finance}
\subtitle{Non-Fungible Tokens \& ERC-721}
\author{Prof. Dr. Fabian Schär}
\institute{University of Basel}

% Config File
% Packages
\usepackage[utf8]{inputenc}
\usepackage{hyperref}
\usepackage{gitinfo2}
\usepackage{tikz}
 \usetikzlibrary{calc}
\usepackage{amsmath}
\usepackage{mathtools}
\usepackage{bibentry}
\usepackage{xcolor}
\usepackage{colortbl} % Add colour to LaTeX tables
\usepackage{caption}
\usepackage[export]{adjustbox}
\usepackage{pgfplots} \pgfplotsset{compat = 1.17}
\usepackage{makecell}
\usepackage{fancybox}
\usepackage{ragged2e}
\usepackage{fontawesome}
\usepackage{seqsplit}
\usepackage{tabularx}
\usepackage{tcolorbox}
\usepackage{booktabs} % use instead  \hline in tables

% Color Options
\definecolor{highlight}{rgb}{0.65,0.84,0.82}
\definecolor{focus}{rgb}{0.72, 0, 0}
\definecolor{lightred}{rgb}{0.8,0.5,0.5}
\definecolor{midgray}{RGB}{190,195,200}

 %UniBas Main Colors
\definecolor{mint}{RGB}{165,215,210}
\definecolor{anthracite}{RGB}{45,55,60}
\definecolor{red}{RGB}{210,5,55}

 %UniBas Color Palette (for graphics)
\definecolor{strongmint}{RGB}{30,165,165}
\definecolor{darkmint}{RGB}{0,110,110}
\definecolor{softanthracite}{RGB}{140,145,150}
\definecolor{brightanthracite}{RGB}{190,195,200}
\definecolor{softred}{RGB}{235,130,155}

%Custom Colors
\definecolor{lightergray}{RGB}{230, 230, 230}



% Beamer Template Options
\beamertemplatenavigationsymbolsempty
\setbeamertemplate{footline}[frame number]
\setbeamercolor{structure}{fg=black}
\setbeamercolor{footline}{fg=black}
\setbeamercolor{title}{fg=black}
\setbeamercolor{frametitle}{fg=black}
\setbeamercolor{item}{fg=black}
\setbeamercolor{}{fg=black}
\setbeamercolor{bibliography item}{fg=black}
\setbeamercolor*{bibliography entry title}{fg=black}
\setbeamercolor{alerted text}{fg=focus}
\setbeamertemplate{items}[square]
\setbeamertemplate{enumerate items}[default]
\captionsetup[figure]{labelfont={color=black},font={color=black}}
\captionsetup[table]{labelfont={color=black},font={color=black}}

\setbeamertemplate{bibliography item}{\insertbiblabel}

%tcolor boxes
\newtcolorbox{samplecode}[2][]{
  colback=mint, colframe=darkmint, coltitle=white,
  fontupper = \ttfamily\scriptsize, fonttitle= \bfseries\scriptsize,
  boxrule = 0mm, arc = 0mm,
  boxsep = 1.3mm, left = 0mm, right = 0mm, top = 0.5mm, bottom = 0mm, middle=0mm,
  #1,title=#2}
  
\newtcolorbox{keytakeaway}[2][]{
  colback=softred, colframe=red, coltitle=white,
  fontupper = \scriptsize, fonttitle= \bfseries\scriptsize,
  boxrule = 0mm, arc = 0mm,
  boxsep = 1.3mm, left = 0mm, right = 0mm, top = 0.5mm, bottom = 0mm, middle=0mm,
  #1,title=#2}

\newtcolorbox{exercise}[2][]{
  colback=brightanthracite, colframe=anthracite, coltitle=white,
  fontupper = \scriptsize, fonttitle= \bfseries\scriptsize,
  boxrule = 0mm, arc = 0mm,
  boxsep = 1.3mm, left = 0mm, right = 0mm, top = 0.5mm, bottom = 0mm, middle=0mm,
  #1,title=#2}



% Link Icon Command 
\newcommand{\link}{%
    \tikz[x=1.2ex, y=1.2ex, baseline=-0.05ex]{%
        \begin{scope}[x=1ex, y=1ex]
            \clip (-0.1,-0.1)
                --++ (-0, 1.2)
                --++ (0.6, 0)
                --++ (0, -0.6)
                --++ (0.6, 0)
                --++ (0, -1);
            \path[draw,
                line width = 0.5,
                rounded corners=0.5]
                (0,0) rectangle (1,1);
        \end{scope}
        \path[draw, line width = 0.5] (0.5, 0.5)
            -- (1, 1);
        \path[draw, line width = 0.5] (0.6, 1)
            -- (1, 1) -- (1, 0.6);
        }
    }

% Other commands
\newcommand\tab[1][0.5cm]{\hspace*{#1}} % for code boxes


% Read Git Data from Github Actions Workflow
% Defaults to gitinfo2 for local builds
\IfFileExists{gitInfo.txt}
	{\input{gitInfo.txt}}
	{
		\newcommand{\gitRelease}{(Local Release)}
		\newcommand{\gitSHA}{\gitHash}
		\newcommand{\gitDate}{\gitAuthorIsoDate}
	}

% Custom Titlepage
\defbeamertemplate*{title page}{customized}[1][]
{
  \vspace{-0cm}\hfill\includegraphics[width=2.5cm]{../config/logo_cif}
  \includegraphics[width=1.9cm]{../config/seal_wwz}
  \\ \vspace{2em}
  \usebeamerfont{title}\textbf{\inserttitle}\par
  \usebeamerfont{title}\usebeamercolor[fg]{title}\insertsubtitle\par  \vspace{1.5em}
  \small\usebeamerfont{author}\insertauthor\par
  \usebeamerfont{author}\insertinstitute\par \vspace{2em}
  \usebeamercolor[fg]{titlegraphic}\inserttitlegraphic
    \tiny \noindent \texttt{Release Ver.: \gitRelease}\\ 
    \texttt{Version Hash: \gitSHA}\\
    \texttt{Version Date: \gitDate}\\ \vspace{1em}
    
    
    \iffalse
  \link \href{https://github.com/cifunibas/Bitcoin-Blockchain-Cryptoassets/blob/main/slides/intro.pdf}
  {Get most recent version}\\
  \link \href{https://github.com/cifunibas/Bitcoin-Blockchain-Cryptoassets/blob/main/slides/intro.pdf}
  {Watch video lecture}\\ 
  
  \fi
  
  \vspace{1em}
  License: \texttt{Creative Commons Attribution-NonCommercial-ShareAlike 4.0 International}\\\vspace{2em}
  \includegraphics[width = 1.2cm]{../config/license}
}


% tikzlibraries
\usetikzlibrary{decorations.pathreplacing}
\usetikzlibrary{decorations.markings}
\usetikzlibrary{positioning}
\usetikzlibrary{calc}
\captionsetup{font=footnotesize}

%%%%%%%%%%%%%%%%%%%%%%%%%%%%%%%%%%%%%%%%%%%%%%
%%%%%%%%%%%%%%%%%%%%%%%%%%%%%%%%%%%%%%%%%%%%%%
\begin{document}

\thispagestyle{empty}
\begin{frame}[noframenumbering]
	\titlepage
\end{frame}

%%%
\begin{frame}{(Non-) Fungibility}

So far, we have only considered fungible tokens. However, tokens can also represent ownership of non-fungible assets.\\

\vspace{1em}

\uncover<2->{
	\textbf{Fungibility:} Assets that are \textit{interchangeable}, e.g., a \$1 USD bill or shares in a company.\\
}

\vspace{0.5em}

\uncover<3->{
	\textbf{Non-fungibility:} Assets that are \textit{distinct}, e.g., real estate or art.
}
\end{frame}
%%%

%%%
\begin{frame}{Some Examples}

Some examples of non-fungible token applications (non-exhaustive list):

\begin{itemize}
	\item<2-> Art
	\item<3-> Collectibles
	\item<4-> Domain names
	\item<5-> Land parcels
	\item<6-> Financial assets, e.g., positions in liquidity pools
	\item<7-> In-game assets
	\item<8-> Tickets
	\item<8-> ...
\end{itemize}

\uncover<8->{
\begin{keytakeaway}{External promises and counterparty risk}
	Once again: any external promises are s.t. counterparty issuer risk. The value of the token will depend on the market's expectation regarding the issuer's willingness and ability to fulfill their promise.
\end{keytakeaway}
}
\end{frame}
%%%

%%%
\begin{frame}{ERC721 Token Standard: Functions}
	A \textbf{standardized function interface} allows smart contracts to interact with any ERC721 token:
\texttt{\scriptsize
\vspace{0.5em}
\begin{itemize}
	\item<2->	function balanceOf(address $\_$owner) \textcolor{red}{external} view returns (uint256)
	\item<3->	\textcolor{red}{function ownerOf(uint256 $\_$tokenId) external view returns (address)}
	\item<4->	\textcolor{red}{function safeTransferFrom(address $\_$from, address $\_$to, uint256 $\_$tokenId, bytes data) external payable}
	\item<5->	\textcolor{red}{function safeTransferFrom(address $\_$from, address $\_$to, uint256 $\_$tokenId) external payable}
	\item<6->	function transferFrom(address $\_$from, address $\_$to, uint256 \textcolor{red}{$\_$tokenId}) \textcolor{red}{external payable}
	\item<7->	function approve(address $\_$approved, uint256 \textcolor{red}{$\_$tokenId}) \textcolor{red}{external payable}
	\item<8->	\textcolor{red}{function setApprovalForAll(address $\_$operator, bool $\_$approved) external}
	\item<9->	\textcolor{red}{function getApproved(uint256 $\_$tokenId) external view returns (address)}
	\item<10->	\textcolor{red}{function isApprovedForAll(address $\_$owner, address $\_$operator) external view returns (bool)}
\end{itemize}
}
\end{frame}
%%%

%%%
\begin{frame}{ERC721 Token Standard: Events}
\vspace{0.5em}
A \textbf{standardized event set} allows frontends and (off-chain) applications to be developed in a more generic way:
\texttt{\scriptsize
\vspace{0.5em}
\begin{itemize}
	\item<2->	event Transfer(address indexed $\_$from, address indexed $\_$to, uint256 \textcolor{red}{indexed $\_$tokenId})
	\item<3->	event Approval(address indexed $\_$owner, address indexed $\_$approved, uint256 \textcolor{red}{indexed $\_$tokenId})
	\item<4->	\textcolor{red}{event ApprovalForAll(address indexed $\_$owner, address indexed $\_$operator, bool $\_$approved)}
\end{itemize}
}
\end{frame}
%%%

%%%
\begin{frame}{ERC165 Interface}
	\begin{samplecode}{Query if a contract implements an interface}
		\begin{lstlisting}[language=Solidity]
	interface ERC165 {
	function supportsInterface(bytes4 interfaceID) external view returns (bool);
}
\end{lstlisting}	
	\end{samplecode}

\uncover<2->{
\textbf{Example:} The identifier for this interface is \texttt{0x01ffc9a7}. You can check this by running \texttt{bytes4(keccak256('supportsInterface(bytes4)'));}.\\
\vspace{0.5em}
}
\uncover<3->{
\texttt{supportsInterface} returns \texttt{true} if the \texttt{interfaceID} is \texttt{0x01ffc9a7} or for any other \texttt{interfaceID} the contract implements. It returns \texttt{false} if \texttt{interfaceID} is \texttt{0xffffffff} or for any other \texttt{interfaceID} that is not implemented in the contract.
}
\end{frame}
%%%

%%%
\begin{frame}{ERC721 Token Standard: Wallet Interface}
A wallet/broker/auction application \textbf{MUST} implement the wallet interface if it will accept safe transfers:
\begin{samplecode}{Wallet Interface}
		\input{../assets/solidity_code/erc721-wallet-interface.tex}
\end{samplecode}
\end{frame}
%%%

%%%
\begin{frame}{Metadata Extension (Optional)}
\begin{samplecode}{Metadata Extension}
		\begin{lstlisting}[language=Solidity]
interface ERC721Metadata {
    function name() external view returns (string _name);
    function symbol() external view returns (string _symbol);
    function tokenURI(uint256 _tokenId) external view returns (string);
}
\end{lstlisting}
\end{samplecode}

\vspace{0.5em}

\uncover<2->{
	\begin{keytakeaway}{Mutability of the URI}
		The URI may be mutable, and thus, privileged users may change the referenced metadata of an NFT. \href{https://twitter.com/neitherconfirm/status/1369285946198396928}{\link Twitter thread about an illustrative case in practice.}
	\end{keytakeaway}
}
\end{frame}
%%%

%%%
\begin{frame}{Metadata JSON Schema Example}
\begin{samplecode}{ERC721 Metadata JSON Schema}
		\input{../assets/other_code/erc721-metadata-json-schema.tex}
\end{samplecode}
\end{frame}
%%%

%%%
\begin{frame}{On-chain vs. Off-chain Data}

The \texttt{tokenURI} may point to the metadata of an NFT, but the metadata is not necessarily secured by the blockchain. Examples:\\

	\begin{itemize}
		\item<2-> On-chain generated %, e.g., Autoglyphs % On-chain generated via draw function. When users createGlyph, the _mint function is called which checks if PRICE is paid, etc. If all requirements are met, the contract draws the glyph and returns the URI. If the tokenURI is called, the contract simply returns the output of the draw function.
		\item<3-> On-chain verifiable hash %, e.g., Cryptopunks % Off-chain picture (https://www.larvalabs.com/public/images/cryptopunks/punks.png) which hashes to ac39af4793119ee46bbff351d8cb6b5f23da60222126add4268e261199a2921b can be verified in the smart contract. Also, Cryptopunks are now fully on-chain: https://etherscan.io/address/0x16f5a35647d6f03d5d3da7b35409d65ba03af3b2. Saving 24x24 pixel pictures is do-able on-chain.
		\item<4-> Off-chain stored metadata on a decentralized file system %, e.g., Bored Apes Yacht Club % Metadata in JSON format on IPFS (ipfs://QmeSjSinHpPnmXmspMjwiXyN6zS4E9zccariGR3jxcaWtq/1) incl. link to image (ipfs://QmPbxeGcXhYQQNgsC6a36dDyYUcHgMLnGKnF8pVFmGsvqi). baseURI can be updated via setBaseURI, but requires the user to be the smart contract owner (onlyOwner). owner address is zero address.
		\item<5-> Off-chain stored metadata on a centralized file system %, e.g., Cryptokitties % JSON and image centralized on Dapper Labs website, e.g., (https://api.cryptokitties.co/kitties/2252) and (https://img.cn.cryptokitties.co/0x06012c8cf97bead5deae237070f9587f8e7a266d/2252.svg)
	\end{itemize}
\end{frame}
%%%

%%%
\begin{frame}{Enumeration Extension}
\begin{samplecode}{Enumeration Extension}
		\begin{lstlisting}[language=Solidity]
interface ERC721Enumerable {
    function totalSupply() external view returns (uint256);
    function tokenByIndex(uint256 _index) external view returns (uint256);
    function tokenOfOwnerByIndex(address _owner, uint256 _index) external view returns (uint256);
}
\end{lstlisting}
\end{samplecode}

\vspace{1 em}
\link \href{https://eips.ethereum.org/EIPS/eip-721}{EIP-721: Non-Fungible Token Standard}
\end{frame}
%%%

%%%
%\begin{frame}%[allowframebreaks]
%\frametitle{References and Recommended Reading}
%	\bibliographystyle{amsplain}
%	\bibliography{../assets/bib/refs}
%\end{frame}
%%%

\end{document}