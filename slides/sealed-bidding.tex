% Choose one to switch between slides and handout
\documentclass[]{beamer}
%\documentclass[handout]{beamer}

% Video Meta Data
\title{Smart Contracts and Decentralized Finance}
\subtitle{Sealed Bidding}
\author{Prof. Dr. Fabian Sch\"ar}
\institute{University of Basel}

% Config File
% Packages
\usepackage[utf8]{inputenc}
\usepackage{hyperref}
\usepackage{gitinfo2}
\usepackage{tikz}
 \usetikzlibrary{calc}
\usepackage{amsmath}
\usepackage{mathtools}
\usepackage{bibentry}
\usepackage{xcolor}
\usepackage{colortbl} % Add colour to LaTeX tables
\usepackage{caption}
\usepackage[export]{adjustbox}
\usepackage{pgfplots} \pgfplotsset{compat = 1.17}
\usepackage{makecell}
\usepackage{fancybox}
\usepackage{ragged2e}
\usepackage{fontawesome}
\usepackage{seqsplit}
\usepackage{tabularx}
\usepackage{tcolorbox}
\usepackage{booktabs} % use instead  \hline in tables

% Color Options
\definecolor{highlight}{rgb}{0.65,0.84,0.82}
\definecolor{focus}{rgb}{0.72, 0, 0}
\definecolor{lightred}{rgb}{0.8,0.5,0.5}
\definecolor{midgray}{RGB}{190,195,200}

 %UniBas Main Colors
\definecolor{mint}{RGB}{165,215,210}
\definecolor{anthracite}{RGB}{45,55,60}
\definecolor{red}{RGB}{210,5,55}

 %UniBas Color Palette (for graphics)
\definecolor{strongmint}{RGB}{30,165,165}
\definecolor{darkmint}{RGB}{0,110,110}
\definecolor{softanthracite}{RGB}{140,145,150}
\definecolor{brightanthracite}{RGB}{190,195,200}
\definecolor{softred}{RGB}{235,130,155}

%Custom Colors
\definecolor{lightergray}{RGB}{230, 230, 230}



% Beamer Template Options
\beamertemplatenavigationsymbolsempty
\setbeamertemplate{footline}[frame number]
\setbeamercolor{structure}{fg=black}
\setbeamercolor{footline}{fg=black}
\setbeamercolor{title}{fg=black}
\setbeamercolor{frametitle}{fg=black}
\setbeamercolor{item}{fg=black}
\setbeamercolor{}{fg=black}
\setbeamercolor{bibliography item}{fg=black}
\setbeamercolor*{bibliography entry title}{fg=black}
\setbeamercolor{alerted text}{fg=focus}
\setbeamertemplate{items}[square]
\setbeamertemplate{enumerate items}[default]
\captionsetup[figure]{labelfont={color=black},font={color=black}}
\captionsetup[table]{labelfont={color=black},font={color=black}}

\setbeamertemplate{bibliography item}{\insertbiblabel}

%tcolor boxes
\newtcolorbox{samplecode}[2][]{
  colback=mint, colframe=darkmint, coltitle=white,
  fontupper = \ttfamily\scriptsize, fonttitle= \bfseries\scriptsize,
  boxrule = 0mm, arc = 0mm,
  boxsep = 1.3mm, left = 0mm, right = 0mm, top = 0.5mm, bottom = 0mm, middle=0mm,
  #1,title=#2}
  
\newtcolorbox{keytakeaway}[2][]{
  colback=softred, colframe=red, coltitle=white,
  fontupper = \scriptsize, fonttitle= \bfseries\scriptsize,
  boxrule = 0mm, arc = 0mm,
  boxsep = 1.3mm, left = 0mm, right = 0mm, top = 0.5mm, bottom = 0mm, middle=0mm,
  #1,title=#2}

\newtcolorbox{exercise}[2][]{
  colback=brightanthracite, colframe=anthracite, coltitle=white,
  fontupper = \scriptsize, fonttitle= \bfseries\scriptsize,
  boxrule = 0mm, arc = 0mm,
  boxsep = 1.3mm, left = 0mm, right = 0mm, top = 0.5mm, bottom = 0mm, middle=0mm,
  #1,title=#2}



% Link Icon Command 
\newcommand{\link}{%
    \tikz[x=1.2ex, y=1.2ex, baseline=-0.05ex]{%
        \begin{scope}[x=1ex, y=1ex]
            \clip (-0.1,-0.1)
                --++ (-0, 1.2)
                --++ (0.6, 0)
                --++ (0, -0.6)
                --++ (0.6, 0)
                --++ (0, -1);
            \path[draw,
                line width = 0.5,
                rounded corners=0.5]
                (0,0) rectangle (1,1);
        \end{scope}
        \path[draw, line width = 0.5] (0.5, 0.5)
            -- (1, 1);
        \path[draw, line width = 0.5] (0.6, 1)
            -- (1, 1) -- (1, 0.6);
        }
    }

% Other commands
\newcommand\tab[1][0.5cm]{\hspace*{#1}} % for code boxes


% Read Git Data from Github Actions Workflow
% Defaults to gitinfo2 for local builds
\IfFileExists{gitInfo.txt}
	{\input{gitInfo.txt}}
	{
		\newcommand{\gitRelease}{(Local Release)}
		\newcommand{\gitSHA}{\gitHash}
		\newcommand{\gitDate}{\gitAuthorIsoDate}
	}

% Custom Titlepage
\defbeamertemplate*{title page}{customized}[1][]
{
  \vspace{-0cm}\hfill\includegraphics[width=2.5cm]{../config/logo_cif}
  \includegraphics[width=1.9cm]{../config/seal_wwz}
  \\ \vspace{2em}
  \usebeamerfont{title}\textbf{\inserttitle}\par
  \usebeamerfont{title}\usebeamercolor[fg]{title}\insertsubtitle\par  \vspace{1.5em}
  \small\usebeamerfont{author}\insertauthor\par
  \usebeamerfont{author}\insertinstitute\par \vspace{2em}
  \usebeamercolor[fg]{titlegraphic}\inserttitlegraphic
    \tiny \noindent \texttt{Release Ver.: \gitRelease}\\ 
    \texttt{Version Hash: \gitSHA}\\
    \texttt{Version Date: \gitDate}\\ \vspace{1em}
    
    
    \iffalse
  \link \href{https://github.com/cifunibas/Bitcoin-Blockchain-Cryptoassets/blob/main/slides/intro.pdf}
  {Get most recent version}\\
  \link \href{https://github.com/cifunibas/Bitcoin-Blockchain-Cryptoassets/blob/main/slides/intro.pdf}
  {Watch video lecture}\\ 
  
  \fi
  
  \vspace{1em}
  License: \texttt{Creative Commons Attribution-NonCommercial-ShareAlike 4.0 International}\\\vspace{2em}
  \includegraphics[width = 1.2cm]{../config/license}
}


% tikzlibraries
\usetikzlibrary{decorations.pathreplacing}
\usetikzlibrary{decorations.markings}
\usetikzlibrary{positioning}
\usetikzlibrary{calc}
\captionsetup{font=footnotesize}

%%%%%%%%%%%%%%%%%%%%%%%%%%%%%%%%%%%%%%%%%%%%%%
%%%%%%%%%%%%%%%%%%%%%%%%%%%%%%%%%%%%%%%%%%%%%%
\begin{document}

\thispagestyle{empty}
\begin{frame}[noframenumbering]
	\titlepage
\end{frame}

%%%
\begin{frame}{Submitting a bid}
\uncover<1->{
	Input:
	\begin{itemize}
		\item A hashed sealed bid
	\end{itemize}
\vspace{.3cm}
}
\uncover<2->{
	Tasks:
	\begin{itemize}
		\item Create a new Bid object
		\item Add the new bid to the sender's list of bids
	\end{itemize}
}
\vspace{.3cm}
\uncover<3->{
	\begin{samplecode}{bid()}
		\begin{lstlisting}[language=Solidity]
function bid(bytes32 sealedBid) external payable {
	Bid memory newBid = Bid({
		sealedBid: sealedBid,
	deposit: msg.value
	});
	
	bids[msg.sender].push(newBid);
}
\end{lstlisting}
	\end{samplecode}
}

\end{frame}
%%%	

%%%
\begin{frame}{Data Location}
	\begin{itemize}
		\item Notice when creating a new Bid object we need to specify a location for the object.
		\item<2-> All structs and arrays (including strings) can exist in three different locations:
			\begin{itemize}
			\item<3-> \texttt{memory}: The object is not written or read from the blockchain and only exists in the current scope.
			\item<4-> \texttt{calldata}: Similar to memory, but can only be used for function arguments in external calls and it is non-modifiable.
			\item<5-> \texttt{storage}: Loaded from or written onto the blockchain. These are expensive operations, avoid whenever possible.
			\end{itemize}

		\item<6-> Good heuristic: Use \texttt{storage} if you want to load or modify a state variable. Use \texttt{memory} otherwise (\texttt{calldata} for gas optimization).
		\end{itemize}
\end{frame}
%%%

%%%
\begin{frame}{The reveal mechanic}
	Let's think about what the reveal function needs to do:
	\begin{itemize}
		\item<2-> Accept a list of unencrypted bids to reveal.
		\item<3-> Validate that the length of this list corresponds to the number of committed bids.
		\item<4-> Check if the reveal period is active.
		\item<5-> Compute hash values and compare them to commit
		\item<6-> Ignore fake bids.
		\item<7-> Update the highest bid and highest bidder if applicable.
		\item<8-> Handle refunds related to reveal mechanism.
	\end{itemize}
\end{frame}
%%%

%%%
\begin{frame}{Split the function}
For better readability, we split the logic into two parts:
	\begin{itemize}
		\item<2-> \texttt{updateBid()} handles the bid related checks and refunds.
		\item<3-> \texttt{reveal()} handles all the reveal related checks and refunds.
	\end{itemize}
\end{frame}
%%%

%%
\begin{frame}{Update Bid}
\texttt{updateBids()} is similar to the simple auction's bid function:
	\begin{itemize}
		\item<2-> The function is internal, meaning it can only be called from the contract itself.
		\item<3-> It returns \texttt{true} if the new highest bid was accepted and \texttt{false} otherwise.
	\end{itemize} 	
\end{frame}
%%%

%%%
\begin{frame}{Update Bid - Sample code}
	\begin{samplecode}{updateBid()}
		\begin{lstlisting}[language=Solidity]
function updateBid(address _bidder, uint _bidAmount) internal returns (bool success) {
	if (_bidAmount <= highestBid) {
		return false;
	}
	if (highestBidder != address(0)) {
		// Refund the previously highest bidder.
		pendingReturns[highestBidder] += highestBid;
	}
	highestBid = _bidAmount;
	highestBidder = _bidder;
	return true;
}
\end{lstlisting}
	\end{samplecode}
\end{frame}
%%%

%%%
\begin{frame}{Simple Reveal}
Start with a simple version: Only one bid exists:
\vspace{.3cm}
	\begin{samplecode}{reveal() - Part I}
		\begin{lstlisting}[language=Solidity]
function reveal(uint bidAmount, bool isLegit, string calldata salt) external {
    bytes32 hashedInput = generateBlindedBid(bidAmount, isLegit, salt);
    Bid storage bidToCheck = bids[msg.sender][0]; // Load the first element of the array
    uint refund;

    if (bidToCheck.blindedBid == hashedInput) {
        // Bid is successfully revealed
        refund = bidToCheck.deposit;
\end{lstlisting}
	\end{samplecode}
\end{frame}
%%%

%%%
\begin{frame}{Simple Reveal}
\vspace{.3cm}
	\begin{samplecode}{reveal() - Part II}
		\begin{lstlisting}[language=Solidity]
		if (_isLegit && bidToCheck.deposit >= _bidAmount) {
			// Bid is valid
			bool success = updateBid(msg.sender, _bidAmount);
			if(success) {
				// Bid is new highest bid
				refund -= _bidAmount;
			}
		}
		// Prevent re-claiming the same deposit
		bidToCheck.sealedBid = bytes32(0);
	}
	if (refund > 0) {
		payable(msg.sender).transfer(refund);
	}
}
\end{lstlisting}
	\end{samplecode}
\end{frame}
%%%

%%%
\begin{frame}{Loops}
	\begin{itemize}
		\item<1->We need to iterate over all bids of a user
		\item<2->Solidity supports most of the control structures known from similar languages such as JavaScript with the usual semantics.
		\item<3->\texttt{for}-loops are typically used to iterate over arrays
		\item<5->\texttt{continue} will jump to the beginning of the next iteration. \texttt{break} will end the loop.
		\item<6->In addition to \texttt{for}-loops, \texttt{do} and \texttt{while} loops are also available.
	\end{itemize}

	\uncover<4->{
		\begin{samplecode}{for()}
			\begin{lstlisting}[language=Solidity]
	// T[]  array;
	for (uint i = 0; i < array.length; i++) {
		T  element = array[i];
	}
\end{lstlisting}
		\end{samplecode}
	}
	
\end{frame}
%%%

%%%
\begin{frame}{Custom Modifiers}
	\begin{itemize}
		\item<1-> The only thing remaining is to end the auction and to implement time constraints.
		\item<2-> We could do the time constraints similar to the simple auction contract.
		\item<3-> However, if we have the same repeated \texttt{require} checks at the start (or end) of functions we can make use of custom modifiers.
		\item<4-> Custom modifiers are a convenient, reusable way to validate inputs to functions.
	\end{itemize}
\end{frame}
%%%

%%%
\begin{frame}{Custom Modifiers - Code Samples}
	\uncover<1->{
	\begin{samplecode}{Only Before}
		\begin{lstlisting}[language=Solidity]
modifier onlyBefore(uint time) {
	require(block.timestamp < time, 'too late');
	_;
}
\end{lstlisting}
	\end{samplecode}
	}
	\uncover<2->{
	\begin{samplecode}{Only After}
		\begin{lstlisting}[language=Solidity]
modifier onlyAfter(uint time) {
	require(block.timestamp > time, 'too early');
	_;
}
\end{lstlisting}
	\end{samplecode}
	}
\end{frame}
%%%

%%%
\begin{frame}{Finish the Contract}
	\uncover<1->{Add a function to end the auction with the modifier onlyAfter:}
		\vspace{.5cm}
	\uncover<2->{
		\begin{samplecode}{Only After}
			\begin{lstlisting}[language=Solidity]
function auctionEnd() external onlyAfter(revealEnd) {
  require(!hasEnded, 'Auction already ended');
  emit AuctionEnded(highestBidder, highestBid);
  hasEnded = true;
  payable(beneficiary).transfer(highestBid);
}
\end{lstlisting}
		\end{samplecode}
	}

\end{frame}
%%%

%%%
\begin{frame}{Exercise: Update Your SealedBidAuction Contract}
		\begin{exercise}{Exercise 1}
			\textbf{a) Only one bid exists}
			\begin{itemize}
				\item Add a function to submit a new bid to your \texttt{SealedBidAuction} contract
				\item Implement a reveal function in your \texttt{SealedBidAuction} contract
				\item Add time constraints to the \texttt{bid()}, \texttt{reveal()} and \texttt{auctionEnd()} functions
			\end{itemize}
			\vspace{0.5cm}
			\textbf{b) Extended version: allow for multiple bids}
			\begin{itemize}
				\item Extend your contract to handle multiple bids
			\end{itemize}
			\vspace{0.5cm}
			\textbf{Hint:} You can find all the code components needed for Exercise 1. a) on the previous slides. For Exercise 1. b) you need to extend your contract using a loop. \\
		\end{exercise}
\end{frame}
%%%

\end{document}