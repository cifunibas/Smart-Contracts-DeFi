% Choose one to switch between slides and handout
%\documentclass[]{beamer}
\documentclass[handout]{beamer}

% Video Meta Data
\title{Smart Contracts and Decentralized Finance}
\subtitle{Smart Contracts and the EVM}
\author{Prof. Dr. Fabian Schär}
\institute{University of Basel}

% Config File
% Packages
\usepackage[utf8]{inputenc}
\usepackage{hyperref}
\usepackage{gitinfo2}
\usepackage{tikz}
 \usetikzlibrary{calc}
\usepackage{amsmath}
\usepackage{mathtools}
\usepackage{bibentry}
\usepackage{xcolor}
\usepackage{colortbl} % Add colour to LaTeX tables
\usepackage{caption}
\usepackage[export]{adjustbox}
\usepackage{pgfplots} \pgfplotsset{compat = 1.17}
\usepackage{makecell}
\usepackage{fancybox}
\usepackage{ragged2e}
\usepackage{fontawesome}
\usepackage{seqsplit}
\usepackage{tabularx}
\usepackage{tcolorbox}
\usepackage{booktabs} % use instead  \hline in tables

% Color Options
\definecolor{highlight}{rgb}{0.65,0.84,0.82}
\definecolor{focus}{rgb}{0.72, 0, 0}
\definecolor{lightred}{rgb}{0.8,0.5,0.5}
\definecolor{midgray}{RGB}{190,195,200}

 %UniBas Main Colors
\definecolor{mint}{RGB}{165,215,210}
\definecolor{anthracite}{RGB}{45,55,60}
\definecolor{red}{RGB}{210,5,55}

 %UniBas Color Palette (for graphics)
\definecolor{strongmint}{RGB}{30,165,165}
\definecolor{darkmint}{RGB}{0,110,110}
\definecolor{softanthracite}{RGB}{140,145,150}
\definecolor{brightanthracite}{RGB}{190,195,200}
\definecolor{softred}{RGB}{235,130,155}

%Custom Colors
\definecolor{lightergray}{RGB}{230, 230, 230}



% Beamer Template Options
\beamertemplatenavigationsymbolsempty
\setbeamertemplate{footline}[frame number]
\setbeamercolor{structure}{fg=black}
\setbeamercolor{footline}{fg=black}
\setbeamercolor{title}{fg=black}
\setbeamercolor{frametitle}{fg=black}
\setbeamercolor{item}{fg=black}
\setbeamercolor{}{fg=black}
\setbeamercolor{bibliography item}{fg=black}
\setbeamercolor*{bibliography entry title}{fg=black}
\setbeamercolor{alerted text}{fg=focus}
\setbeamertemplate{items}[square]
\setbeamertemplate{enumerate items}[default]
\captionsetup[figure]{labelfont={color=black},font={color=black}}
\captionsetup[table]{labelfont={color=black},font={color=black}}

\setbeamertemplate{bibliography item}{\insertbiblabel}

%tcolor boxes
\newtcolorbox{samplecode}[2][]{
  colback=mint, colframe=darkmint, coltitle=white,
  fontupper = \ttfamily\scriptsize, fonttitle= \bfseries\scriptsize,
  boxrule = 0mm, arc = 0mm,
  boxsep = 1.3mm, left = 0mm, right = 0mm, top = 0.5mm, bottom = 0mm, middle=0mm,
  #1,title=#2}
  
\newtcolorbox{keytakeaway}[2][]{
  colback=softred, colframe=red, coltitle=white,
  fontupper = \scriptsize, fonttitle= \bfseries\scriptsize,
  boxrule = 0mm, arc = 0mm,
  boxsep = 1.3mm, left = 0mm, right = 0mm, top = 0.5mm, bottom = 0mm, middle=0mm,
  #1,title=#2}

\newtcolorbox{exercise}[2][]{
  colback=brightanthracite, colframe=anthracite, coltitle=white,
  fontupper = \scriptsize, fonttitle= \bfseries\scriptsize,
  boxrule = 0mm, arc = 0mm,
  boxsep = 1.3mm, left = 0mm, right = 0mm, top = 0.5mm, bottom = 0mm, middle=0mm,
  #1,title=#2}



% Link Icon Command 
\newcommand{\link}{%
    \tikz[x=1.2ex, y=1.2ex, baseline=-0.05ex]{%
        \begin{scope}[x=1ex, y=1ex]
            \clip (-0.1,-0.1)
                --++ (-0, 1.2)
                --++ (0.6, 0)
                --++ (0, -0.6)
                --++ (0.6, 0)
                --++ (0, -1);
            \path[draw,
                line width = 0.5,
                rounded corners=0.5]
                (0,0) rectangle (1,1);
        \end{scope}
        \path[draw, line width = 0.5] (0.5, 0.5)
            -- (1, 1);
        \path[draw, line width = 0.5] (0.6, 1)
            -- (1, 1) -- (1, 0.6);
        }
    }

% Other commands
\newcommand\tab[1][0.5cm]{\hspace*{#1}} % for code boxes


% Read Git Data from Github Actions Workflow
% Defaults to gitinfo2 for local builds
\IfFileExists{gitInfo.txt}
	{\input{gitInfo.txt}}
	{
		\newcommand{\gitRelease}{(Local Release)}
		\newcommand{\gitSHA}{\gitHash}
		\newcommand{\gitDate}{\gitAuthorIsoDate}
	}

% Custom Titlepage
\defbeamertemplate*{title page}{customized}[1][]
{
  \vspace{-0cm}\hfill\includegraphics[width=2.5cm]{../config/logo_cif}
  \includegraphics[width=1.9cm]{../config/seal_wwz}
  \\ \vspace{2em}
  \usebeamerfont{title}\textbf{\inserttitle}\par
  \usebeamerfont{title}\usebeamercolor[fg]{title}\insertsubtitle\par  \vspace{1.5em}
  \small\usebeamerfont{author}\insertauthor\par
  \usebeamerfont{author}\insertinstitute\par \vspace{2em}
  \usebeamercolor[fg]{titlegraphic}\inserttitlegraphic
    \tiny \noindent \texttt{Release Ver.: \gitRelease}\\ 
    \texttt{Version Hash: \gitSHA}\\
    \texttt{Version Date: \gitDate}\\ \vspace{1em}
    
    
    \iffalse
  \link \href{https://github.com/cifunibas/Bitcoin-Blockchain-Cryptoassets/blob/main/slides/intro.pdf}
  {Get most recent version}\\
  \link \href{https://github.com/cifunibas/Bitcoin-Blockchain-Cryptoassets/blob/main/slides/intro.pdf}
  {Watch video lecture}\\ 
  
  \fi
  
  \vspace{1em}
  License: \texttt{Creative Commons Attribution-NonCommercial-ShareAlike 4.0 International}\\\vspace{2em}
  \includegraphics[width = 1.2cm]{../config/license}
}


% tikzlibraries
\usetikzlibrary{decorations.pathreplacing}
\usetikzlibrary{decorations.markings}
\usetikzlibrary{positioning}
\usetikzlibrary{calc}
\captionsetup{font=footnotesize}

\captionsetup[figure]{font=tiny,labelformat=simple}


%%%%%%%%%%%%%%%%%%%%%%%%%%%%%%%%%%%%%%%%%%%%%%
%%%%%%%%%%%%%%%%%%%%%%%%%%%%%%%%%%%%%%%%%%%%%%
\begin{document}

\thispagestyle{empty}
\begin{frame}[noframenumbering]
	\titlepage
\end{frame}
%%%

\definecolor{mypink}{RGB}{245, 204, 213}
\definecolor{myred}{RGB}{209, 7, 54}

%%%
\begin{frame}{The Ethereum Virtual Machine (EVM)}
	\vspace{1em}
\begin{minipage}{0.49\textwidth}
		\begin{tikzpicture}[scale=0.35, every node/.style={scale=0.35}]
		%Set Anchor Points
		
\coordinate (1) at (0,0);
\coordinate (2) at (2.5, 1.5);
\coordinate (3) at (-2.5,1.5);
\coordinate (4) at (0,3);
\coordinate (5) at (2.5,-1.5);
\coordinate (6) at (-2.5, -1.5);
\coordinate (7) at (0, -3);

\node(node1)[minimum size = 1.3cm] at (1) {};
\node(node2)[minimum size = 1.3cm] at (2) {};
\node(node3)[minimum size = 1.3cm] at (3) {};
\node(node4)[minimum size = 1.3cm] at (4) {};
\node(node5)[minimum size = 1.3cm] at (5) {};
\node(node6)[minimum size = 1.3cm] at (6) {};
\node(node7)[minimum size = 1.3cm] at (7) {};


%Draw Lines
\draw [line width=0.8mm ] (1) -- (2) -- (4) -- (3) -- (1) -- (5) -- (7) --(6)-- (1) -- (4)--(6)--(5)-- (4)--(2)--(5)--(1)--(3)--(6)--(7)--(1)--cycle;

%Draw Nodes

\filldraw[fill=white, line width=0.8mm](1) circle (.4);

\filldraw[fill=white, line width=0.8mm](2) circle (.4)
node[above right=8pt] [xshift=4pt,yshift=-34pt] {
  \begin{tikzpicture}[scale=0.6, every node/.style={scale=0.35}, anchor=center] 
	%Set Cordinate Points and Node
\coordinate (a) at (0,0);
\node(nodea)[minimum size = 1.3cm] at (a) {};

%Include Graphics
\node(a)  {\includegraphics{../assets/images/computer.png }};
\node(a)[above= 0pt] [xshift=-4pt](1)  {\includegraphics{../assets/images/ethereum_bw.png }};
		
  \end{tikzpicture}
  };

%{\includegraphics{../assets/images/computer.jpg }};


\filldraw[fill=white, line width=0.8mm](3) circle (.4)
node[above left=8pt][xshift=-4pt,yshift=-34pt] {
  \begin{tikzpicture}[scale=0.6, every node/.style={scale=0.35}, anchor=center] 
	%Set Cordinate Points and Node
\coordinate (a) at (0,0);
\node(nodea)[minimum size = 1.3cm] at (a) {};

%Include Graphics
\node(a)  {\includegraphics{../assets/images/computer.png }};
\node(a)[above= 0pt] [xshift=-4pt](1)  {\includegraphics{../assets/images/ethereum_bw.png }};
		
  \end{tikzpicture}
  };

\filldraw[fill=white, line width=0.8mm](4) circle (.4)
node[above=2pt] {
  \begin{tikzpicture}[scale=0.6, every node/.style={scale=0.35}, anchor=center] 
	%Set Cordinate Points and Node
\coordinate (a) at (0,0);
\node(nodea)[minimum size = 1.3cm] at (a) {};

%Include Graphics
\node(a)  {\includegraphics{../assets/images/computer.png }};
\node(a)[above= 0pt] [xshift=-4pt](1)  {\includegraphics{../assets/images/ethereum_bw.png }};
		
  \end{tikzpicture}
  };
\filldraw[fill=white, line width=0.8mm](5) circle (.4)
node[below right=8pt] [xshift=4pt,yshift=40pt] {
  \begin{tikzpicture}[scale=0.6, every node/.style={scale=0.35}, anchor=center] 
	%Set Cordinate Points and Node
\coordinate (a) at (0,0);
\node(nodea)[minimum size = 1.3cm] at (a) {};

%Include Graphics
\node(a)  {\includegraphics{../assets/images/computer.png }};
\node(a)[above= 0pt] [xshift=-4pt](1)  {\includegraphics{../assets/images/ethereum_bw.png }};
		
  \end{tikzpicture}
  };

\filldraw[fill=white, line width=0.8mm](6) circle (.4)
node[below left=8pt] [xshift=-4pt,yshift=40pt] {
  \begin{tikzpicture}[scale=0.6, every node/.style={scale=0.35}, anchor=center] 
	%Set Cordinate Points and Node
\coordinate (a) at (0,0);
\node(nodea)[minimum size = 1.3cm] at (a) {};

%Include Graphics
\node(a)  {\includegraphics{../assets/images/computer.png }};
\node(a)[above= 0pt] [xshift=-4pt](1)  {\includegraphics{../assets/images/ethereum_bw.png }};
		
  \end{tikzpicture}
  };

\filldraw[fill=white, line width=0.8mm](7) circle (.4)
node[below=6pt] {
  \begin{tikzpicture}[scale=0.6, every node/.style={scale=0.35}, anchor=center] 
	%Set Cordinate Points and Node
\coordinate (a) at (0,0);
\node(nodea)[minimum size = 1.3cm] at (a) {};

%Include Graphics
\node(a)  {\includegraphics{../assets/images/computer.png }};
\node(a)[above= 0pt] [xshift=-4pt](1)  {\includegraphics{../assets/images/ethereum_bw.png }};
		
  \end{tikzpicture}
  };
	\end{tikzpicture}
	\end{minipage}
	\begin{minipage}{0.46\textwidth}
	\begin{small}
		\begin{itemize}
		
			\item<1-> Runs on every (full) node
			\item<2-> Processes transactions and performs state changes (deterministically)
			\item<3-> Is Turing complete
			\item<4-> State changes as part of consensus; everyone performing all computations
		\end{itemize}
		\end{small}
	\end{minipage}	
			\vspace{1em}
			
			
\uncover<5->{\begin{keytakeaway}{A Slow State Machine}
				The Ethereum Virtual Machine is often referred to as a World computer. It is a relatively slow computer network. However, the strength of the network is that any code executed on the EVM will be executed exactly as specified and state changes can be tracked and verified by any network participant.
\end{keytakeaway}}			
			
	
\end{frame}
%%%	

%%%
\begin{frame}{The EVM and State Changes}
	\vspace{1em}
	\vspace{1em}

		\begin{tikzpicture}[scale=0.5, every node/.style={scale=0.5}]
	
%Define Coordinates

\coordinate (1) at (-8,0);
\coordinate (2) at (-6,0);
\coordinate (3) at (-4,0);
\coordinate (4) at (-2,0);
\coordinate (5) at (0,0);
\coordinate (6) at (2,0);
\coordinate (7) at (4,0);
\coordinate (8) at (6,0);
\coordinate (9) at (8,0);
\coordinate (10) at (10,0);

\coordinate (11) at (-6,-3);
\coordinate (12) at (-2,-3);
\coordinate (13) at (2,-3);
\coordinate (14) at (6,-3);

\coordinate (15) at (10,0);
\coordinate (16) at (10,0);
\coordinate (17) at (10,0);
\coordinate (18) at (10,0);


%Empty Files

\node (state)at(1)  [scale=0.26]{\includegraphics{../assets/images/empty_file.png}};
\node (state)at (3) [scale=0.26] {\includegraphics{../assets/images/empty_file.png}};
\node (state)at (5) [scale=0.26] {\includegraphics{../assets/images/empty_file.png}};
\node (state)at (7) [scale=0.26] {\includegraphics{../assets/images/empty_file.png}};
\node (state)at (9) [scale=0.26] {\includegraphics{../assets/images/empty_file.png}};


%1s and 0 within Empty Files

\node [align=left, yshift= -8pt] at(1){\textbf{10100}\\\textbf{11010}\\\textbf{11101}};
\node [align=left, yshift= -8pt] at(3){\textbf{11101}\\\textbf{11011}\\\textbf{10001}};
\node [align=left, yshift= -8pt] at(5){\textbf{01101}\\\textbf{10010}\\\textbf{10110}};
\node [align=left, yshift= -8pt] at(7){\textbf{11101}\\\textbf{00011}\\\textbf{01001}};
\node [align=left, yshift= -8pt] at(9){\textbf{01111}\\\textbf{10110}\\\textbf{10010}};



%...

\node [align=left] at(10){\textbf{{\Huge ...}}};	


%Arrow Empty File --> Empty File (mintgreen)

\node at(2)[scale=0.08] {\includegraphics{../assets/images/next_mint.png}};
\node at(4)[scale=0.08] {\includegraphics{../assets/images/next_mint.png}};
\node at(6)[scale=0.08] {\includegraphics{../assets/images/next_mint.png}};
\node at(8)[scale=0.08] {\includegraphics{../assets/images/next_mint.png}};
	

%Upper Arrow <->

\draw  [<->, shorten >=15pt, shorten <=15pt, thick] 	(1.east) + (up:2.5cm) 
	coordinate (1_east_shifted) -- 
	(3.west |- 1_east_shifted);
\draw  [<->, shorten >=15pt, shorten <=15pt, thick] 	(3.east) + (up:2.5cm) 
	coordinate (3_east_shifted) -- 
	(5.west |- 3_east_shifted);
\draw  [<->, shorten >=15pt, shorten <=15pt, thick] 	(5.east) + (up:2.5cm) 
	coordinate (5_east_shifted) -- 
	(7.west |- 5_east_shifted);
\draw  [<->, shorten >=15pt, shorten <=15pt, thick] 	(7.east) + (up:2.5cm) 
	coordinate (7_east_shifted) -- 
	(9.west |- 7_east_shifted);
	
%Upper Arrow Label 15 sec	
		
\node [yshift=2.9cm] at (2) {$\approx$ 15 sec};
\node [yshift=2.9cm] at (4) {$\approx$ 15 sec};
\node [yshift=2.9cm] at (6) {$\approx$ 15 sec};
\node [yshift=2.9cm] at (8) {$\approx$ 15 sec};
	%\draw[<->, thick] (1.east)  {} (2.west);		

%States

\node [yshift=1.8cm] at (1) {{\large State $t$}};	
\node [yshift=1.8cm] at (3) {{\large State $t _{\textit{ + 1}}$}};	
\node [yshift=1.8cm] at (5) {{\large State $t _{\textit{ + 2}}$}};
\node [yshift=1.8cm] at (7) {{\large State $t _{\textit{ + 3}}$}};
\node [yshift=1.8cm] at (9) {{\large State $t _{\textit{ + 4}}$}};	


%Network

\node at (11) {
	\scalebox{0.4}{
		\begin{tikzpicture}[scale=0.35, every node/.style={scale=0.35}]
			%Set Anchor Points
		
\coordinate (1) at (0,0);
\coordinate (2) at (2.5, 1.5);
\coordinate (3) at (-2.5,1.5);
\coordinate (4) at (0,3);
\coordinate (5) at (2.5,-1.5);
\coordinate (6) at (-2.5, -1.5);
\coordinate (7) at (0, -3);

\node(node1)[minimum size = 1.3cm] at (1) {};
\node(node2)[minimum size = 1.3cm] at (2) {};
\node(node3)[minimum size = 1.3cm] at (3) {};
\node(node4)[minimum size = 1.3cm] at (4) {};
\node(node5)[minimum size = 1.3cm] at (5) {};
\node(node6)[minimum size = 1.3cm] at (6) {};
\node(node7)[minimum size = 1.3cm] at (7) {};


%Draw Lines
\draw [line width=0.8mm ] (1) -- (2) -- (4) -- (3) -- (1) -- (5) -- (7) --(6)-- (1) -- (4)--(6)--(5)-- (4)--(2)--(5)--(1)--(3)--(6)--(7)--(1)--cycle;

%Draw Nodes

\filldraw[fill=white, line width=0.8mm](1) circle (.4);

\filldraw[fill=white, line width=0.8mm](2) circle (.4)
node[above right=8pt] [xshift=4pt,yshift=-34pt] {
  \begin{tikzpicture}[scale=0.6, every node/.style={scale=0.35}, anchor=center] 
	%Set Cordinate Points and Node
\coordinate (a) at (0,0);
\node(nodea)[minimum size = 1.3cm] at (a) {};

%Include Graphics
\node(a)  {\includegraphics{../assets/images/computer.png }};
\node(a)[above= 0pt] [xshift=-4pt](1)  {\includegraphics{../assets/images/ethereum_bw.png }};
		
  \end{tikzpicture}
  };

%{\includegraphics{../assets/images/computer.jpg }};


\filldraw[fill=white, line width=0.8mm](3) circle (.4)
node[above left=8pt][xshift=-4pt,yshift=-34pt] {
  \begin{tikzpicture}[scale=0.6, every node/.style={scale=0.35}, anchor=center] 
	%Set Cordinate Points and Node
\coordinate (a) at (0,0);
\node(nodea)[minimum size = 1.3cm] at (a) {};

%Include Graphics
\node(a)  {\includegraphics{../assets/images/computer.png }};
\node(a)[above= 0pt] [xshift=-4pt](1)  {\includegraphics{../assets/images/ethereum_bw.png }};
		
  \end{tikzpicture}
  };

\filldraw[fill=white, line width=0.8mm](4) circle (.4)
node[above=2pt] {
  \begin{tikzpicture}[scale=0.6, every node/.style={scale=0.35}, anchor=center] 
	%Set Cordinate Points and Node
\coordinate (a) at (0,0);
\node(nodea)[minimum size = 1.3cm] at (a) {};

%Include Graphics
\node(a)  {\includegraphics{../assets/images/computer.png }};
\node(a)[above= 0pt] [xshift=-4pt](1)  {\includegraphics{../assets/images/ethereum_bw.png }};
		
  \end{tikzpicture}
  };
\filldraw[fill=white, line width=0.8mm](5) circle (.4)
node[below right=8pt] [xshift=4pt,yshift=40pt] {
  \begin{tikzpicture}[scale=0.6, every node/.style={scale=0.35}, anchor=center] 
	%Set Cordinate Points and Node
\coordinate (a) at (0,0);
\node(nodea)[minimum size = 1.3cm] at (a) {};

%Include Graphics
\node(a)  {\includegraphics{../assets/images/computer.png }};
\node(a)[above= 0pt] [xshift=-4pt](1)  {\includegraphics{../assets/images/ethereum_bw.png }};
		
  \end{tikzpicture}
  };

\filldraw[fill=white, line width=0.8mm](6) circle (.4)
node[below left=8pt] [xshift=-4pt,yshift=40pt] {
  \begin{tikzpicture}[scale=0.6, every node/.style={scale=0.35}, anchor=center] 
	%Set Cordinate Points and Node
\coordinate (a) at (0,0);
\node(nodea)[minimum size = 1.3cm] at (a) {};

%Include Graphics
\node(a)  {\includegraphics{../assets/images/computer.png }};
\node(a)[above= 0pt] [xshift=-4pt](1)  {\includegraphics{../assets/images/ethereum_bw.png }};
		
  \end{tikzpicture}
  };

\filldraw[fill=white, line width=0.8mm](7) circle (.4)
node[below=6pt] {
  \begin{tikzpicture}[scale=0.6, every node/.style={scale=0.35}, anchor=center] 
	%Set Cordinate Points and Node
\coordinate (a) at (0,0);
\node(nodea)[minimum size = 1.3cm] at (a) {};

%Include Graphics
\node(a)  {\includegraphics{../assets/images/computer.png }};
\node(a)[above= 0pt] [xshift=-4pt](1)  {\includegraphics{../assets/images/ethereum_bw.png }};
		
  \end{tikzpicture}
  };
		\end{tikzpicture}
		}
};

\node  at (12) {
	\scalebox{0.4}{
		\begin{tikzpicture}[scale=0.35, every node/.style={scale=0.35}]
			%Set Anchor Points
		
\coordinate (1) at (0,0);
\coordinate (2) at (2.5, 1.5);
\coordinate (3) at (-2.5,1.5);
\coordinate (4) at (0,3);
\coordinate (5) at (2.5,-1.5);
\coordinate (6) at (-2.5, -1.5);
\coordinate (7) at (0, -3);

\node(node1)[minimum size = 1.3cm] at (1) {};
\node(node2)[minimum size = 1.3cm] at (2) {};
\node(node3)[minimum size = 1.3cm] at (3) {};
\node(node4)[minimum size = 1.3cm] at (4) {};
\node(node5)[minimum size = 1.3cm] at (5) {};
\node(node6)[minimum size = 1.3cm] at (6) {};
\node(node7)[minimum size = 1.3cm] at (7) {};


%Draw Lines
\draw [line width=0.8mm ] (1) -- (2) -- (4) -- (3) -- (1) -- (5) -- (7) --(6)-- (1) -- (4)--(6)--(5)-- (4)--(2)--(5)--(1)--(3)--(6)--(7)--(1)--cycle;

%Draw Nodes

\filldraw[fill=white, line width=0.8mm](1) circle (.4);

\filldraw[fill=white, line width=0.8mm](2) circle (.4)
node[above right=8pt] [xshift=4pt,yshift=-34pt] {
  \begin{tikzpicture}[scale=0.6, every node/.style={scale=0.35}, anchor=center] 
	%Set Cordinate Points and Node
\coordinate (a) at (0,0);
\node(nodea)[minimum size = 1.3cm] at (a) {};

%Include Graphics
\node(a)  {\includegraphics{../assets/images/computer.png }};
\node(a)[above= 0pt] [xshift=-4pt](1)  {\includegraphics{../assets/images/ethereum_bw.png }};
		
  \end{tikzpicture}
  };

%{\includegraphics{../assets/images/computer.jpg }};


\filldraw[fill=white, line width=0.8mm](3) circle (.4)
node[above left=8pt][xshift=-4pt,yshift=-34pt] {
  \begin{tikzpicture}[scale=0.6, every node/.style={scale=0.35}, anchor=center] 
	%Set Cordinate Points and Node
\coordinate (a) at (0,0);
\node(nodea)[minimum size = 1.3cm] at (a) {};

%Include Graphics
\node(a)  {\includegraphics{../assets/images/computer.png }};
\node(a)[above= 0pt] [xshift=-4pt](1)  {\includegraphics{../assets/images/ethereum_bw.png }};
		
  \end{tikzpicture}
  };

\filldraw[fill=white, line width=0.8mm](4) circle (.4)
node[above=2pt] {
  \begin{tikzpicture}[scale=0.6, every node/.style={scale=0.35}, anchor=center] 
	%Set Cordinate Points and Node
\coordinate (a) at (0,0);
\node(nodea)[minimum size = 1.3cm] at (a) {};

%Include Graphics
\node(a)  {\includegraphics{../assets/images/computer.png }};
\node(a)[above= 0pt] [xshift=-4pt](1)  {\includegraphics{../assets/images/ethereum_bw.png }};
		
  \end{tikzpicture}
  };
\filldraw[fill=white, line width=0.8mm](5) circle (.4)
node[below right=8pt] [xshift=4pt,yshift=40pt] {
  \begin{tikzpicture}[scale=0.6, every node/.style={scale=0.35}, anchor=center] 
	%Set Cordinate Points and Node
\coordinate (a) at (0,0);
\node(nodea)[minimum size = 1.3cm] at (a) {};

%Include Graphics
\node(a)  {\includegraphics{../assets/images/computer.png }};
\node(a)[above= 0pt] [xshift=-4pt](1)  {\includegraphics{../assets/images/ethereum_bw.png }};
		
  \end{tikzpicture}
  };

\filldraw[fill=white, line width=0.8mm](6) circle (.4)
node[below left=8pt] [xshift=-4pt,yshift=40pt] {
  \begin{tikzpicture}[scale=0.6, every node/.style={scale=0.35}, anchor=center] 
	%Set Cordinate Points and Node
\coordinate (a) at (0,0);
\node(nodea)[minimum size = 1.3cm] at (a) {};

%Include Graphics
\node(a)  {\includegraphics{../assets/images/computer.png }};
\node(a)[above= 0pt] [xshift=-4pt](1)  {\includegraphics{../assets/images/ethereum_bw.png }};
		
  \end{tikzpicture}
  };

\filldraw[fill=white, line width=0.8mm](7) circle (.4)
node[below=6pt] {
  \begin{tikzpicture}[scale=0.6, every node/.style={scale=0.35}, anchor=center] 
	%Set Cordinate Points and Node
\coordinate (a) at (0,0);
\node(nodea)[minimum size = 1.3cm] at (a) {};

%Include Graphics
\node(a)  {\includegraphics{../assets/images/computer.png }};
\node(a)[above= 0pt] [xshift=-4pt](1)  {\includegraphics{../assets/images/ethereum_bw.png }};
		
  \end{tikzpicture}
  };
		\end{tikzpicture}
		}
};


\node  at (13) {
	\scalebox{0.4}{
		\begin{tikzpicture}[scale=0.35, every node/.style={scale=0.35}]
			%Set Anchor Points
		
\coordinate (1) at (0,0);
\coordinate (2) at (2.5, 1.5);
\coordinate (3) at (-2.5,1.5);
\coordinate (4) at (0,3);
\coordinate (5) at (2.5,-1.5);
\coordinate (6) at (-2.5, -1.5);
\coordinate (7) at (0, -3);

\node(node1)[minimum size = 1.3cm] at (1) {};
\node(node2)[minimum size = 1.3cm] at (2) {};
\node(node3)[minimum size = 1.3cm] at (3) {};
\node(node4)[minimum size = 1.3cm] at (4) {};
\node(node5)[minimum size = 1.3cm] at (5) {};
\node(node6)[minimum size = 1.3cm] at (6) {};
\node(node7)[minimum size = 1.3cm] at (7) {};


%Draw Lines
\draw [line width=0.8mm ] (1) -- (2) -- (4) -- (3) -- (1) -- (5) -- (7) --(6)-- (1) -- (4)--(6)--(5)-- (4)--(2)--(5)--(1)--(3)--(6)--(7)--(1)--cycle;

%Draw Nodes

\filldraw[fill=white, line width=0.8mm](1) circle (.4);

\filldraw[fill=white, line width=0.8mm](2) circle (.4)
node[above right=8pt] [xshift=4pt,yshift=-34pt] {
  \begin{tikzpicture}[scale=0.6, every node/.style={scale=0.35}, anchor=center] 
	%Set Cordinate Points and Node
\coordinate (a) at (0,0);
\node(nodea)[minimum size = 1.3cm] at (a) {};

%Include Graphics
\node(a)  {\includegraphics{../assets/images/computer.png }};
\node(a)[above= 0pt] [xshift=-4pt](1)  {\includegraphics{../assets/images/ethereum_bw.png }};
		
  \end{tikzpicture}
  };

%{\includegraphics{../assets/images/computer.jpg }};


\filldraw[fill=white, line width=0.8mm](3) circle (.4)
node[above left=8pt][xshift=-4pt,yshift=-34pt] {
  \begin{tikzpicture}[scale=0.6, every node/.style={scale=0.35}, anchor=center] 
	%Set Cordinate Points and Node
\coordinate (a) at (0,0);
\node(nodea)[minimum size = 1.3cm] at (a) {};

%Include Graphics
\node(a)  {\includegraphics{../assets/images/computer.png }};
\node(a)[above= 0pt] [xshift=-4pt](1)  {\includegraphics{../assets/images/ethereum_bw.png }};
		
  \end{tikzpicture}
  };

\filldraw[fill=white, line width=0.8mm](4) circle (.4)
node[above=2pt] {
  \begin{tikzpicture}[scale=0.6, every node/.style={scale=0.35}, anchor=center] 
	%Set Cordinate Points and Node
\coordinate (a) at (0,0);
\node(nodea)[minimum size = 1.3cm] at (a) {};

%Include Graphics
\node(a)  {\includegraphics{../assets/images/computer.png }};
\node(a)[above= 0pt] [xshift=-4pt](1)  {\includegraphics{../assets/images/ethereum_bw.png }};
		
  \end{tikzpicture}
  };
\filldraw[fill=white, line width=0.8mm](5) circle (.4)
node[below right=8pt] [xshift=4pt,yshift=40pt] {
  \begin{tikzpicture}[scale=0.6, every node/.style={scale=0.35}, anchor=center] 
	%Set Cordinate Points and Node
\coordinate (a) at (0,0);
\node(nodea)[minimum size = 1.3cm] at (a) {};

%Include Graphics
\node(a)  {\includegraphics{../assets/images/computer.png }};
\node(a)[above= 0pt] [xshift=-4pt](1)  {\includegraphics{../assets/images/ethereum_bw.png }};
		
  \end{tikzpicture}
  };

\filldraw[fill=white, line width=0.8mm](6) circle (.4)
node[below left=8pt] [xshift=-4pt,yshift=40pt] {
  \begin{tikzpicture}[scale=0.6, every node/.style={scale=0.35}, anchor=center] 
	%Set Cordinate Points and Node
\coordinate (a) at (0,0);
\node(nodea)[minimum size = 1.3cm] at (a) {};

%Include Graphics
\node(a)  {\includegraphics{../assets/images/computer.png }};
\node(a)[above= 0pt] [xshift=-4pt](1)  {\includegraphics{../assets/images/ethereum_bw.png }};
		
  \end{tikzpicture}
  };

\filldraw[fill=white, line width=0.8mm](7) circle (.4)
node[below=6pt] {
  \begin{tikzpicture}[scale=0.6, every node/.style={scale=0.35}, anchor=center] 
	%Set Cordinate Points and Node
\coordinate (a) at (0,0);
\node(nodea)[minimum size = 1.3cm] at (a) {};

%Include Graphics
\node(a)  {\includegraphics{../assets/images/computer.png }};
\node(a)[above= 0pt] [xshift=-4pt](1)  {\includegraphics{../assets/images/ethereum_bw.png }};
		
  \end{tikzpicture}
  };
		\end{tikzpicture}
		}
};



\node at (14) {
	\scalebox{0.4}{
		\begin{tikzpicture}[scale=0.35, every node/.style={scale=0.35}]
			%Set Anchor Points
		
\coordinate (1) at (0,0);
\coordinate (2) at (2.5, 1.5);
\coordinate (3) at (-2.5,1.5);
\coordinate (4) at (0,3);
\coordinate (5) at (2.5,-1.5);
\coordinate (6) at (-2.5, -1.5);
\coordinate (7) at (0, -3);

\node(node1)[minimum size = 1.3cm] at (1) {};
\node(node2)[minimum size = 1.3cm] at (2) {};
\node(node3)[minimum size = 1.3cm] at (3) {};
\node(node4)[minimum size = 1.3cm] at (4) {};
\node(node5)[minimum size = 1.3cm] at (5) {};
\node(node6)[minimum size = 1.3cm] at (6) {};
\node(node7)[minimum size = 1.3cm] at (7) {};


%Draw Lines
\draw [line width=0.8mm ] (1) -- (2) -- (4) -- (3) -- (1) -- (5) -- (7) --(6)-- (1) -- (4)--(6)--(5)-- (4)--(2)--(5)--(1)--(3)--(6)--(7)--(1)--cycle;

%Draw Nodes

\filldraw[fill=white, line width=0.8mm](1) circle (.4);

\filldraw[fill=white, line width=0.8mm](2) circle (.4)
node[above right=8pt] [xshift=4pt,yshift=-34pt] {
  \begin{tikzpicture}[scale=0.6, every node/.style={scale=0.35}, anchor=center] 
	%Set Cordinate Points and Node
\coordinate (a) at (0,0);
\node(nodea)[minimum size = 1.3cm] at (a) {};

%Include Graphics
\node(a)  {\includegraphics{../assets/images/computer.png }};
\node(a)[above= 0pt] [xshift=-4pt](1)  {\includegraphics{../assets/images/ethereum_bw.png }};
		
  \end{tikzpicture}
  };

%{\includegraphics{../assets/images/computer.jpg }};


\filldraw[fill=white, line width=0.8mm](3) circle (.4)
node[above left=8pt][xshift=-4pt,yshift=-34pt] {
  \begin{tikzpicture}[scale=0.6, every node/.style={scale=0.35}, anchor=center] 
	%Set Cordinate Points and Node
\coordinate (a) at (0,0);
\node(nodea)[minimum size = 1.3cm] at (a) {};

%Include Graphics
\node(a)  {\includegraphics{../assets/images/computer.png }};
\node(a)[above= 0pt] [xshift=-4pt](1)  {\includegraphics{../assets/images/ethereum_bw.png }};
		
  \end{tikzpicture}
  };

\filldraw[fill=white, line width=0.8mm](4) circle (.4)
node[above=2pt] {
  \begin{tikzpicture}[scale=0.6, every node/.style={scale=0.35}, anchor=center] 
	%Set Cordinate Points and Node
\coordinate (a) at (0,0);
\node(nodea)[minimum size = 1.3cm] at (a) {};

%Include Graphics
\node(a)  {\includegraphics{../assets/images/computer.png }};
\node(a)[above= 0pt] [xshift=-4pt](1)  {\includegraphics{../assets/images/ethereum_bw.png }};
		
  \end{tikzpicture}
  };
\filldraw[fill=white, line width=0.8mm](5) circle (.4)
node[below right=8pt] [xshift=4pt,yshift=40pt] {
  \begin{tikzpicture}[scale=0.6, every node/.style={scale=0.35}, anchor=center] 
	%Set Cordinate Points and Node
\coordinate (a) at (0,0);
\node(nodea)[minimum size = 1.3cm] at (a) {};

%Include Graphics
\node(a)  {\includegraphics{../assets/images/computer.png }};
\node(a)[above= 0pt] [xshift=-4pt](1)  {\includegraphics{../assets/images/ethereum_bw.png }};
		
  \end{tikzpicture}
  };

\filldraw[fill=white, line width=0.8mm](6) circle (.4)
node[below left=8pt] [xshift=-4pt,yshift=40pt] {
  \begin{tikzpicture}[scale=0.6, every node/.style={scale=0.35}, anchor=center] 
	%Set Cordinate Points and Node
\coordinate (a) at (0,0);
\node(nodea)[minimum size = 1.3cm] at (a) {};

%Include Graphics
\node(a)  {\includegraphics{../assets/images/computer.png }};
\node(a)[above= 0pt] [xshift=-4pt](1)  {\includegraphics{../assets/images/ethereum_bw.png }};
		
  \end{tikzpicture}
  };

\filldraw[fill=white, line width=0.8mm](7) circle (.4)
node[below=6pt] {
  \begin{tikzpicture}[scale=0.6, every node/.style={scale=0.35}, anchor=center] 
	%Set Cordinate Points and Node
\coordinate (a) at (0,0);
\node(nodea)[minimum size = 1.3cm] at (a) {};

%Include Graphics
\node(a)  {\includegraphics{../assets/images/computer.png }};
\node(a)[above= 0pt] [xshift=-4pt](1)  {\includegraphics{../assets/images/ethereum_bw.png }};
		
  \end{tikzpicture}
  };
		\end{tikzpicture}
		}
};


%Arrow Empty File --> Network, Network --> Empty File

\draw [-stealth, shorten >=18pt, thick, yshift=-18pt, xshift = 8pt](-8,-1) -- (11);

\draw [stealth-, shorten >=18pt, thick, yshift=-18pt, xshift = -8pt](-4,-1) -- (11);

\draw [-stealth, shorten >=18pt, thick, yshift=-18pt, xshift = 8pt](-4,-1) -- (12);

\draw [stealth-, shorten >=18pt, thick, yshift=-18pt, xshift = -8pt](0,-1) -- (12);

\draw [-stealth, shorten >=18pt, thick, yshift=-18pt, xshift = 8pt](0,-1) -- (13);

\draw [stealth-, shorten >=18pt, thick, yshift=-18pt, xshift = -8pt](4,-1) -- (13);

\draw [-stealth, shorten >=18pt, thick, yshift=-18pt, xshift = 8pt](4,-1) -- (14);

\draw [stealth-, shorten >=18pt, thick, yshift=-18pt, xshift = -8pt](8,-1) -- (14);



%Transactions Box

\node at (-6,-6.4) {
	\scalebox{0.5}{
		\begin{tikzpicture}[scale=1, every node/.style={scale=1}]
			%Define Coordinates

%Files and mintgreen Arrows	
\coordinate (1) at (-0.5,1);
\coordinate (2) at (1.6,0);
\coordinate (3) at (-0.1,-0.7);

%Transactions

\node (state)at(1)  [scale=0.5]{\includegraphics{../assets/images/transactions.png}};
\node (state)at (2) [scale=0.5] {\includegraphics{../assets/images/transactions.png}};
\node (state)at (3) [scale=0.5] {\includegraphics{../assets/images/transactions.png}};

%Box

\draw[dashed, ultra thick] (-1.5,2) -- (2.7,2);
\draw[dashed, ultra thick] (-1.5,2) -- (-1.5,-1.7);
\draw[dashed, ultra thick] (-1.5,-1.7) -- (2.7,-1.7);
\draw[dashed, ultra thick] (2.7,2) -- (2.7,-1.7);

		\end{tikzpicture}
		}
};

\node at (-2,-6.4) {
	\scalebox{0.5}{
		\begin{tikzpicture}[scale=1, every node/.style={scale=1}]
			%Define Coordinates

%Files and mintgreen Arrows	
\coordinate (1) at (-0.5,1);
\coordinate (2) at (1.6,0);
\coordinate (3) at (-0.1,-0.7);

%Transactions

\node (state)at(1)  [scale=0.5]{\includegraphics{../assets/images/transactions.png}};
\node (state)at (2) [scale=0.5] {\includegraphics{../assets/images/transactions.png}};
\node (state)at (3) [scale=0.5] {\includegraphics{../assets/images/transactions.png}};

%Box

\draw[dashed, ultra thick] (-1.5,2) -- (2.7,2);
\draw[dashed, ultra thick] (-1.5,2) -- (-1.5,-1.7);
\draw[dashed, ultra thick] (-1.5,-1.7) -- (2.7,-1.7);
\draw[dashed, ultra thick] (2.7,2) -- (2.7,-1.7);

		\end{tikzpicture}
		}
};

\node at (2,-6.4) {
	\scalebox{0.5}{
		\begin{tikzpicture}[scale=1, every node/.style={scale=1}]
			%Define Coordinates

%Files and mintgreen Arrows	
\coordinate (1) at (-0.5,1);
\coordinate (2) at (1.6,0);
\coordinate (3) at (-0.1,-0.7);

%Transactions

\node (state)at(1)  [scale=0.5]{\includegraphics{../assets/images/transactions.png}};
\node (state)at (2) [scale=0.5] {\includegraphics{../assets/images/transactions.png}};
\node (state)at (3) [scale=0.5] {\includegraphics{../assets/images/transactions.png}};

%Box

\draw[dashed, ultra thick] (-1.5,2) -- (2.7,2);
\draw[dashed, ultra thick] (-1.5,2) -- (-1.5,-1.7);
\draw[dashed, ultra thick] (-1.5,-1.7) -- (2.7,-1.7);
\draw[dashed, ultra thick] (2.7,2) -- (2.7,-1.7);

		\end{tikzpicture}
		}
};

\node at (6,-6.4) {
	\scalebox{0.5}{
		\begin{tikzpicture}[scale=1, every node/.style={scale=1}]
			%Define Coordinates

%Files and mintgreen Arrows	
\coordinate (1) at (-0.5,1);
\coordinate (2) at (1.6,0);
\coordinate (3) at (-0.1,-0.7);

%Transactions

\node (state)at(1)  [scale=0.5]{\includegraphics{../assets/images/transactions.png}};
\node (state)at (2) [scale=0.5] {\includegraphics{../assets/images/transactions.png}};
\node (state)at (3) [scale=0.5] {\includegraphics{../assets/images/transactions.png}};

%Box

\draw[dashed, ultra thick] (-1.5,2) -- (2.7,2);
\draw[dashed, ultra thick] (-1.5,2) -- (-1.5,-1.7);
\draw[dashed, ultra thick] (-1.5,-1.7) -- (2.7,-1.7);
\draw[dashed, ultra thick] (2.7,2) -- (2.7,-1.7);

		\end{tikzpicture}
		}
};


%Text Transactions

\node at (-6,-7.9) {Transactions};
\node at (-2,-7.9) {Transactions};
\node at (2,-7.9) {Transactions};
\node at (6,-7.9) {Transactions};



%Arrows Transaction --> Network

\draw [-stealth, shorten >=18pt, shorten <=18pt, thick, ](-6,-6.4) -- (11);
\draw [-stealth, shorten >=18pt, shorten <=18pt, thick, ](-2,-6.4) -- (12);
\draw [-stealth, shorten >=18pt, shorten <=18pt, thick, ](2,-6.4) -- (13);
\draw [-stealth, shorten >=18pt, shorten <=18pt, thick, ](6,-6.4) -- (14);
	
	\end{tikzpicture}	
\end{frame}
%%%

%%%
\begin{frame}{Pros and Cons of the EVM}

\begin{minipage}{0.15\textwidth}
		\center
		\vspace{3.5em}
		\includegraphics[width=1.8cm]{../assets/images/minus.png}
		
		\vspace{3.5em}
		\uncover<2->{
		\includegraphics[width=1.8cm]{../assets/images/plus.png}}
		\vspace{0.5em}
	\end{minipage}
    \hfill
	\begin{minipage}{0.7\textwidth}
	\vspace{1em}

		\begin{itemize}
			\item<1-> EVM is slow
			\item<1-> Every full node processes all transactions
			\item<1-> Tradeoff: inclusion vs performance
			\begin{itemize}				
    				\item<1-> Verification (computation resources)
    				\item<1-> Data exchange (network resources)
  			\end{itemize}
			\item<1-> Currently 10-20 transactions per second	
		\end{itemize}
		\vspace{1em}

		\begin{itemize}
			\item<2-> Permissionless
			\item<2-> Distributed / very robust
			\item<2-> Trustless / verifiable
			\item<2-> Irreversible


		\end{itemize}
	\end{minipage}
	
	
	\uncover<3->{\begin{keytakeaway}{EVM Properties and Smart Contracts}
				The EVM is an ideal execution environment for smart contracts.
\end{keytakeaway}}				

\end{frame}
%%%

%%%
\begin{frame}{A Simple Contract in Pseudo Code}
		\vspace{1em}

\begin{minipage}{0.15\textwidth}
		\center
		\vspace{0.5em}

		\uncover<1->{\includegraphics[width=1.6cm]{../assets/images/smart_contract.png}
		{\scriptsize 0x281ad20ff212…}}

		
	\end{minipage}
    \hfill
	\begin{minipage}{0.7\textwidth}
	\uncover<1->{\begin{samplecode}{Shared Storage Contract (Pseudo Code):}
	
				store <- function(parameter)\{ \\ \text{ } \text{}
persistentStorage <- parameter \\
\} 
			\end{samplecode}	}
	
	\end{minipage}
	\vspace{1em}

	\uncover<2->{\begin{minipage}{0.35\textwidth}
		
\textbf{{Transaction}: \\
}\begin{tiny}
\textbf{Recipient Address:} 0x281ad20ff212… \\
\textbf{Nonce:} 0 \\
\textbf{Signature:} V, R and S \\
\textbf{Gas Limit:} 200000 \\
\textbf{Gas Price:} 20 \\
\textbf{Value (optional):} 0 (WEI)\\
\textbf{Data (optional):} store() function w/ \\ \text{  } \text{  } \text{  } \text{  } \text{  } \text{  } \text{  } \text{  } \text{  } \text{  } \text{ }  \text{} parameter 133
\end{tiny}
	\end{minipage}}\uncover<3->{
	\begin{minipage}{0.6\textwidth}
	\scalebox{1}{
	\begin{tikzpicture} [scale=0.5, every node/.style={scale=0.5}]
			
%Define Coordinates

\coordinate (3) at (-4,0);
\coordinate (4) at (-2,0);
\coordinate (5) at (0,0);
\coordinate (12) at (-2,-3);

%Empty Files

\node (state)at (3) [scale=0.26] {\includegraphics{../assets/images/empty_file.png}};
\node (state)at (5) [scale=0.26] {\includegraphics{../assets/images/empty_file.png}};


%Magnifier
\node (state)at (4) [scale=0.15, yshift=14cm, xshift=6cm] {\includegraphics{../assets/images/magnifier.png}};

\node (state)at (4) [scale=0.15, yshift=14cm, xshift=-6cm]{\reflectbox{\includegraphics{../assets/images/magnifier.png}}};


%Text

\node (state)at (4) [align=left, fill=mint, scale=1, yshift=2.5cm, xshift=4cm] {0x281ad20ff212… \\
Persistent Storage = 133};

\node (state)at (4) [align=left,fill=mint, scale=1, yshift=2.5cm, xshift=-4cm] {0x281ad20ff212… \\
Persistent Storage = 22};


%1s and 0 within Empty Files

\node [align=left, yshift= -8pt] at(3){\textbf{01101}\\\textbf{10010}\\\textbf{10110}};
\node [align=left, yshift= -8pt] at(5){\textbf{01010}\\\textbf{10111}\\\textbf{01100}};


%Arrow Empty File --> Empty File (mintgreen)

\node at(4)[scale=0.08] {\includegraphics{../assets/images/next_mint.png}};
	

%Network

\node  at (-2,-4.5) {
	\scalebox{1}{
		\begin{tikzpicture}[scale=0.35, every node/.style={scale=0.35}]
			%Set Anchor Points
		
\coordinate (1) at (0,0);
\coordinate (2) at (2.5, 1.5);
\coordinate (3) at (-2.5,1.5);
\coordinate (4) at (0,3);
\coordinate (5) at (2.5,-1.5);
\coordinate (6) at (-2.5, -1.5);
\coordinate (7) at (0, -3);

\node(node1)[minimum size = 1.3cm] at (1) {};
\node(node2)[minimum size = 1.3cm] at (2) {};
\node(node3)[minimum size = 1.3cm] at (3) {};
\node(node4)[minimum size = 1.3cm] at (4) {};
\node(node5)[minimum size = 1.3cm] at (5) {};
\node(node6)[minimum size = 1.3cm] at (6) {};
\node(node7)[minimum size = 1.3cm] at (7) {};


%Draw Lines
\draw [line width=0.8mm ] (1) -- (2) -- (4) -- (3) -- (1) -- (5) -- (7) --(6)-- (1) -- (4)--(6)--(5)-- (4)--(2)--(5)--(1)--(3)--(6)--(7)--(1)--cycle;

%Draw Nodes

\filldraw[fill=white, line width=0.8mm](1) circle (.4);

\filldraw[fill=white, line width=0.8mm](2) circle (.4)
node[above right=8pt] [xshift=4pt,yshift=-34pt] {
  \begin{tikzpicture}[scale=0.6, every node/.style={scale=0.35}, anchor=center] 
	%Set Cordinate Points and Node
\coordinate (a) at (0,0);
\node(nodea)[minimum size = 1.3cm] at (a) {};

%Include Graphics
\node(a)  {\includegraphics{../assets/images/computer.png }};
\node(a)[above= 0pt] [xshift=-4pt](1)  {\includegraphics{../assets/images/ethereum_bw.png }};
		
  \end{tikzpicture}
  };

%{\includegraphics{../assets/images/computer.jpg }};


\filldraw[fill=white, line width=0.8mm](3) circle (.4)
node[above left=8pt][xshift=-4pt,yshift=-34pt] {
  \begin{tikzpicture}[scale=0.6, every node/.style={scale=0.35}, anchor=center] 
	%Set Cordinate Points and Node
\coordinate (a) at (0,0);
\node(nodea)[minimum size = 1.3cm] at (a) {};

%Include Graphics
\node(a)  {\includegraphics{../assets/images/computer.png }};
\node(a)[above= 0pt] [xshift=-4pt](1)  {\includegraphics{../assets/images/ethereum_bw.png }};
		
  \end{tikzpicture}
  };

\filldraw[fill=white, line width=0.8mm](4) circle (.4)
node[above=2pt] {
  \begin{tikzpicture}[scale=0.6, every node/.style={scale=0.35}, anchor=center] 
	%Set Cordinate Points and Node
\coordinate (a) at (0,0);
\node(nodea)[minimum size = 1.3cm] at (a) {};

%Include Graphics
\node(a)  {\includegraphics{../assets/images/computer.png }};
\node(a)[above= 0pt] [xshift=-4pt](1)  {\includegraphics{../assets/images/ethereum_bw.png }};
		
  \end{tikzpicture}
  };
\filldraw[fill=white, line width=0.8mm](5) circle (.4)
node[below right=8pt] [xshift=4pt,yshift=40pt] {
  \begin{tikzpicture}[scale=0.6, every node/.style={scale=0.35}, anchor=center] 
	%Set Cordinate Points and Node
\coordinate (a) at (0,0);
\node(nodea)[minimum size = 1.3cm] at (a) {};

%Include Graphics
\node(a)  {\includegraphics{../assets/images/computer.png }};
\node(a)[above= 0pt] [xshift=-4pt](1)  {\includegraphics{../assets/images/ethereum_bw.png }};
		
  \end{tikzpicture}
  };

\filldraw[fill=white, line width=0.8mm](6) circle (.4)
node[below left=8pt] [xshift=-4pt,yshift=40pt] {
  \begin{tikzpicture}[scale=0.6, every node/.style={scale=0.35}, anchor=center] 
	%Set Cordinate Points and Node
\coordinate (a) at (0,0);
\node(nodea)[minimum size = 1.3cm] at (a) {};

%Include Graphics
\node(a)  {\includegraphics{../assets/images/computer.png }};
\node(a)[above= 0pt] [xshift=-4pt](1)  {\includegraphics{../assets/images/ethereum_bw.png }};
		
  \end{tikzpicture}
  };

\filldraw[fill=white, line width=0.8mm](7) circle (.4)
node[below=6pt] {
  \begin{tikzpicture}[scale=0.6, every node/.style={scale=0.35}, anchor=center] 
	%Set Cordinate Points and Node
\coordinate (a) at (0,0);
\node(nodea)[minimum size = 1.3cm] at (a) {};

%Include Graphics
\node(a)  {\includegraphics{../assets/images/computer.png }};
\node(a)[above= 0pt] [xshift=-4pt](1)  {\includegraphics{../assets/images/ethereum_bw.png }};
		
  \end{tikzpicture}
  };
		\end{tikzpicture}
		}
};


%Transaction Picture

\node  at (-7.5,-4.3) {
	\scalebox{0.7}
{\includegraphics{../assets/images/transactions.png}}};


%Arrow Empty File --> Network, Network --> Empty File

\draw [-stealth, shorten >=18pt, thick, yshift=-18pt, xshift = 8pt](-4,-1) -- (12);

\draw [stealth-, shorten >=18pt, thick, yshift=-18pt, xshift = -8pt](0,-1) -- (12);


%Arrow Transaction --> Network

\draw [-stealth, shorten >=40pt, shorten <=23pt, thick, ](-7.5,-4.3) -- (-2,-4.3);
		\end{tikzpicture}}
	\end{minipage}}

					

\end{frame}
%%%


%%%
\begin{frame}{Limited Computation Context}
		\vspace{1em}

		\uncover<1->{\begin{minipage}{0.3\textwidth}
		\center
		\includegraphics[scale=0.016]{../assets/images/eth.png}
		
	\begin{center}
		{\tiny e.g. ETH transactions}
	\end{center}
		
		\end{minipage}}
		\begin{minipage}{0.65\textwidth}
\begin{scriptsize}
\uncover<1->{\textbf{Native On-Chain Data}}
		\begin{itemize}
			\item<2-> Data stored on-chain and fully secured by  consensus protocol
			\item<3-> Native protocol token transactions and some endogeneous (token) contracts
			\item<4->On-chain validation
		\end{itemize}
\end{scriptsize}					
		\end{minipage}
				
				
				
				
				
				
				
				\vspace{1em}
		\uncover<5->{\begin{minipage}{0.3\textwidth}
		 \center
			\begin{tikzpicture}[scale=0.02]
			
			\node  at (0,0) {
	\scalebox{0.12}
{\includegraphics{../assets/images/fcb.png}}};			
			
\node  at (60,0) {
	\scalebox{0.12}
{\includegraphics{../assets/images/rain.png}}};	
			
			
			\end{tikzpicture}
		\begin{tiny}
\begin{center}
				  e.g. football scores \\
				  and weather data	

		\end{center}		\end{tiny}
		\end{minipage}}
		\begin{minipage}{0.65\textwidth}
\begin{scriptsize}
\uncover<5->{\textbf{Static Off-Chain Data}}
		\begin{itemize}
			\item<6-> No native on-chain representation
			\item<7-> Data can be hashed
			\item<8-> Requires trustworthy data interfaces (oracles)
		\end{itemize}
\end{scriptsize}					
		\end{minipage}
				
				\vspace{1em}
		\uncover<9->{\begin{minipage}{0.3\textwidth}
		\center
		\includegraphics[scale=0.13]{../assets/images/supply_chain.png}
\begin{tiny}
				\begin{center}
				e.g. shipment containers
				\end{center}

		\end{tiny}		\end{minipage}}
		\begin{minipage}{0.65\textwidth}
\begin{scriptsize}
\uncover<9->{\textbf{Dynamic Off-Chain Data}}
		\begin{itemize}
			\item<10-> No native on-chain representation
			\item<11-> Data cannot be hashed
			\item<12-> Requires trustworthy data interfaces (oracles) as well as reliable cryptoanchors.
		\end{itemize}
\end{scriptsize}					
		\end{minipage}			
\vspace{1em}
\end{frame}
%%%

%%%
\begin{frame}{Differences to Bitcoin Mining}

\begin{itemize}
	\item<1-> Block creation rate much faster, $\sim$ every 12-15 sec
	\item<2-> ETHASH less common and requiring more memory compared to dSHA256() $\rightarrow$  ASIC production much more expensive
			\item<3-> Ethereum is in process of switching to Proof of Stake
			\item<4-> Node Verification Process is more complex due to state verifications
		\end{itemize}

%		\vspace{3em}		
%\uncover<4->{\textbf{Full Nodes Verification Process}}
%		\begin{itemize}
%			\item<4-> Check PoW Block Header
%			\item<5-> Check if state transitions are valid %(0.1 – 0.2 seconds)
%		\end{itemize}

		
\end{frame}
%%%

%%%
\begin{frame}{Ethereum’s Tree Structure}
\vspace{1em}
\uncover<1->{
	\begin{figure}[htp]
    		\centering
    		\includegraphics[scale= 0.4]{../assets/images/buterin15_tree_structure.jpg}
      	\caption{
      		\tiny Ethereum Tree Structuce, \cite{merkleTree}
      	}
      	\label{fig:1}
    \end{figure}}
\vspace{1em}
\begin{itemize}
	\item<2-> \textbf{Transaction Root:} Very similar to Merkle tree in Bitcoin
	\item<3-> \textbf{State Root:} Merkle Patricia Tree containing the states
	\item<4-> \textbf{Receipts Root:} Essentially showing effect of transactions
\end{itemize}

\end{frame}
%%% 

%%%
\begin{frame}{The Modified Merkle Patricia Trie}
\vspace{1em}
\captionsetup[figure]{font=tiny,labelformat=empty}
\begin{figure}[htp]
\centering
\includegraphics[scale= 0.5]{../assets/images/modified_merkle_patricia_state_tree.jpg}
      \caption{{\tiny Source: \link \url{https://ethereum.stackexchange.com/questions/268/ethereum-block-architecture}}}\label{fig:2}
\end{figure}

		
\end{frame}
%%%


%%%
\begin{frame}{Ethereum Block Uncles}
\vspace{1em}
\uncover<1->{Uncle = Valid orphan Block \& parent max. 6 blocks from present}
\begin{itemize}
	\item<2-> Equivalent of Bitcoin’s orphan blocks
	\item<3-> Short block time $\rightarrow $ higher orphan rate
\end{itemize}
\uncover<3->{ Allowing Uncle inclusion...}
\begin{itemize}
			\item<4->Mitigates disadvantages from network delays
			\item<4->Encourages solo mining
\end{itemize}


%Options for upcoming Table
\setlength{\extrarowheight}{5pt}
\setlength{\aboverulesep}{0pt}
\setlength{\belowrulesep}{0pt}
\newcommand\bigstrut{\rule[-7pt]{7pt}{0pt}}


\center
\uncover<4->{
	\scalebox{0.8}{
		\begin{tabularx}{300pt}{X X} 
			\toprule
			\rowcolor{mint} \multicolumn{2}{c} 
			{\textbf{Current Reward Schedule}
			\bigstrut
			}\\
			\bottomrule
			\rowcolor{lightergray} {\footnotesize Block Reward} & {\footnotesize \hspace{1pt}\hspace{1pt}\hspace{1pt}\hspace{1pt}\hspace{1pt}\hspace{1pt}\hspace{1pt}\hspace{1pt}\hspace{1pt}\hspace{1pt}\hspace{1pt}\hspace{1pt}\hspace{1pt}\hspace{1pt}\hspace{1pt}\hspace{1pt}\hspace{1pt}\hspace{1pt}\hspace{1pt}\hspace{1pt}  2 ETH}
			\bigstrut \\
			{\footnotesize Uncle Block Reward} & {\footnotesize $\frac{7}{8}$ \hspace{1pt}  \hspace{1pt}  $*$  2 ETH = 1.75 ETH}
			\bigstrut \\


			\rowcolor{lightergray}{\footnotesize Uncle Block Inclusion} &{\footnotesize $\frac{1} {32}$ \hspace{1pt} $*$ 2 ETH = 0.0625 ETH}  
			\bigstrut \\
			\bottomrule
		\end{tabularx}
	}
\center 
\begin{tiny}
If a miner includes an uncle block, both the miner of this block and the miner of
the uncle block get rewarded.
\end{tiny}
\vspace{0.5em} }
\uncover<5->{\begin{minipage}[c]{0.05\textwidth}
\includegraphics[scale=0.5]{../assets/images/book.jpg}
\end{minipage}
\begin{minipage}[c]{0.8\textwidth}
$\rightarrow$ {\scriptsize An example: \url{https://etherscan.io/block/8652103}}
\end{minipage}}
\end{frame}
%%%


%%%
\begin{frame}%[allowframebreaks]
\frametitle{References and Recommended Reading}
	\bibliographystyle{amsplain}
	\bibliography{../assets/bib/refs}
\end{frame}
%%%



\end{document}