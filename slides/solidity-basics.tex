% Choose one to switch between slides and handout
%\documentclass[]{beamer}
\documentclass[handout]{beamer}

% Video Meta Data
\title{Smart Contracts and Decentralized Finance}
\subtitle{Solidity Basics 1}
\author{Prof. Dr. Fabian Schär}
\institute{University of Basel}

% Config File
% Packages
\usepackage[utf8]{inputenc}
\usepackage{hyperref}
\usepackage{gitinfo2}
\usepackage{tikz}
 \usetikzlibrary{calc}
\usepackage{amsmath}
\usepackage{mathtools}
\usepackage{bibentry}
\usepackage{xcolor}
\usepackage{colortbl} % Add colour to LaTeX tables
\usepackage{caption}
\usepackage[export]{adjustbox}
\usepackage{pgfplots} \pgfplotsset{compat = 1.17}
\usepackage{makecell}
\usepackage{fancybox}
\usepackage{ragged2e}
\usepackage{fontawesome}
\usepackage{seqsplit}
\usepackage{tabularx}
\usepackage{tcolorbox}
\usepackage{booktabs} % use instead  \hline in tables

% Color Options
\definecolor{highlight}{rgb}{0.65,0.84,0.82}
\definecolor{focus}{rgb}{0.72, 0, 0}
\definecolor{lightred}{rgb}{0.8,0.5,0.5}
\definecolor{midgray}{RGB}{190,195,200}

 %UniBas Main Colors
\definecolor{mint}{RGB}{165,215,210}
\definecolor{anthracite}{RGB}{45,55,60}
\definecolor{red}{RGB}{210,5,55}

 %UniBas Color Palette (for graphics)
\definecolor{strongmint}{RGB}{30,165,165}
\definecolor{darkmint}{RGB}{0,110,110}
\definecolor{softanthracite}{RGB}{140,145,150}
\definecolor{brightanthracite}{RGB}{190,195,200}
\definecolor{softred}{RGB}{235,130,155}

%Custom Colors
\definecolor{lightergray}{RGB}{230, 230, 230}



% Beamer Template Options
\beamertemplatenavigationsymbolsempty
\setbeamertemplate{footline}[frame number]
\setbeamercolor{structure}{fg=black}
\setbeamercolor{footline}{fg=black}
\setbeamercolor{title}{fg=black}
\setbeamercolor{frametitle}{fg=black}
\setbeamercolor{item}{fg=black}
\setbeamercolor{}{fg=black}
\setbeamercolor{bibliography item}{fg=black}
\setbeamercolor*{bibliography entry title}{fg=black}
\setbeamercolor{alerted text}{fg=focus}
\setbeamertemplate{items}[square]
\setbeamertemplate{enumerate items}[default]
\captionsetup[figure]{labelfont={color=black},font={color=black}}
\captionsetup[table]{labelfont={color=black},font={color=black}}

\setbeamertemplate{bibliography item}{\insertbiblabel}

%tcolor boxes
\newtcolorbox{samplecode}[2][]{
  colback=mint, colframe=darkmint, coltitle=white,
  fontupper = \ttfamily\scriptsize, fonttitle= \bfseries\scriptsize,
  boxrule = 0mm, arc = 0mm,
  boxsep = 1.3mm, left = 0mm, right = 0mm, top = 0.5mm, bottom = 0mm, middle=0mm,
  #1,title=#2}
  
\newtcolorbox{keytakeaway}[2][]{
  colback=softred, colframe=red, coltitle=white,
  fontupper = \scriptsize, fonttitle= \bfseries\scriptsize,
  boxrule = 0mm, arc = 0mm,
  boxsep = 1.3mm, left = 0mm, right = 0mm, top = 0.5mm, bottom = 0mm, middle=0mm,
  #1,title=#2}

\newtcolorbox{exercise}[2][]{
  colback=brightanthracite, colframe=anthracite, coltitle=white,
  fontupper = \scriptsize, fonttitle= \bfseries\scriptsize,
  boxrule = 0mm, arc = 0mm,
  boxsep = 1.3mm, left = 0mm, right = 0mm, top = 0.5mm, bottom = 0mm, middle=0mm,
  #1,title=#2}



% Link Icon Command 
\newcommand{\link}{%
    \tikz[x=1.2ex, y=1.2ex, baseline=-0.05ex]{%
        \begin{scope}[x=1ex, y=1ex]
            \clip (-0.1,-0.1)
                --++ (-0, 1.2)
                --++ (0.6, 0)
                --++ (0, -0.6)
                --++ (0.6, 0)
                --++ (0, -1);
            \path[draw,
                line width = 0.5,
                rounded corners=0.5]
                (0,0) rectangle (1,1);
        \end{scope}
        \path[draw, line width = 0.5] (0.5, 0.5)
            -- (1, 1);
        \path[draw, line width = 0.5] (0.6, 1)
            -- (1, 1) -- (1, 0.6);
        }
    }

% Other commands
\newcommand\tab[1][0.5cm]{\hspace*{#1}} % for code boxes


% Read Git Data from Github Actions Workflow
% Defaults to gitinfo2 for local builds
\IfFileExists{gitInfo.txt}
	{\input{gitInfo.txt}}
	{
		\newcommand{\gitRelease}{(Local Release)}
		\newcommand{\gitSHA}{\gitHash}
		\newcommand{\gitDate}{\gitAuthorIsoDate}
	}

% Custom Titlepage
\defbeamertemplate*{title page}{customized}[1][]
{
  \vspace{-0cm}\hfill\includegraphics[width=2.5cm]{../config/logo_cif}
  \includegraphics[width=1.9cm]{../config/seal_wwz}
  \\ \vspace{2em}
  \usebeamerfont{title}\textbf{\inserttitle}\par
  \usebeamerfont{title}\usebeamercolor[fg]{title}\insertsubtitle\par  \vspace{1.5em}
  \small\usebeamerfont{author}\insertauthor\par
  \usebeamerfont{author}\insertinstitute\par \vspace{2em}
  \usebeamercolor[fg]{titlegraphic}\inserttitlegraphic
    \tiny \noindent \texttt{Release Ver.: \gitRelease}\\ 
    \texttt{Version Hash: \gitSHA}\\
    \texttt{Version Date: \gitDate}\\ \vspace{1em}
    
    
    \iffalse
  \link \href{https://github.com/cifunibas/Bitcoin-Blockchain-Cryptoassets/blob/main/slides/intro.pdf}
  {Get most recent version}\\
  \link \href{https://github.com/cifunibas/Bitcoin-Blockchain-Cryptoassets/blob/main/slides/intro.pdf}
  {Watch video lecture}\\ 
  
  \fi
  
  \vspace{1em}
  License: \texttt{Creative Commons Attribution-NonCommercial-ShareAlike 4.0 International}\\\vspace{2em}
  \includegraphics[width = 1.2cm]{../config/license}
}


% tikzlibraries
\usetikzlibrary{decorations.pathreplacing}
\usetikzlibrary{decorations.markings}
\usetikzlibrary{positioning}
\usetikzlibrary{calc}
\captionsetup{font=footnotesize}

%%%%%%%%%%%%%%%%%%%%%%%%%%%%%%%%%%%%%%%%%%%%%%
%%%%%%%%%%%%%%%%%%%%%%%%%%%%%%%%%%%%%%%%%%%%%%
\begin{document}

\thispagestyle{empty}
\begin{frame}[noframenumbering]
	\titlepage
\end{frame}

%%%
\begin{frame}{What is Solidity?}

\begin{columns}[T]
	\begin{column}{0.7\textwidth}
		Solidity is \dots
			\begin{itemize}
			\item<1-> the most popular language for EVM Smart Contracts.
				\item<2-> influenced by C++, Python and JavaScript.
				\item<3-> statically typed, supports inheritance, libraries and complex user-defined types.
				\item<4-> still under rapid development with regular breaking changes.
			\end{itemize}
	\end{column}
	\begin{column}{0.3\textwidth}
	\center
			\includegraphics[scale=0.04]{../assets/images/solidity_logo_new}
	\end{column}
\end{columns}	

\vspace{1em}
	
\uncover<5->{
	\begin{keytakeaway}{Cutting edge tech}
		Use documentation frequently: \link \href{https://docs.soliditylang.org}{https://docs.soliditylang.org}
	\end{keytakeaway}
	}

\end{frame}
%%%	


%%%
\begin{frame}{Solidity Version and the Pragma}

\genkey{pragma solidity}{ }\^{}0.8.9;	

\vspace{1em}

	\begin{itemize}
		\item<1-> The pragma declares the version of the compiler to be used.
 		\item<2-> The versioning follows \link \href{https://semver.org/}{Semantic Versioning 2.0.0}: \textbf{Major.Minor.Patch}
 		\item<3-> Breaking changes only come with new minor versions. \\ {\small $\rightarrow$ Next time with 0.9.0}
		\item<4-> The \textbf{\^{}} is commonly used and means "including any future version until the next minor release". \\ % or "patch updates are allowed". 
 {$\rightarrow$ \small Everything from 0.8.9 until 0.9.0}
 		\item<5-> Use the same pragma for all files in a project!
	\end{itemize}
	
\end{frame}
%%%


%%%
\begin{frame}{Basic Contract Structure}

	\begin{samplecode}{Basic Contract Structure}
		\begin{lstlisting}[language=Solidity]
// SPDX-License-Identifier: MIT
pragma solidity ^0.8.9
contract HelloWorld {
	// This is a comment
	/* 
	This is where you write your code
	*/	
}
\end{lstlisting}
	\end{samplecode}
	\vspace{1em}
	\begin{itemize}
		\item \textbf{License identifier:} \texttt{UNLICENSED} is possible, but don't omit it.
		\item \textbf{Contract:} \texttt{\genkey{contract}{ }<ContractName> \{\}}
	\end{itemize}

\end{frame}
%%%


%%%
\begin{frame}{State Variables}

	\begin{itemize}
		\item State variables for a contract are permanently stored on the blockchain.
		\item Arbitrarily large area to store variables, but it is the most expensive operation.
		\item Complex types can be larger and usually store their length in the initial 32 bytes slot.
	\end{itemize}
	
	\begin{figure}
		\includegraphics[scale=0.2]{../assets/images/sc_storage}
\caption{\link \href{https://programtheblockchain.com/posts/2018/03/09/understanding-ethereum-smart-contract-storage/}{Understanding Ethereum Smart Contract Storage}}
		\label{fig:sc_storage}
	\end{figure}

\end{frame}
%%%

%%%
\begin{frame}{Integers}

	\begin{samplecode}{Integers}
		\begin{lstlisting}[language=Solidity]
// SPDX-License-Identifier: MIT
pragma solidity ^0.8.9;

contract HelloWorld {
  uint public answer = 42;  // After deployment, this is stored on the blockchain
}
\end{lstlisting}
	\end{samplecode}
	\begin{itemize}
		\item<2->{\typesunits{uint}, which is short for \typesunits{uint256}, is the native integer.}
		\item<3->{It is the most common type and can store a single number up to 32 bytes.}
		\item<4->{Non-negative values from \texttt{[0,$2^{256}-1$]} can be stored.}
	\end{itemize}

\end{frame}
%%%

%%%
\begin{frame}{Integers}

\begin{itemize}
 \item<1->{There exist smaller \typesunits{uints}, like \typesunits{uint8}, \typesunits{uint16}, \dots , \typesunits{uint248}, but under normal circumstances they are all converted to \typesunits{uint256}{ }before use.}
 
\item<2->{\typesunits{int}{ }resp. \typesunits{int256}{ }can store negative numbers in half the range of \typesunits{uint256}. Only use \typesunits{int}{ }when you explicitly need negative numbers. }
\end{itemize}

\end{frame}
%%%

%%%
\begin{frame}{More Elementary Types}

\typesunits{address}: Stores a 20 byte Ethereum address. \\
 	\vspace{0.5em}  
\typesunits{bool}: Can be \genkey{true}{ }or \genkey{false}{ }and is used for logical operations. \\
	\vspace{0.5em}
\typesunits{bytes1},\dots, \typesunits{bytes32}: Stores arbitrary data of a fixed size.\\
	\vspace{0.5em}
\typesunits{fixed} / \typesunits{ufixed}: Fixed point numbers are not fully supported yet. Do not use them!	\\
	\vspace{1em}
\uncover<2->{\textbf{We will look at more variable types as we use them.}}
	
\end{frame}
%%%

%%%
\begin{frame}{Let's Build an Auction Platform}

	\begin{itemize}
		\item<1->{One separate contract for each auction.}
		\item<1->{Anyone should be able to participate.}
		\item<1->{Start with a simple timed auction and explore other auction types later.}
	\end{itemize}
	
\uncover<2->{
	\begin{samplecode}{Auction Contract}
		\begin{lstlisting}[language=Solidity]
// SPDX-License-Identifier: MIT
pragma solidity ^0.8.9;

contract SimpleAuction {
  // State of the auction
  uint highestBid;
  address highestBidder;
  bool hasEnded;
}
\end{lstlisting}
	\end{samplecode}
	}
	
\end{frame}
%%%

%%%
\begin{frame}{Variable Default Values}

	\begin{table}
		\begin{tabular}{m{4cm} m{4cm}}
		\rowcolor{highlight}
			\hline
			Data Types & Zero Values\\
			\hline 
			\rowcolor{brightanthracite}
			Integer types & \texttt{0} \\
			bool &  \texttt{false} \\
			\rowcolor{brightanthracite}
			address & \texttt{0x0} \\
			bytesX & \texttt{0x0} \\
			\rowcolor{brightanthracite}
			Array & [ ] (length = 0) \\
			mapping & empty (no keys) \\
			\rowcolor{brightanthracite}
			enum & (first choice) \\
			string & "" \\
			\hline
		\end{tabular}
%		\caption{Variable default values}
%		\label{tbl:variable_default_values}
	\end{table}	
\vspace{1em}
Each variable has a default value when initialized. E.g. \typesunits{bool}{ }\texttt{hasEnded;} is the same as \typesunits{bool}{ }\texttt{hasEnded} = \genkey{false};	
	
\end{frame}
%%%

\end{document}