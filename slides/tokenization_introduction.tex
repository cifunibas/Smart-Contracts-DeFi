% Choose one to switch between slides and handout
%\documentclass[]{beamer}
\documentclass[handout]{beamer}

% Video Meta Data
\title{Smart Contracts and Decentralized Finance}
\subtitle{Intro Tokenization and ERC20}
\author{Prof. Dr. Fabian Schär}
\institute{University of Basel}

% Config File
% Packages
\usepackage[utf8]{inputenc}
\usepackage{hyperref}
\usepackage{gitinfo2}
\usepackage{tikz}
 \usetikzlibrary{calc}
\usepackage{amsmath}
\usepackage{mathtools}
\usepackage{bibentry}
\usepackage{xcolor}
\usepackage{colortbl} % Add colour to LaTeX tables
\usepackage{caption}
\usepackage[export]{adjustbox}
\usepackage{pgfplots} \pgfplotsset{compat = 1.17}
\usepackage{makecell}
\usepackage{fancybox}
\usepackage{ragged2e}
\usepackage{fontawesome}
\usepackage{seqsplit}
\usepackage{tabularx}
\usepackage{tcolorbox}
\usepackage{booktabs} % use instead  \hline in tables

% Color Options
\definecolor{highlight}{rgb}{0.65,0.84,0.82}
\definecolor{focus}{rgb}{0.72, 0, 0}
\definecolor{lightred}{rgb}{0.8,0.5,0.5}
\definecolor{midgray}{RGB}{190,195,200}

 %UniBas Main Colors
\definecolor{mint}{RGB}{165,215,210}
\definecolor{anthracite}{RGB}{45,55,60}
\definecolor{red}{RGB}{210,5,55}

 %UniBas Color Palette (for graphics)
\definecolor{strongmint}{RGB}{30,165,165}
\definecolor{darkmint}{RGB}{0,110,110}
\definecolor{softanthracite}{RGB}{140,145,150}
\definecolor{brightanthracite}{RGB}{190,195,200}
\definecolor{softred}{RGB}{235,130,155}

%Custom Colors
\definecolor{lightergray}{RGB}{230, 230, 230}



% Beamer Template Options
\beamertemplatenavigationsymbolsempty
\setbeamertemplate{footline}[frame number]
\setbeamercolor{structure}{fg=black}
\setbeamercolor{footline}{fg=black}
\setbeamercolor{title}{fg=black}
\setbeamercolor{frametitle}{fg=black}
\setbeamercolor{item}{fg=black}
\setbeamercolor{}{fg=black}
\setbeamercolor{bibliography item}{fg=black}
\setbeamercolor*{bibliography entry title}{fg=black}
\setbeamercolor{alerted text}{fg=focus}
\setbeamertemplate{items}[square]
\setbeamertemplate{enumerate items}[default]
\captionsetup[figure]{labelfont={color=black},font={color=black}}
\captionsetup[table]{labelfont={color=black},font={color=black}}

\setbeamertemplate{bibliography item}{\insertbiblabel}

%tcolor boxes
\newtcolorbox{samplecode}[2][]{
  colback=mint, colframe=darkmint, coltitle=white,
  fontupper = \ttfamily\scriptsize, fonttitle= \bfseries\scriptsize,
  boxrule = 0mm, arc = 0mm,
  boxsep = 1.3mm, left = 0mm, right = 0mm, top = 0.5mm, bottom = 0mm, middle=0mm,
  #1,title=#2}
  
\newtcolorbox{keytakeaway}[2][]{
  colback=softred, colframe=red, coltitle=white,
  fontupper = \scriptsize, fonttitle= \bfseries\scriptsize,
  boxrule = 0mm, arc = 0mm,
  boxsep = 1.3mm, left = 0mm, right = 0mm, top = 0.5mm, bottom = 0mm, middle=0mm,
  #1,title=#2}

\newtcolorbox{exercise}[2][]{
  colback=brightanthracite, colframe=anthracite, coltitle=white,
  fontupper = \scriptsize, fonttitle= \bfseries\scriptsize,
  boxrule = 0mm, arc = 0mm,
  boxsep = 1.3mm, left = 0mm, right = 0mm, top = 0.5mm, bottom = 0mm, middle=0mm,
  #1,title=#2}



% Link Icon Command 
\newcommand{\link}{%
    \tikz[x=1.2ex, y=1.2ex, baseline=-0.05ex]{%
        \begin{scope}[x=1ex, y=1ex]
            \clip (-0.1,-0.1)
                --++ (-0, 1.2)
                --++ (0.6, 0)
                --++ (0, -0.6)
                --++ (0.6, 0)
                --++ (0, -1);
            \path[draw,
                line width = 0.5,
                rounded corners=0.5]
                (0,0) rectangle (1,1);
        \end{scope}
        \path[draw, line width = 0.5] (0.5, 0.5)
            -- (1, 1);
        \path[draw, line width = 0.5] (0.6, 1)
            -- (1, 1) -- (1, 0.6);
        }
    }

% Other commands
\newcommand\tab[1][0.5cm]{\hspace*{#1}} % for code boxes


% Read Git Data from Github Actions Workflow
% Defaults to gitinfo2 for local builds
\IfFileExists{gitInfo.txt}
	{\input{gitInfo.txt}}
	{
		\newcommand{\gitRelease}{(Local Release)}
		\newcommand{\gitSHA}{\gitHash}
		\newcommand{\gitDate}{\gitAuthorIsoDate}
	}

% Custom Titlepage
\defbeamertemplate*{title page}{customized}[1][]
{
  \vspace{-0cm}\hfill\includegraphics[width=2.5cm]{../config/logo_cif}
  \includegraphics[width=1.9cm]{../config/seal_wwz}
  \\ \vspace{2em}
  \usebeamerfont{title}\textbf{\inserttitle}\par
  \usebeamerfont{title}\usebeamercolor[fg]{title}\insertsubtitle\par  \vspace{1.5em}
  \small\usebeamerfont{author}\insertauthor\par
  \usebeamerfont{author}\insertinstitute\par \vspace{2em}
  \usebeamercolor[fg]{titlegraphic}\inserttitlegraphic
    \tiny \noindent \texttt{Release Ver.: \gitRelease}\\ 
    \texttt{Version Hash: \gitSHA}\\
    \texttt{Version Date: \gitDate}\\ \vspace{1em}
    
    
    \iffalse
  \link \href{https://github.com/cifunibas/Bitcoin-Blockchain-Cryptoassets/blob/main/slides/intro.pdf}
  {Get most recent version}\\
  \link \href{https://github.com/cifunibas/Bitcoin-Blockchain-Cryptoassets/blob/main/slides/intro.pdf}
  {Watch video lecture}\\ 
  
  \fi
  
  \vspace{1em}
  License: \texttt{Creative Commons Attribution-NonCommercial-ShareAlike 4.0 International}\\\vspace{2em}
  \includegraphics[width = 1.2cm]{../config/license}
}


% tikzlibraries
\usetikzlibrary{decorations.pathreplacing}
\usetikzlibrary{decorations.markings}
\usetikzlibrary{positioning}
\usetikzlibrary{calc}
\captionsetup{font=footnotesize}

%%%%%%%%%%%%%%%%%%%%%%%%%%%%%%%%%%%%%%%%%%%%%%
%%%%%%%%%%%%%%%%%%%%%%%%%%%%%%%%%%%%%%%%%%%%%%
\begin{document}

\thispagestyle{empty}
\begin{frame}[noframenumbering]
	\titlepage
\end{frame}
%%%

%%%
\begin{frame}{Recap: Asset Layer}

\vspace{2 em}

\begin{figure}[t]
	\centering
	
	\resizebox{0.9\textwidth}{!}{
	\begin{tikzpicture}[scale=1.0, every node/.style={scale=1.0}]
			\input{../assets/figures/defi_stack_ALhighlight.tex}
	\end{tikzpicture}}
	\caption{The DeFi Stack \cite{FS:21}}
\end{figure}

\end{frame}
%%%

%%%
\begin{frame}{Cryptoasset Categories}
	\begin{minipage}{0.2\textwidth}
			\begin{center}
				\includegraphics[height=2em]{../assets/images/ethertoken}
				\includegraphics[height=2em]{../assets/images/bitcointoken}
			\end{center}
		\end{minipage}
		\begin{minipage}{0.75\textwidth}
			\textbf{Native Protocol Asset (e.g. Bitcoin or Ether):} \\
			Usually used as block reward and for the transaction fees of the corresponding protocol.
		\end{minipage}
	
		\pause
		\vspace{2 em}
		\begin{minipage}{0.2\textwidth}
			\begin{center}
				\includegraphics[height=2em]{../assets/images/green_coin}
				\includegraphics[height=2em]{../assets/images/blue_coin}
				\includegraphics[height=2em]{../assets/images/grey_coin}
			\end{center}
		\end{minipage}
		\begin{minipage}{0.75\textwidth}
			\textbf{Colored Coins (UTXO-based Standards):} \\
			Adds data to specific output strings and thereby marks them as having a purpose in addition to the original (Bitcoin) unit.
		\end{minipage}
	
		\pause
		\vspace{2 em}
		\begin{minipage}{0.2\textwidth}
			\begin{center}
				\includegraphics[height=4em]{../assets/images/CA}
			\end{center}
		\end{minipage}
		\begin{minipage}{0.75\textwidth}
			\textbf{Smart Contract-based Token Standards:} \\
			Smart contract that creates and manages a new token type by tracking states of the current owners.	
		\end{minipage}

\end{frame}
%%%

%%%
\begin{frame}{Asset Tokenization}
	\begin{figure} [h]
 		\center
			\begin{tikzpicture}[align = center, scale = 1.2]
	
	% define coordinates
	\coordinate (1) at (-4, 0);
	\coordinate (2) at (0, 0);
	\coordinate (3) at (4, 0);
	
%\node at (1) {\filldraw[color = highlight!70, fill = highlight!15, thick](6,4)circle(2pt)};
	
\node at (2) {\includegraphics[height = 3.0cm]{../assets/images/next.png}};

\node at (3) {\includegraphics[height = 3.0cm]{../assets/images/token.png}};

\end{tikzpicture}
 	\end{figure}
\end{frame}
%%%

%%%
\begin{frame}{Some Examples}
Some examples of token applications (non-exhaustive list):
	\begin{itemize}
		\item<2-> Governance tokens
		\item<3-> Utility tokens
		\item<4-> Basket / Wrapper tokens
		\item<5-> Security tokens
		\item<6-> Synthetic tokens
		\item<7-> Stablecoins
		\begin{itemize}
			\item<8-> Off-chain collateralized
			\item<9-> On-chain collateralized
			\item<10-> Pure algorithmic (no collateral)
			\item<11-> Rebase tokens
		\end{itemize}
		\item<12-> ...
 	\end{itemize}
 	
 	\uncover<13->{
 	\begin{keytakeaway}{External promises and Counterparty Risk}
Any external promises are s.t. counterparty issuer risk. The value of the token will depend on the market's expectation regarding the issuer's willingness and ability to fulfill their promise.
\end{keytakeaway}
}
\end{frame}
%%%

%%%
\begin{frame}{Token Contracts}

\begin{minipage}{0.35\textwidth}
	\begin{tikzpicture}[scale=1.0, every node/.style={scale=1.0}]
			\coordinate (1) at (1, 0.5);
\coordinate (2) at (0, 0);

	

	\node (a) at (1) {\includegraphics[height = 2cm]{../assets/images/CA}};
	
	\node (b)[circle,fill=white, minimum size=1.2cm] at (2){};
	\node (c) at (2) {\includegraphics[height = 1.2cm]{../assets/images/token}};
	\end{tikzpicture}
\end{minipage}
\begin{minipage}{0.6\textwidth}
			In essence, a smart contract-based token is \color{focus}a mapping \color{black}of accounts with token balances and a \color{focus} set of functions \color{black} that define how these balances can be changed.  
\end{minipage}

\vspace{2 em}

\begin{itemize}
	\item<2-> \textbf{Any smart contract} containing these elements can be interpreted as token contract.
	\item<3-> To facilitate usage by multiple smart contracts, \textbf{token standards} (ERC20, ERC721, etc.) have emerged.
	\item<4-> Token standards set \textbf{minimum requirements}, but do not restrict the design beyond that. 
\end{itemize}	
	

\end{frame}
%%%


%%%
\begin{frame}{ERC20 Token Standard: Functions}

A \textbf{standardized function interface} allows correspondingly programmed smart contracts to interact with any ERC20 token:
\texttt{
\begin{itemize}
	\item	function totalSupply() public view returns (uint256)
	\item	function balanceOf(address $\_$owner) public view returns (uint256 balance)
	\item	function transfer(address $\_$to, uint256 $\_$value) public returns (bool success)
	\item	function transferFrom(address $\_$from, address $\_$to, uint256 $\_$value) public returns (bool success)
	\item	function approve(address $\_$spender, uint256 $\_$value) public returns (bool success)
	\item	function allowance(address $\_$owner, address $\_$spender) public view returns (uint256 remaining)
\end{itemize}
}

\end{frame}
%%%
%%%
\begin{frame}{ERC20 Token Standard: Events and Limitations }

A \textbf{standardized event set} allow off-chain applications to work with a respective listener across multiple ERC20 tokens:
\texttt{
\begin{itemize}
	\item	event Transfer(address indexed $\_$from, address indexed $\_$to, uint256 $\_$value)
	\item	event Approval(address indexed $\_$owner, address indexed $\_$spender, uint256 $\_$value)
\end{itemize}
}

\uncover<2->{

\vspace{2 em}
As \textbf{interface standard}, ERC20 prescribes:
\begin{itemize}
	\item	a set of functions and events but \textbf{not their implementation}.
	\item	\textbf{a minimum set}. It does not limit other functionalities the contract may have.
\end{itemize}

\vspace{1 em}
\link \href{https://eips.ethereum.org/EIPS/eip-20}{ERC20 Token Standard Proposal}
}

\end{frame}
%%%



%%%
\begin{frame}{Programming the Cryptolectures Coin}

\textbf{Goal: }Create our own smart contract-based Cryptolectures Coin.
\vspace{1.5em}

\centering
	\begin{tikzpicture}[scale=1.5, every node/.style={scale=1.5}]
			\coordinate (1) at (0, 0);
\coordinate (2) at (-0.06, 0);

	
	
	\node (a) at (1){\textbf{CLC}};
	\node (b) at (2) {\includegraphics[height = 1.5cm]{../assets/images/token}};
	\end{tikzpicture}

\vspace{1em}
\begin{itemize}
	\item Simple token in line with ERC20 standard.
	\item A few nice-to-have features, such as setting a token symbol.
	\item Using a constructor to define the key variables at contract deployment.
\end{itemize}


\end{frame}
%%%

%%%
\begin{frame}{CLC: Defining the Variables}

	\begin{samplecode}{Simple ERC20: Variables}
		\begin{lstlisting}[language=Solidity]
pragma solidity ^0.8.9;

contract SimpleToken {

  // 1. Variables
  mapping (address => uint256) private balances;
  mapping (address => mapping (address => uint256)) private allowed;
  uint256 public totalSupply;
  string public name;
  uint8 public decimals;
  string public symbol;
    
...}
\end{lstlisting}
	\end{samplecode}

\end{frame}
%%%


%%%
\begin{frame}{CLC: Setting Up Events}

	\begin{samplecode}{Simple ERC20: Events}
		\begin{lstlisting}[language=Solidity]
contract SimpleToken {...

    // 2. Events
    event Transfer(
        address indexed _from,
        address indexed _to,
        uint256 value
    );

    event Approval(
        address indexed _owner,
        address indexed _spender,
        uint256 _value
    );
    
...}
\end{lstlisting}
	\end{samplecode}

\end{frame}
%%%

%%%
\begin{frame}{CLC: The Constructor}

	\begin{samplecode}{Simple ERC20: Constructor}
		\begin{lstlisting}[language=Solidity]
contract SimpleToken {...

  // 3. Constructor
  constructor (
      uint256 _initialAmount,
      string memory _tokenName,
      uint8 _decimalUnits,
      string memory _tokenSymbol
  ) {
      balances[msg.sender] = _initialAmount;
      totalSupply = _initialAmount;
      name = _tokenName;
      decimals = _decimalUnits;
      symbol = _tokenSymbol;
      emit Transfer(address(0), msg.sender, _initialAmount);
  }   
...}
\end{lstlisting}
	\end{samplecode}

\end{frame}
%%%

%%%
\begin{frame}{CLC: Transfer Own Tokens}

	\begin{samplecode}{Simple ERC20: Functions 1/3}
		\begin{lstlisting}[language=Solidity]
contract SimpleToken {...

  // 4. Functions
  function transfer(address _to, uint256 _value) public returns (bool success) {
     require(balances[msg.sender] >= _value);
     balances[msg.sender] -= _value;
     balances[_to] += _value;
     emit Transfer(msg.sender, _to, _value); 
     return true;
  }
    
...}
\end{lstlisting}
	\end{samplecode}

\end{frame}
%%%

%%%
\begin{frame}{CLC: Transfer Tokens from Allowance}

	\begin{samplecode}{Simple ERC20: Functions 2/3}
		\begin{lstlisting}[language=Solidity]
contract SimpleToken {...
  function transferFrom(address _from, address _to, uint256 _value) public returns (bool success) {
      require(allowed[_from][msg.sender] >= _value, "Insufficient allowance");
      require(balances[_from] >= _value, 
      	  "Insufficient balance");
      if (allowed[_from][msg.sender] < type(uint256).max) {
          allowed[_from][msg.sender] -= _value;
      }
      balances[_to] += _value;
      balances[_from] -= _value;
      emit Transfer(_from, _to, _value);
      return true;
  }    
  ...}
\end{lstlisting}
	\end{samplecode}

\end{frame}
%%%

%%%
\begin{frame}{CLC: Granting Allowance and Checking Balance}

	\begin{samplecode}{Simple ERC20: Functions 3/3}
		\begin{lstlisting}[language=Solidity]
contract SimpleToken {...
  function balanceOf(address _owner) public view returns (uint256 balance) {
      return balances[_owner];
  }
  function approve(address _spender, uint256 _value) public returns (bool success) {
      allowed[msg.sender][_spender] = _value;
      emit Approval(msg.sender, _spender, _value);
      return true;
  }
  function allowance(address _owner, address _spender) public view returns (uint256 remaining) {
      return allowed[_owner][_spender];
  }
}
\end{lstlisting}
	\end{samplecode}

\end{frame}
%%%


%%%
\begin{frame}{Exercise}

\begin{exercise}{YourCoin Exercise}
\begin{enumerate}
	\item	Apply the learnings from this session and program your own token contract
	\item	Deploy the contract on testnet
	\item 	Try sending tokens and granting allowances with a classmate or on your own with two different EOAs.
\end{enumerate}
\end{exercise}					

\end{frame}
%%%


%%%
\begin{frame}%[allowframebreaks]
\frametitle{References and Recommended Reading}
	\bibliographystyle{amsplain}
	\bibliography{../assets/bib/refs}
\end{frame}
%%%


\end{document}
