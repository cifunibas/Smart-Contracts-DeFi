% Choose one to switch between slides and handout
%\documentclass[]{beamer}
\documentclass[handout]{beamer}

% Video Meta Data
\title{Smart Contracts and Decentralized Finance Applications}
\subtitle{Transaction Legitimacy}
\author{Prof. Dr. Fabian Schär}
\institute{University of Basel}

% Config File
% Packages
\usepackage[utf8]{inputenc}
\usepackage{hyperref}
\usepackage{gitinfo2}
\usepackage{tikz}
 \usetikzlibrary{calc}
\usepackage{amsmath}
\usepackage{mathtools}
\usepackage{bibentry}
\usepackage{xcolor}
\usepackage{colortbl} % Add colour to LaTeX tables
\usepackage{caption}
\usepackage[export]{adjustbox}
\usepackage{pgfplots} \pgfplotsset{compat = 1.17}
\usepackage{makecell}
\usepackage{fancybox}
\usepackage{ragged2e}
\usepackage{fontawesome}
\usepackage{seqsplit}
\usepackage{tabularx}
\usepackage{tcolorbox}
\usepackage{booktabs} % use instead  \hline in tables

% Color Options
\definecolor{highlight}{rgb}{0.65,0.84,0.82}
\definecolor{focus}{rgb}{0.72, 0, 0}
\definecolor{lightred}{rgb}{0.8,0.5,0.5}
\definecolor{midgray}{RGB}{190,195,200}

 %UniBas Main Colors
\definecolor{mint}{RGB}{165,215,210}
\definecolor{anthracite}{RGB}{45,55,60}
\definecolor{red}{RGB}{210,5,55}

 %UniBas Color Palette (for graphics)
\definecolor{strongmint}{RGB}{30,165,165}
\definecolor{darkmint}{RGB}{0,110,110}
\definecolor{softanthracite}{RGB}{140,145,150}
\definecolor{brightanthracite}{RGB}{190,195,200}
\definecolor{softred}{RGB}{235,130,155}

%Custom Colors
\definecolor{lightergray}{RGB}{230, 230, 230}



% Beamer Template Options
\beamertemplatenavigationsymbolsempty
\setbeamertemplate{footline}[frame number]
\setbeamercolor{structure}{fg=black}
\setbeamercolor{footline}{fg=black}
\setbeamercolor{title}{fg=black}
\setbeamercolor{frametitle}{fg=black}
\setbeamercolor{item}{fg=black}
\setbeamercolor{}{fg=black}
\setbeamercolor{bibliography item}{fg=black}
\setbeamercolor*{bibliography entry title}{fg=black}
\setbeamercolor{alerted text}{fg=focus}
\setbeamertemplate{items}[square]
\setbeamertemplate{enumerate items}[default]
\captionsetup[figure]{labelfont={color=black},font={color=black}}
\captionsetup[table]{labelfont={color=black},font={color=black}}

\setbeamertemplate{bibliography item}{\insertbiblabel}

%tcolor boxes
\newtcolorbox{samplecode}[2][]{
  colback=mint, colframe=darkmint, coltitle=white,
  fontupper = \ttfamily\scriptsize, fonttitle= \bfseries\scriptsize,
  boxrule = 0mm, arc = 0mm,
  boxsep = 1.3mm, left = 0mm, right = 0mm, top = 0.5mm, bottom = 0mm, middle=0mm,
  #1,title=#2}
  
\newtcolorbox{keytakeaway}[2][]{
  colback=softred, colframe=red, coltitle=white,
  fontupper = \scriptsize, fonttitle= \bfseries\scriptsize,
  boxrule = 0mm, arc = 0mm,
  boxsep = 1.3mm, left = 0mm, right = 0mm, top = 0.5mm, bottom = 0mm, middle=0mm,
  #1,title=#2}

\newtcolorbox{exercise}[2][]{
  colback=brightanthracite, colframe=anthracite, coltitle=white,
  fontupper = \scriptsize, fonttitle= \bfseries\scriptsize,
  boxrule = 0mm, arc = 0mm,
  boxsep = 1.3mm, left = 0mm, right = 0mm, top = 0.5mm, bottom = 0mm, middle=0mm,
  #1,title=#2}



% Link Icon Command 
\newcommand{\link}{%
    \tikz[x=1.2ex, y=1.2ex, baseline=-0.05ex]{%
        \begin{scope}[x=1ex, y=1ex]
            \clip (-0.1,-0.1)
                --++ (-0, 1.2)
                --++ (0.6, 0)
                --++ (0, -0.6)
                --++ (0.6, 0)
                --++ (0, -1);
            \path[draw,
                line width = 0.5,
                rounded corners=0.5]
                (0,0) rectangle (1,1);
        \end{scope}
        \path[draw, line width = 0.5] (0.5, 0.5)
            -- (1, 1);
        \path[draw, line width = 0.5] (0.6, 1)
            -- (1, 1) -- (1, 0.6);
        }
    }

% Other commands
\newcommand\tab[1][0.5cm]{\hspace*{#1}} % for code boxes


% Read Git Data from Github Actions Workflow
% Defaults to gitinfo2 for local builds
\IfFileExists{gitInfo.txt}
	{\input{gitInfo.txt}}
	{
		\newcommand{\gitRelease}{(Local Release)}
		\newcommand{\gitSHA}{\gitHash}
		\newcommand{\gitDate}{\gitAuthorIsoDate}
	}

% Custom Titlepage
\defbeamertemplate*{title page}{customized}[1][]
{
  \vspace{-0cm}\hfill\includegraphics[width=2.5cm]{../config/logo_cif}
  \includegraphics[width=1.9cm]{../config/seal_wwz}
  \\ \vspace{2em}
  \usebeamerfont{title}\textbf{\inserttitle}\par
  \usebeamerfont{title}\usebeamercolor[fg]{title}\insertsubtitle\par  \vspace{1.5em}
  \small\usebeamerfont{author}\insertauthor\par
  \usebeamerfont{author}\insertinstitute\par \vspace{2em}
  \usebeamercolor[fg]{titlegraphic}\inserttitlegraphic
    \tiny \noindent \texttt{Release Ver.: \gitRelease}\\ 
    \texttt{Version Hash: \gitSHA}\\
    \texttt{Version Date: \gitDate}\\ \vspace{1em}
    
    
    \iffalse
  \link \href{https://github.com/cifunibas/Bitcoin-Blockchain-Cryptoassets/blob/main/slides/intro.pdf}
  {Get most recent version}\\
  \link \href{https://github.com/cifunibas/Bitcoin-Blockchain-Cryptoassets/blob/main/slides/intro.pdf}
  {Watch video lecture}\\ 
  
  \fi
  
  \vspace{1em}
  License: \texttt{Creative Commons Attribution-NonCommercial-ShareAlike 4.0 International}\\\vspace{2em}
  \includegraphics[width = 1.2cm]{../config/license}
}


% tikzlibraries
\usetikzlibrary{decorations.pathreplacing}
\usetikzlibrary{decorations.markings}
\usetikzlibrary{positioning}
\usetikzlibrary{calc}
\captionsetup{font=footnotesize}

%%%%%%%%%%%%%%%%%%%%%%%%%%%%%%%%%%%%%%%%%%%%%%
%%%%%%%%%%%%%%%%%%%%%%%%%%%%%%%%%%%%%%%%%%%%%%
\begin{document}

\thispagestyle{empty}
\begin{frame}[noframenumbering]
	\titlepage
\end{frame}

%%%
\begin{frame}{Transaction Legitimacy}
\textbf{Goal:} Ensure transaction \color{focus}authenticity \color{black}and \color{focus} integrity\color{black}, i.e., ensure that the transaction was initiated by its owner of the funds and has not been changed.
\uncover<2->{
		\center
		\input{../assets/figures/transaction-legit.tex}
		}
\uncover<3->{
		\vspace{2em}
		\input{../assets/figures/transaction-manipulation.tex}
	}
\end{frame}
%%%

%%%
\begin{frame}{Public/Private Key Pair}
	\input{../assets/figures/key_creation}

\vspace{1em}

\textbf{Two key principles:}
\begin{enumerate}
\item<1-> Private key is created (chosen) without the help of an intermediary, and can be used to derive public key.
\item<2-> If information is encrypted with one key, it can only be decrypted with the other key.
\end{enumerate}

\vspace{1em}

\uncover<3->{
\textbf{Key assumption:}\\
Asymmetric cryptography and its applications assume that you are willing and able to keep your private key secret. }
\end{frame}
%%%

%%%
\begin{frame}{Two Distinct Applications}
	\input{../assets/figures/publickeycrypto_applications}
\end{frame}
%%%


%%%
\begin{frame}{Transaction Legitimacy}
	\begin{figure}[h!]
		\center
		\input{../assets/figures/transaction-encryption.tex}
		\caption*{Encryption and decryption of the transaction message}
		\label{fig:asymmeinfach}
	\end{figure}
\end{frame}
%%%

%%%
\begin{frame}{What Is a Hash Function?}

Deterministic algorithm (\color{focus}function, $H()$\color{black}) that maps data of quasi-arbitrary size (\color{focus}pre-image, $m$\color{black}) to fixed-length bit string (\color{focus}hash value, $h$\color{black}).

	\begin{align}
		h = H(m)
		\label{eq:hash_function}
	\end{align}

\vspace{1.5em}
	
\uncover<2->{
\textbf{Application fields:} (non-exhaustive)
	\begin{itemize}
		\item Data protection
		\item Verification and authentication
		\item Proof-of-work
		\item Data lookup optimization
		\item Error detection
	\end{itemize}
}	
	
\end{frame}
%%%

%%%
\begin{frame}{Cryptographic Hash Functions}

Additional criteria:
	\begin{enumerate}
		\item Approximately uniform hash value distribution.
		\item Quick to compute for any given pre-image.
		\item Trap-door: Infeasible to generate pre-image from hash value.
		\item Avalanche effect: Small change in input results in totally different output.
		\item Very low collision probability: Unlikely that two pre-images generate the same hash value.
	\end{enumerate}
	\vspace{1em}

\uncover<2->{In Bitcoin context, the functions \color{focus}\texttt{SHA2.256} \color{black} (in short \texttt{SHA256}) and \color{focus}\texttt{RIPEMD160} \color{black}are used. Both satisfy the above criteria.}
	
\end{frame}
%%%	

%%%
\begin{frame}{Avalanche Effect with \texttt{SHA2}}

\begin{center}
$\texttt{SHA256}(<$\textit{pre-image}$>)$
\end{center}

This is the pre-image

$\Rightarrow$ \footnotesize 5e13fffedd642aaddea7872463fd44d8b8a336cb822ebd4c12d9e6282c88cea8 \normalsize
\vspace{1em}

\color{focus}t\color{black}his is the pre-image

$\Rightarrow$ \footnotesize \color{focus}aba56d7c281f8d\color{black}a\color{focus}cb1076baed182538d2ce21494075f7377982bf9d5d0\color{black}8\color{focus}e6ccf\color{black} \normalsize
\vspace{1em}

\begin{columns}[T]
	\begin{column}{0.35\textwidth}
		\includegraphics[width = 4 cm, frame]{../assets/images/manual_hashing_video.png}
	\end{column} %\hfill
	\begin{column}{0.65\textwidth}
		\begin{itemize}
			\item Nonlinarity due to choice, majority, mod, rotation and shifting operations.
			\item Efficient for computers vs. 0.67 hashes / day by hand.
			\item \link \href{https://www.youtube.com/watch?v=y3dqhixzGVo}{Video: Hash value by hand}.
		\end{itemize}
	\end{column}
\end{columns}
	
\end{frame}
%%%	

%%%
\begin{frame}{Low Collision Probability}

Due to the fixed size of $h$, the corresponding hash function $H()$ can only produce a finite set of distinct hashes.
\vspace{1em}

\texttt{SHA256}:
	\begin{itemize}
		\item All possible combinations of 256 Bits, i.e., $2^{256}$ $h$.
		\item In base 10, the number of combinations corresponds to 115,792,089,237,316,195,423,570,985,008,687,907,853,269,\\984,665,640,564,039,457,584,007,913,129,639,936
	\end{itemize}
\vspace{1em}

\uncover<2->{Probability of a hexadecimal hash with certain characteristics:\\
\vspace{0.75em}
\tiny 
$P(h \leq \color{focus}0\color{black}FFFFFFFFFFFFFFFFFFFFFFFFFFFFFFFFFFFFFFFFFFFFFFFFFFFFFFFFFFFFFFF) = \dfrac{1}{16}$\\
\vspace{0.75em}
$P(h \leq \color{focus}00\color{black}FFFFFFFFFFFFFFFFFFFFFFFFFFFFFFFFFFFFFFFFFFFFFFFFFFFFFFFFFFFFFF) = \left( \dfrac{1}{16} \right)^{2} $
\normalsize}
	
\end{frame}
%%%	

%%%
\begin{frame}{Example of an Elliptic Curve}

	\begin{columns}
		\begin{column}{0.5\textwidth}
			\begin{figure}
			\input{../assets/figures/elliptic.tex}
				\caption*{Elliptic curve with $a = -2$, $b = 2$}
			\end{figure}	
		\end{column}
		\begin{column}{0.5\textwidth}
			Weierstrass equation:\\
			\vspace{1em}
			$y^2 = x^3 + ax + b$\\
			\vspace{2em}
			Non-singularity condition:\\
			\vspace{1em}
			$4a^3 + 27b^2 \neq 0$
		\end{column}
	\end{columns}
	
\end{frame}
%%%

%%%
\begin{frame}{Elliptic Curves over Finite Fields}
	Bitcoin uses \texttt{secp256k1}:\ $y^2 = x^3 + 7\ (mod\ p)$ over ${\rm I\!F}_p$ where $p = 2^{256} - 2^{32} - 2^9 - 2^6 - 2^4 - 1$.\\
	\vspace{1em}
	\uncover<2->{\textbf{Simplified example:} $y^2 = x^3 + 7\ (mod\ 37)$ in ${\rm I\!F}_{37}$ \uncover<3->{with \textcolor{highlight}{$x = 5$}}}
	\begin{columns}[T]
		\begin{column}{0.5\textwidth}
			\uncover<2->{\begin{figure}
				\input{../assets/figures/elliptic_finite.tex}
				\end{figure}
			}
		\end{column}
		\begin{column}{0.5\textwidth}
			\begin{align*}
					\uncover<4->{\textcolor{focus}{y}^2\ (mod\ 37) &\equiv \textcolor{highlight}{x}^3 + 7\ (mod\ 37)\\
					&\dots }\\
					\uncover<5->{\textcolor{highlight}{5}^3 + 7\ (mod\ 37) &= 132\ (mod\ 37)\\
					&= 21\\
					&\dots }\\
					\uncover<6->{\textcolor{focus}{y}^2\ (mod\ 37) &\equiv 21}\\
					\uncover<7->{\textcolor{focus}{13}^2\ (mod\ 37) &\equiv 21\\			
					\textcolor{focus}{24}^2\ (mod\ 37) &\equiv 21}
				\end{align*}
		\end{column}
	\end{columns}

\end{frame}
%%%




\begin{frame}{Bullets and Uncover Animations}

This is how lists can be used in the template. \\ \vspace{1em}

Some of the main features of this template:
	
	\begin{itemize}
		\item<1-> stay focused
		\item<2-> save ink
		\item<3-> do not get distracted
	\end{itemize}

	\vspace{1em}	
	\uncover<4->{The same works for Enumerate environments:}

	\begin{enumerate}
		\item<5-> This is the first point
		\item<6-> and the second one
		\item<7-> Last but not least, the third one
	\end{enumerate}
	
\end{frame}
%%%	

%%%
\begin{frame}{Tables}
	\begin{table}
		\begin{tabular}{ll}
			A & B\\
			C & D
		\end{tabular}
		\caption{This is a table}
		\label{tbl:simpletable}
	\end{table}
\end{frame}

%%%

%%%
\begin{frame}{Block Conventions I}
	
	\begin{columns}[T]
		\begin{column}{0.45\textwidth}
			\vspace{-1.0em}
			\begin{exampleblock}{Basic Contract Structure (Solidity)}
	 			\texttt{
					Pragma solidity  \textasciicircum 0.4.19; \
					contract HelloWorld \{ \\
					\color{softanthracite} /* This is where you write your code */
	 			}
			\end{exampleblock}			
		\end{column} %\hfill
		\begin{column}{0.5\textwidth}
			\textbf{Sample Code Box}\
			Contains code snippets in R, Solidity \& pseudo code.
		\end{column}
	\end{columns}

\vspace{2.0em}

	\begin{columns}[T]
		\begin{column}{0.45\textwidth}
			\vspace{-1.0em}
			\begin{alertblock}{Transaction Definitions}
				A transaction is a message send by an externally owned account.
			\end{alertblock}		
		\end{column}
		\begin{column}{0.5\textwidth}
			\textbf{Key Take-Away}\\
			Highlights important concepts and definitions.
		\end{column}
	\end{columns}

\vspace{2.0em}

	\begin{columns}[T]
		\begin{column}{0.45\textwidth}
			\vspace{-1.0em}
			\begin{block}{Exercise 1}
				Get Ropsten Ether from the faucet and deploy your first smart contract on Ropsten Testnet.
			\end{block}		
		\end{column}
		\begin{column}{0.5\textwidth}
		\textbf{Exercises}\\
			Things for you to try out.
		\end{column}
	\end{columns}

\end{frame}

%%%


%%%
\begin{frame}{Equations}

Some in-line math can be used like this $y=x^2+2x-5$. \\ \vspace{1em}

However, in many cases it makes sense to use numbered equations. The $\&$-sign can be used to set anchor points, e.g. around the ``$=$'' for multiple equations. This is shown below in Equation \eqref{eq:firststep} and Equation \eqref{eq:secondstep}.
	\begin{align}
		y + 2x &= x^2+2x-5 \label{eq:firststep}\\
		y &= x^2-5 \label{eq:secondstep}
	\end{align}
\end{frame}
%%%

%%%
\begin{frame}{Figures}
	\begin{figure}
		\center
		\includegraphics[width=0.8\textwidth]{../config/logo_cif}	
		\caption{Center for Innovative Finance Logo}
		\label{fig:logo}
	\end{figure}
\end{frame}
%%%

%%%
\begin{frame}{Links}
		Links are signaled with the respective icon, which is set up in the config file and can be called with the respective command.\\
		
		\link \href{https://github.com/cifunibas}{Repository: CIF on GitHub}.
		
\end{frame}
%%%


%%%
\begin{frame}{Labels and (Non-Bib) References}
	In some cases in can make sense to reference to a figure, table or equation on a different slide. Use \texttt{$\backslash$pageref\{$<$label$>$\}} if you want to highlight, that there is a related resource on slide \pageref{fig:logo}.\\ \vspace{1em}
	
	Do - under no circumstances - hardcode references.	
\end{frame}
%%%

%%%
\begin{frame}{Bib-References}
		Read the Bitcoin Whitepaper, \cite{nakamotoBitcoin2008}.
\end{frame}
%%%

%%%
\begin{frame}%[allowframebreaks]
\frametitle{References and Recommended Reading}
	\bibliographystyle{amsplain}
	\bibliography{../assets/bib/refs}
\end{frame}
%%%



\end{document}