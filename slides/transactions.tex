% Choose one to switch between slides and handout
%\documentclass[]{beamer}
\documentclass[handout]{beamer}

% Video Meta Data
\title{Smart Contracts and Decentralized Finance}
\subtitle{Transactions}
\author{Prof. Dr. Fabian Schär}
\institute{University of Basel}

% Config File
% Packages
\usepackage[utf8]{inputenc}
\usepackage{hyperref}
\usepackage{gitinfo2}
\usepackage{tikz}
 \usetikzlibrary{calc}
\usepackage{amsmath}
\usepackage{mathtools}
\usepackage{bibentry}
\usepackage{xcolor}
\usepackage{colortbl} % Add colour to LaTeX tables
\usepackage{caption}
\usepackage[export]{adjustbox}
\usepackage{pgfplots} \pgfplotsset{compat = 1.17}
\usepackage{makecell}
\usepackage{fancybox}
\usepackage{ragged2e}
\usepackage{fontawesome}
\usepackage{seqsplit}
\usepackage{tabularx}
\usepackage{tcolorbox}
\usepackage{booktabs} % use instead  \hline in tables

% Color Options
\definecolor{highlight}{rgb}{0.65,0.84,0.82}
\definecolor{focus}{rgb}{0.72, 0, 0}
\definecolor{lightred}{rgb}{0.8,0.5,0.5}
\definecolor{midgray}{RGB}{190,195,200}

 %UniBas Main Colors
\definecolor{mint}{RGB}{165,215,210}
\definecolor{anthracite}{RGB}{45,55,60}
\definecolor{red}{RGB}{210,5,55}

 %UniBas Color Palette (for graphics)
\definecolor{strongmint}{RGB}{30,165,165}
\definecolor{darkmint}{RGB}{0,110,110}
\definecolor{softanthracite}{RGB}{140,145,150}
\definecolor{brightanthracite}{RGB}{190,195,200}
\definecolor{softred}{RGB}{235,130,155}

%Custom Colors
\definecolor{lightergray}{RGB}{230, 230, 230}



% Beamer Template Options
\beamertemplatenavigationsymbolsempty
\setbeamertemplate{footline}[frame number]
\setbeamercolor{structure}{fg=black}
\setbeamercolor{footline}{fg=black}
\setbeamercolor{title}{fg=black}
\setbeamercolor{frametitle}{fg=black}
\setbeamercolor{item}{fg=black}
\setbeamercolor{}{fg=black}
\setbeamercolor{bibliography item}{fg=black}
\setbeamercolor*{bibliography entry title}{fg=black}
\setbeamercolor{alerted text}{fg=focus}
\setbeamertemplate{items}[square]
\setbeamertemplate{enumerate items}[default]
\captionsetup[figure]{labelfont={color=black},font={color=black}}
\captionsetup[table]{labelfont={color=black},font={color=black}}

\setbeamertemplate{bibliography item}{\insertbiblabel}

%tcolor boxes
\newtcolorbox{samplecode}[2][]{
  colback=mint, colframe=darkmint, coltitle=white,
  fontupper = \ttfamily\scriptsize, fonttitle= \bfseries\scriptsize,
  boxrule = 0mm, arc = 0mm,
  boxsep = 1.3mm, left = 0mm, right = 0mm, top = 0.5mm, bottom = 0mm, middle=0mm,
  #1,title=#2}
  
\newtcolorbox{keytakeaway}[2][]{
  colback=softred, colframe=red, coltitle=white,
  fontupper = \scriptsize, fonttitle= \bfseries\scriptsize,
  boxrule = 0mm, arc = 0mm,
  boxsep = 1.3mm, left = 0mm, right = 0mm, top = 0.5mm, bottom = 0mm, middle=0mm,
  #1,title=#2}

\newtcolorbox{exercise}[2][]{
  colback=brightanthracite, colframe=anthracite, coltitle=white,
  fontupper = \scriptsize, fonttitle= \bfseries\scriptsize,
  boxrule = 0mm, arc = 0mm,
  boxsep = 1.3mm, left = 0mm, right = 0mm, top = 0.5mm, bottom = 0mm, middle=0mm,
  #1,title=#2}



% Link Icon Command 
\newcommand{\link}{%
    \tikz[x=1.2ex, y=1.2ex, baseline=-0.05ex]{%
        \begin{scope}[x=1ex, y=1ex]
            \clip (-0.1,-0.1)
                --++ (-0, 1.2)
                --++ (0.6, 0)
                --++ (0, -0.6)
                --++ (0.6, 0)
                --++ (0, -1);
            \path[draw,
                line width = 0.5,
                rounded corners=0.5]
                (0,0) rectangle (1,1);
        \end{scope}
        \path[draw, line width = 0.5] (0.5, 0.5)
            -- (1, 1);
        \path[draw, line width = 0.5] (0.6, 1)
            -- (1, 1) -- (1, 0.6);
        }
    }

% Other commands
\newcommand\tab[1][0.5cm]{\hspace*{#1}} % for code boxes


% Read Git Data from Github Actions Workflow
% Defaults to gitinfo2 for local builds
\IfFileExists{gitInfo.txt}
	{\input{gitInfo.txt}}
	{
		\newcommand{\gitRelease}{(Local Release)}
		\newcommand{\gitSHA}{\gitHash}
		\newcommand{\gitDate}{\gitAuthorIsoDate}
	}

% Custom Titlepage
\defbeamertemplate*{title page}{customized}[1][]
{
  \vspace{-0cm}\hfill\includegraphics[width=2.5cm]{../config/logo_cif}
  \includegraphics[width=1.9cm]{../config/seal_wwz}
  \\ \vspace{2em}
  \usebeamerfont{title}\textbf{\inserttitle}\par
  \usebeamerfont{title}\usebeamercolor[fg]{title}\insertsubtitle\par  \vspace{1.5em}
  \small\usebeamerfont{author}\insertauthor\par
  \usebeamerfont{author}\insertinstitute\par \vspace{2em}
  \usebeamercolor[fg]{titlegraphic}\inserttitlegraphic
    \tiny \noindent \texttt{Release Ver.: \gitRelease}\\ 
    \texttt{Version Hash: \gitSHA}\\
    \texttt{Version Date: \gitDate}\\ \vspace{1em}
    
    
    \iffalse
  \link \href{https://github.com/cifunibas/Bitcoin-Blockchain-Cryptoassets/blob/main/slides/intro.pdf}
  {Get most recent version}\\
  \link \href{https://github.com/cifunibas/Bitcoin-Blockchain-Cryptoassets/blob/main/slides/intro.pdf}
  {Watch video lecture}\\ 
  
  \fi
  
  \vspace{1em}
  License: \texttt{Creative Commons Attribution-NonCommercial-ShareAlike 4.0 International}\\\vspace{2em}
  \includegraphics[width = 1.2cm]{../config/license}
}


% tikzlibraries
\usetikzlibrary{decorations.pathreplacing}
\usetikzlibrary{decorations.markings}
\usetikzlibrary{positioning}
\usetikzlibrary{calc}
\captionsetup{font=footnotesize}

%%%%%%%%%%%%%%%%%%%%%%%%%%%%%%%%%%%%%%%%%%%%%%
%%%%%%%%%%%%%%%%%%%%%%%%%%%%%%%%%%%%%%%%%%%%%%
\begin{document}

\thispagestyle{empty}
\begin{frame}[noframenumbering]
	\titlepage
\end{frame}

%%%
\begin{frame}{Transactions in Ethereum}
A transaction is a message, sent from an externally owned account.
\vspace{1em}

Transactions can modify states:
	\begin{itemize}
		\item<1-> Ether balances of account
		\item<2-> Contract account's storage
	\end{itemize}
\vspace{1em}

Transactions optionally contain:
	\begin{itemize}
		\item<1-> Ether
		\item<2-> Data
	\end{itemize}
\vspace{1em}

\begin{tikzpicture}[align = center, scale = 1.2]
	
	% define coordinates
	\coordinate	(0) at (-3, 0.7);	
	\coordinate (1) at (-3, 0);
	\coordinate (2) at (0, 0);
	\coordinate (3) at (3, 0);
	
	
	% adjust fontsize
	\begin{footnotesize}	
	
	% Title and background
%	\uncover<5->{
	\filldraw [fill = mint, draw = mint] (-4.4,-1.6)rectangle ++(9,2.6);
	\node [right = - 45pt] at (0)[darkmint] {\textbf{Transactions by Recipient}};
	
	
	% EOA to EOA
	\node[left] at (1) {\includegraphics[height = 0.12\textheight]{../assets/images/EOA}};
	\node[right] at (1) {\includegraphics[height = 0.12\textheight]{../assets/images/EOA}};
	\node[below = 18pt] at (1) {\textbf{EOA to EOA}\\
	ETH transfer};
%}


	% EOA to CA
%	\uncover<6->{
		\node[left] at (2) {\includegraphics[height = 0.12\textheight]{../assets/images/EOA}};
		\node[right] at (2) {\includegraphics[height = 0.12\textheight]{../assets/images/CA}};
		\node[below = 18pt] at (2) {\textbf{EOA to CA}\\
	ETH transfer \\
	Contract execution	};
%	}
	
	% EOA to 0
%	\uncover<7->{
		\node[left] at (3) {\includegraphics[height = 0.12\textheight]{../assets/images/EOA}};
		\node[right] at (3) {\includegraphics[height = 0.12\textheight]{../assets/images/zero}};
		\node[below = 18pt] at (3) {\textbf{EOA to 0} \\
		Contract deployment};
%	}
	
	% Close fontsize
	\end{footnotesize}
	
\end{tikzpicture}
% insert Transactions by Recipient figure here.

\end{frame}
%%%

%%%
\begin{frame}{EOA to EOA}
\center
\includegraphics[width=8cm]{../assets/images/EOA_to_EOA}

\end{frame}
%%%

%%%
\begin{frame}{EOA to CA}
\center
\includegraphics[width=8cm]{../assets/images/EOA_to_CA}

\end{frame}
%%%

%%%
\begin{frame}{EOA to 0}
\center
\includegraphics[width=10cm]{../assets/images/EOA_to_0}


\end{frame}
%%%

%%%
\begin{frame}{Internal Transactions}
\begin{columns}[T]
	\begin{column}{0.35\textwidth}
		\includegraphics[width=3cm]{../assets/images/CA}
		\end{column} %\hfill
	\begin{column}{0.65\textwidth}
		\begin{alertblock}{Contract Accounts Send Internal Transactions}
				CA can only send internal transactions. They may react to incoming transactions/messages by sending internal transactions and pass on information to other CA. $\rightarrow$ Initially, CA must be triggered by an EOA transaction.
		\end{alertblock}
	\end{column}
\end{columns}

\end{frame}
%%%

%%%
\begin{frame}{Transactions, Internal Transactions and Calls}
\textbf{Internal Transaction:} This is contract-to-contract. These are not delayed by mining because they are part of the transaction execution.
\vspace{1em}

\textbf{Transaction:} Originates with an externally owned account. It's always a transaction that gets things started, but multiple internal transactions may be fired off to complete the action.
\vspace{1em}

\textbf{Call:} Transaction is executed locally on the user's local machine which only evaluates the result. These are read-only and fast. They can't change the blockchain in any way because they are never sent to the network. 


%Source: \cite{ethereum.stackexchange...}.
\vspace{1em}

\begin{alertblock}{Contract Call}
				However: The term "contract call" is generally used for EOA to CA transactions.
			\end{alertblock}

\end{frame}
%%%

%%%
\begin{frame}{A Detailed Look at Ethereum Transactions}
Ethereum Transactions contain:
	\begin{itemize}
		\item<1-> \textbf{Recipient Address:} The recipient's hexadecimal 		address.
		\item<2-> \textbf{Nonce:} Transaction count from sender.
		\item<3-> \textbf{Signature:} Consisting of the variables V, R and S.
		\item<4-> \textbf{Gas Limit:} Maximum number of transaction execution steps.
		\item<5-> \textbf{Gas Price:} Fee the sender is willing to pay per execution step. 
		\item<6-> \textbf{Value (optional):} Ether value in Wei, where $10^{18}$ Wei = 1 ETH. 
		\item<7-> \textbf{Data (optional):} Contract execution or deployment instructions.
	\end{itemize}

\end{frame}
%%%


\end{document}